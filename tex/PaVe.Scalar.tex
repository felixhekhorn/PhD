We focus on:
\begin{equation}
\Pggx(q) + \Pg(k_1) \rightarrow \PQ(p_1)+\PaQ(p_2)
\end{equation}
\begin{equation}
k_1^2 = 0 \quad p_1^2 = p_2^2 = m^2 \quad (p_1+p_2)^2=s\quad (p_2-q)^2=t\quad (p_1-q)^2=u
\end{equation}

define some shortcuts
\begin{align}
0\leq&\rho = \frac {4m^2} s\leq 1 &0\leq&\beta = \sqrt{1-\rho}\leq 1 &0\leq&\chi = \frac{1-\beta}{1+\beta}\leq 1\\
&\rho_q = \frac {4m^2} {q^2}\leq 0 &1\leq&\beta_q = \sqrt{1-\rho_q} &0\leq&\chi_q = -\frac{1-\beta_q}{1+\beta_q}\leq 1
\end{align}

\subsection[One-Point Function A0]{One-Point Function $A_0$}
\cite{Denner:2005nn}:
\begin{align}
A_0(m^2)&=-\frac{i}{16\pi^2}m^2\left(\frac{m^2}{4\pi\mu^2}\right)^{(n-4)/2}\Gamma(1-n/2)\\
 &= \frac{im^2}{16\pi^2}\left(\Delta-\log(m^2/\mu^2)+1\right) + O(n-4)\\
 &= iC_\epsilon m^2 \left(-\frac 2 \epsilon+1\right) + O(n-4)\\
\Delta &= \frac 2 {4-n}-\gamma_E+\log(4\pi)\\
C_\epsilon &= \frac 1 {16\pi^2}\exp\left(\left(\gamma_E-\log(4\pi)+\log\left(m^2/\mu^2\right)\right)\frac{\epsilon} 2\right)
\end{align}
this is \textit{up to order} $O(n-4)$ in accordance with \cite{Bojak:2000eu}\cite{PhysRevD.40.54}, but NOT beyond - see also \cite[eq. (A.12)]{Bojak:2000eu}. So we can treat $C_\epsilon$ and $\Delta$ as equal.

\subsection[Two-Point Function B0]{Two-Point Function $B_0$}
In \cite[eq. (4.23)]{Denner:2005nn} is a generic function given and we end up with
\begin{align}
B_0(s,m^2,m^2) &= iC_\epsilon \left(-\frac 2 \epsilon+2+\beta\log(\chi)\right)\\
B_0(q^2,m^2,m^2) &= iC_\epsilon \left(-\frac 2 \epsilon+2+\beta_q\log(\chi_q)\right)\\
B_0(0,m^2,m^2) &= iC_\epsilon \left(-\frac 2 \epsilon\right)\\
B_0(m^2,0,m^2) &= iC_\epsilon \left(-\frac 2 \epsilon+2\right)\\
B_0(t,0,m^2) &= iC_\epsilon \left(-\frac 2 \epsilon+2-\frac{t-m^2}{t}\ln\left(-\frac{t-m^2}{m^2}\right)\right)
\end{align}
focussing on imaginary part \textit{only}; this in accordance with \cite{Bojak:2000eu}\cite{PhysRevD.40.54}.

\subsection[Three-Point Function C0]{Three-Point Function $C_0$}
Again, in \cite[eq. (4.26)]{Denner:2005nn} is a generic function given.

First, we compute $C_0(s,q^2,0,m^2,m^2,m^2)$ and by taking the limit $k_1^2\rightarrow 0$ (or equivalenty $s_4\rightarrow 0$) we end up with:
\begin{align}
C_0(s,q^2,0,m^2,m^2,m^2)&=\frac{i}{16\pi^2}\cdot\frac 1{s-q^2}\left(\DiLog\left(\frac 2 {1+\beta_q}\right)+\DiLog\left(\frac 2 {1-\beta_q}\right)\right.\nonumber\\
&\hspace{60pt}\left.-\DiLog\left(\frac 2 {1+\beta}\right)-\DiLog\left(\frac 2 {1-\beta}\right)\right)
\end{align}
with \cite{Zagier2007} we find:
\begin{equation}
\DiLog\left(\frac 2 {1+b}\right)+\DiLog\left(\frac 2 {1-b}\right) =3\zeta(2)+\frac 1 2\ln^2\left(\frac{1-b}{1+b}\right)-\ln\left(\frac{1-b}{1+b}\right)\ln\left(-\frac{1-b}{1+b}\right)
\end{equation}
and if we focus on real part \textit{only}, we find:
\begin{align}
\DiLog\left(\frac 2 {1+\beta}\right)+\DiLog\left(\frac 2 {1-\beta}\right) &=3\zeta(2)-\frac 1 2\ln^2(\chi)\\
\DiLog\left(\frac 2 {1+\beta_q}\right)+\DiLog\left(\frac 2 {1-\beta_q}\right) &=-\frac 1 2\ln^2(\chi_q)
\end{align}

Additionally, we find
\begin{equation}
\lim_{q^2\rightarrow 0}\left[\DiLog\left(\frac 2 {1+\beta_q}\right)+\DiLog\left(\frac 2 {1-\beta_q}\right)\right] = 0
\end{equation}

So we get:
\begin{align}
C_0(s,q^2,0,m^2,m^2,m^2)&=iC_\epsilon\frac 1{s-q^2}\left(\frac 1 2\ln^2(\chi)-\frac 1 2\ln^2(\chi_q)-3\zeta(2)\right)\\
C_0(s,0,0,m^2,m^2,m^2)&=iC_\epsilon\frac 1{s}\left(\frac 1 2\ln^2(\chi)-3\zeta(2)\right)
\end{align}
in accordance with \cite{Bojak:2000eu}\cite{PhysRevD.40.54}\cite{Laenen1993162}. These results can also be obtained by the methods described in \cite[chap. 3]{Bojak:2000eu}.

Next, we compute $C_0(m^2,0,t,0,m^2,m^2)$ again by taking the limit $k_1^2\rightarrow 0$ we end up with:
\begin{align}
C_0(m^2,0,t,0,m^2,m^2) &= \frac{i}{16\pi^2}\cdot\frac 1{t-m^2}\left(2\DiLog(2) + \DiLog(m^2/t)-\frac{\pi^2}6\right.\nonumber\\
 &\hspace{40pt}\left.-\DiLog((t+m^2)/m^2)-\DiLog((m^2+t)/t)\right)
\end{align}
Using \cite{Zagier2007} and focussing on real part, we find
\begin{align}
\DiLog(2) &= \frac{\pi^2}4-i\pi\ln(2)\\
2\DiLog(2) + \DiLog(1/z)-\frac{\pi^2}6-\DiLog(1+z)-\DiLog(1+1/z) &=\frac{\pi^2}6-\DiLog(z)
\end{align}
So we get:
\begin{equation}
C_0(m^2,0,t,0,m^2,m^2)=iC_\epsilon\frac 1{t-m^2}\left(\zeta(2)-\DiLog(t/m^2)\right)
\end{equation}
in accordance with \cite{Bojak:2000eu}\cite{PhysRevD.40.54}.

To compute $C_0(m^2,s,m^2,0,m^2,m^2)$ we use \cite{Bojak:2000eu} and find
\begin{align}
C_0(m^2,s,m^2,0,m^2,m^2) &= \frac{i C_\epsilon}{s\beta}\left(-\frac 2 {\epsilon} \ln(\chi)-\frac{\pi^2}2 + \frac 1 2\ln^2(\chi) - \ln(\chi)\ln(1-\chi)\right.\nonumber\\
 &\hspace{40pt}\left.-\DiLog(1/(1-\chi))+\DiLog(\chi/(\chi-1))\right)
\end{align}
Using \cite{Zagier2007} and focussing on real part, we find
\begin{align}
-\DiLog(1/(1-\chi)) + \DiLog(\chi/(\chi-1))&=-2\DiLog(\chi)-\ln(\chi)\ln(1-\chi)-\frac{\pi^2}6
\end{align}
So we get:
\begin{align}
C_0(m^2,s,m^2,0,m^2,m^2) &=iC_\epsilon\frac 1{s\beta}\left(-\frac 2 {\epsilon} \ln(\chi)-2\ln(\chi)\ln(1-\chi)-2\DiLog(\chi)\right.\nonumber\\
 &\hspace{60pt}\left.+\frac 1 2\ln^2(\chi)-4\zeta(2)\right)
\end{align}
in accordance with \cite{Bojak:2000eu}\cite{PhysRevD.40.54}.

To compute $C_0(t,m^2,q^2,0,m^2,m^2)$ we use \cite{Denner:2005nn} and find immediatelty:
\begin{align}
C_0(t,m^2,q^2,0,m^2,m^2) &= \frac{i C_\epsilon}{\alpha}\left[-\zeta(2)+2\DiLog\left(\frac{t_1+\alpha}{t_1}\right)+\DiLog\left(\frac{q^2-t-m^2+\alpha}{q^2-t-m^2-\alpha}\right)\right.\nonumber\\
&\hspace{30pt}\DiLog\left(\frac{t_1-q^2\beta_q^2+\alpha}{t_1-q^2\beta_q^2-\beta_q\alpha}\right)-\DiLog\left(\frac{t_1-q^2\beta_q^2-\alpha}{t_1-q^2\beta_q^2+\beta_q\alpha}\right)\nonumber\\
&\hspace{30pt}\DiLog\left(\frac{t_1-q^2\beta_q^2+\alpha}{t_1+q^2\beta_q^2-\beta_q\alpha}\right)-\DiLog\left(\frac{t_1-q^2\beta_q^2-\alpha}{t_1+q^2\beta_q^2+\beta_q\alpha}\right)\nonumber\\
&\hspace{40pt}-\DiLog\left(\frac{t_1(q^2-t-m^2-\alpha)-2m^2\alpha}{t_1(q^2-t-m^2+\alpha)}\right)\nonumber\\
&\hspace{40pt}\left.-\DiLog\left(\frac{t_1(q^2-t-m^2-\alpha)-2m^2\alpha}{t_1(q^2-t-m^2-\alpha)}\right)\right]
\end{align}
with $\alpha=\kappa(t,q^2,m^2)$ and the Källén function (as defined in \cite[eq. (4.27)]{Denner:2005nn})
\begin{equation}
\kappa(x,y,z)=\sqrt{x^2+y^2+z^2-2(xy+xz+yx)}
\end{equation}
This is in accordance with \cite[eq. (A.8)]{Laenen1993162}(Note the typo there!).

Additionally, we find
\begin{align}
\lim_{q^2\rightarrow 0}C_0(t,m^2,q^2,0,m^2,m^2) = C_0(t,m^2,0,0,m^2,m^2) = C_0(m^2,0,t,0,m^2,m^2)
\end{align}

\subsection[Four-Point Function D0]{Four-Point Function $D_0$}
To compute $D_0(m^2,0,q^2,m^2,t,s,0,m^2,m^2,m^2)$ we follow Ingos way\cite{Bojak:2000eu} of computing his $D_0(p_1,-k_1,-k_2,0,m,m,m)=D_0(m^2,0,0,m^2,t,s,0,m^2,m^2,m^2)$ and find
\begin{align}
\tilde t &= -\frac{t_1}{m^2}\\
K &= \frac x {\rho \rho_q}\left[4 x (-1+y) y z \rho+yz \rho\rho_q\tilde t +x (-4 (-1+y) y (-1+z)+\rho-\tilde t y z \rho) \rho_q\right]\\
I_{xy} &= \frac{2 x^{\epsilon/2} \rho\rho_q^{2-\epsilon/2} \left[\tilde t y \rho_q+x (\rho_q+y (4 (y-1)-\tilde t \rho_q))\right]^{-1+\epsilon/2}}{(-2+\epsilon) \left[4 x (-1+y) \rho+\tilde t \rho \rho_q-x (4(y-1)+\tilde t \rho) \rho_q\right]}\\
II_{xy} &= -\frac{2 x^{-1+\epsilon} \rho^{2-\epsilon/2} \rho_q \left[4 (-1+y) y+\rho\right]^{-1+\epsilon/2} }{(-2+\epsilon) \left[4 x (-1+y) \rho+\tilde t \rho \rho_q-x (-4+4 y+\tilde t \rho) \rho_q\right]}
\end{align}
\textquote{The integration of $I_{xy}$ does not diverge and one easily gets upon setting $\epsilon\rightarrow 0$}
\begin{align}
I &= \frac{m^4}{s t_1 \beta}\left[\ln^2(\chi)+4\DiLog(-\chi)+\frac{\pi^2}3 +2\ln(\chi_q)\ln\left(\frac{\beta_q+\beta}{\beta_q-\beta}\right)-2\ln(\chi)\ln(1-q^2/s)\right.\nonumber\\
 &\hspace{40pt}\left. + 2\DiLog\left(\frac{\beta_q-1}{\beta_q-\beta}\right)-2\DiLog\left(\frac{\beta_q+1}{\beta_q-\beta}\right)+2\DiLog\left(\frac{\beta_q+1}{\beta_q+\beta}\right)-2\DiLog\left(\frac{\beta_q-1}{\beta_q+\beta}\right) \right]\\
&= \frac{m^4}{s t_1 \beta}\left[\ln^2(\chi)+4\DiLog(-\chi)+\frac{\pi^2}3 +\ln\left(\frac{\beta_q^2-\beta^2}{(\beta_q-1)^2}\right)\ln\left(\frac{\beta_q-\beta}{\beta_q+\beta}\right)-2\ln(\chi)\ln(1-q^2/s)\right.\nonumber\\
 &\hspace{40pt}\left. + 2\DiLog\left(\frac{\beta_q-1}{\beta_q-\beta}\right)+2\DiLog\left(\frac{\beta_q-\beta}{\beta_q+1}\right)-2\DiLog\left(\frac{\beta_q+\beta}{\beta_q+1}\right)-2\DiLog\left(\frac{\beta_q-1}{\beta_q+\beta}\right) \right]
\end{align}
\textquote{Integrating $II_{xy}$ over $x$ gives}
\begin{equation}
II_y = -\frac {2}{\tilde t(-2+\epsilon)\epsilon}\left(\frac{4y(y-1)+\rho}{\rho}\right)^{-1+\epsilon/2}\pFq 2 1 \left(1,\epsilon;1+\epsilon;1-\frac{4(y-1)(\rho_q-\rho)}{\tilde t \rho\rho_q}\right)
\end{equation}
\textquote{The integration over $y$ does not give an additional pole, so we can expand to $O(1)$ using (\cite[eq. B.5]{Bojak:2000eu}) and then integrate to obtain}
\begin{equation}
II = -\frac{m^4}{\beta s t_1}\left(\frac{2\ln(\chi)}{\epsilon} +  \ln(\chi)\left(1+2\ln(\beta \tilde t)+\ln(\chi)-2\ln\left(1-q^2/s\right)\right)+\DiLog(\chi^2)+\frac{5\pi^2}{6}\right)
\end{equation}
with
\begin{equation}
\lim_{q^2\rightarrow 0}\ln(1-q^2/s)=0
\end{equation}
\textquote{The final result is then}
\begin{align}
&D_0(m^2,0,q^2,m^2,t,s,0,m^2,m^2,m^2)\nonumber\\
 &=\frac{i C_\epsilon}{\beta s t_1}\left[-\frac{2\ln(\chi)}{\epsilon} -2\ln(\chi)\ln(\beta\tilde t)+2\DiLog(-\chi)-2\DiLog(\chi)-3\zeta(2) \right.\nonumber\\
 &\hspace{40pt}+\ln\left(\frac{\beta_q^2-\beta^2}{(\beta_q-1)^2}\right)\ln\left(\frac{\beta_q-\beta}{\beta_q+\beta}\right)-2\ln(\chi)\ln(1-q^2/s)\nonumber\\
 &\hspace{40pt}\left. + 2\DiLog\left(\frac{\beta_q-1}{\beta_q-\beta}\right)+2\DiLog\left(\frac{\beta_q-\beta}{\beta_q+1}\right)-2\DiLog\left(\frac{\beta_q+\beta}{\beta_q+1}\right)-2\DiLog\left(\frac{\beta_q-1}{\beta_q+\beta}\right) \right]
\end{align}
This is NOT in accordance with \cite[eq. (A.3)]{Laenen1993162} - but I suspect a bunch of typos there.

We get the match to \cite{Bojak:2000eu} and \cite{PhysRevD.40.54} by using
\begin{align}
&\lim_{q^2\rightarrow 0}\left[\ln\left(\frac{\beta_q^2-\beta^2}{(\beta_q-1)^2}\right)\ln\left(\frac{\beta_q-\beta}{\beta_q+\beta}\right)-2\ln(\chi)\ln(1-q^2/s)\right.\nonumber\\
&\hspace{20pt}\left. 2\DiLog\left(\frac{\beta_q-1}{\beta_q-\beta}\right)+2\DiLog\left(\frac{\beta_q-\beta}{\beta_q+1}\right)-2\DiLog\left(\frac{\beta_q+\beta}{\beta_q+1}\right)-2\DiLog\left(\frac{\beta_q-1}{\beta_q+\beta}\right) \right]&&=0
\end{align}
