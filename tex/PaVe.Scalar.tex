We focus on:
\begin{equation}
\Pggx(q) + \Pg(k_1) \rightarrow \PQ(p_1)+\PaQ(p_2)
\end{equation}
\begin{equation}
k_1^2 = 0 \quad p_1^2 = p_2^2 = m^2 \quad (p_1+p_2)^2=s\quad (p_2-q)^2=t\quad (p_1-q)^2=u
\end{equation}

define some shortcuts
\begin{align}
0\leq&\rho = \frac {4m^2} s\leq 1 &0\leq&\beta = \sqrt{1-\rho}\leq 1 &0\leq&\chi = \frac{1-\beta}{1+\beta}\leq 1\\
&\rho_q = \frac {4m^2} {q^2}\leq 0 &1\leq&\beta_q = \sqrt{1-\rho_q} &0\leq&\chi_q = -\frac{1-\beta_q}{1+\beta_q}\leq 1
\end{align}

\subsection[One-Point Function A0]{One-Point Function $A_0$}
\cite{Denner:2005nn}:
\begin{align}
A_0(m^2)&=-\frac{i}{16\pi^2}m^2\left(\frac{m^2}{4\pi\mu^2}\right)^{(n-4)/2}\Gamma(1-n/2)\\
 &= \frac{im^2}{16\pi^2}\left(\Delta-\log(m^2/\mu^2)+1\right) + O(n-4)\\
 &= iC_\epsilon m^2 \left(-\frac 2 \epsilon+1\right) + O(n-4)\\
\Delta &= \frac 2 {4-n}-\gamma_E+\log(4\pi)\\
C_\epsilon &= \frac 1 {16\pi^2}\exp\left(\left(\gamma_E-\log(4\pi)+\log\left(m^2/\mu^2\right)\right)\frac{\epsilon} 2\right)
\end{align}
this is \textit{up to order} $O(n-4)$ in accordance with \cite{Bojak:2000eu}\cite{PhysRevD.40.54}, but NOT beyond - see also \cite[eq. (A.12)]{Bojak:2000eu}. So we can treat $C_\epsilon$ and $\Delta$ as equal.

\subsection[Two-Point Function B0]{Two-Point Function $B_0$}
In \cite[eq. (4.23)]{Denner:2005nn} is a generic function given and we end up with
\begin{align}
B_0(s,m^2,m^2) &= iC_\epsilon \left(-\frac 2 \epsilon+2+\beta\log(\chi)\right)\\
B_0(q^2,m^2,m^2) &= iC_\epsilon \left(-\frac 2 \epsilon+2+\beta_q\log(\chi_q)\right)\\
B_0(0,m^2,m^2) &= iC_\epsilon \left(-\frac 2 \epsilon\right)\\
B_0(m^2,0,m^2) &= iC_\epsilon \left(-\frac 2 \epsilon+2\right)\\
B_0(t,0,m^2) &= iC_\epsilon \left(-\frac 2 \epsilon+2-\frac{t-m^2}{t}\ln\left(-\frac{t-m^2}{m^2}\right)\right)
\end{align}
focussing on imaginary part \textit{only}; this in accordance with \cite{Bojak:2000eu}\cite{PhysRevD.40.54}.

\subsection[Three-Point Function C0]{Three-Point Function $C_0$}
Again, in \cite[eq. (4.26)]{Denner:2005nn} is a generic function given.

First, we compute $C_0(s,q^2,0,m^2,m^2,m^2)$ and by taking the limit $k_1^2\rightarrow 0$ (or equivalenty $s_4\rightarrow 0$) we end up with:
\begin{align}
C_0(s,q^2,0,m^2,m^2,m^2)&=\frac{i}{16\pi^2}\cdot\frac 1{s-q^2}\left(\DiLog\left(\frac 2 {1+\beta_q}\right)+\DiLog\left(\frac 2 {1-\beta_q}\right)\right.\nonumber\\
&\hspace{60pt}\left.-\DiLog\left(\frac 2 {1+\beta}\right)-\DiLog\left(\frac 2 {1-\beta}\right)\right)
\end{align}
with \cite{Zagier2007} we find:
\begin{equation}
\DiLog\left(\frac 2 {1+b}\right)+\DiLog\left(\frac 2 {1-b}\right) =3\zeta(2)+\frac 1 2\ln^2\left(\frac{1-b}{1+b}\right)-\ln\left(\frac{1-b}{1+b}\right)\ln\left(-\frac{1-b}{1+b}\right)
\end{equation}
and if we focus on real part \textit{only}, we find:
\begin{align}
\DiLog\left(\frac 2 {1+\beta}\right)+\DiLog\left(\frac 2 {1-\beta}\right) &=3\zeta(2)-\frac 1 2\ln^2(\chi)\\
\DiLog\left(\frac 2 {1+\beta_q}\right)+\DiLog\left(\frac 2 {1-\beta_q}\right) &=-\frac 1 2\ln^2(\chi_q)
\end{align}

Additionally, we find
\begin{equation}
\lim_{q^2\rightarrow 0}\left[\DiLog\left(\frac 2 {1+\beta_q}\right)+\DiLog\left(\frac 2 {1-\beta_q}\right)\right] = 0
\end{equation}

So we get:
\begin{align}
C_0(s,q^2,0,m^2,m^2,m^2)&=iC_\epsilon\frac 1{s-q^2}\left(\frac 1 2\ln^2(\chi)-\frac 1 2\ln^2(\chi_q)-3\zeta(2)\right)\\
C_0(s,0,0,m^2,m^2,m^2)&=iC_\epsilon\frac 1{s}\left(\frac 1 2\ln^2(\chi)-3\zeta(2)\right)
\end{align}
in accordance with \cite{Bojak:2000eu}\cite{PhysRevD.40.54}\cite{Laenen1993162}. These results can also be obtained by the methods described in \cite[chap. 3]{Bojak:2000eu}.

Next, we compute $C_0(m^2,0,t,0,m^2,m^2)$ again by taking the limit $k_1^2\rightarrow 0$ we end up with:
\begin{align}
C_0(m^2,0,t,0,m^2,m^2) &= \frac{i}{16\pi^2}\cdot\frac 1{t-m^2}\left(2\DiLog(2) + \DiLog(m^2/t)-\frac{\pi^2}6\right.\nonumber\\
 &\hspace{40pt}\left.-\DiLog((t+m^2)/m^2)-\DiLog((m^2+t)/t)\right)
\end{align}
Using \cite{Zagier2007} and focussing on real part, we find
\begin{align}
\DiLog(2) &= \frac{\pi^2}4-i\pi\ln(2)\\
2\DiLog(2) + \DiLog(1/z)-\frac{\pi^2}6-\DiLog(1+z)-\DiLog(1+1/z) &=\frac{\pi^2}6-\DiLog(z)
\end{align}
So we get:
\begin{equation}
C_0(m^2,0,t,0,m^2,m^2)=iC_\epsilon\frac 1{t-m^2}\left(\zeta(2)-\DiLog(t/m^2)\right)
\end{equation}
in accordance with \cite{Bojak:2000eu}\cite{PhysRevD.40.54}.

To compute $C_0(m^2,s,m^2,0,m^2,m^2)$ we use \cite{Bojak:2000eu} and find
\begin{align}
C_0(m^2,s,m^2,0,m^2,m^2) &= \frac{i C_\epsilon}{s\beta}\left(-\frac 2 {\epsilon} \ln(\chi)-\frac{\pi^2}2 + \frac 1 2\ln^2(\chi) - \ln(\chi)\ln(1-\chi)\right.\nonumber\\
 &\hspace{40pt}\left.-\DiLog(1/(1-\chi))+\DiLog(\chi/(\chi-1))\right)
\end{align}
Using \cite{Zagier2007} and focussing on real part, we find
\begin{align}
-\DiLog(1/(1-x)) + \DiLog(x/(x-1))&=-2\DiLog(z)-\ln(x)\ln(1-x)-\frac{\pi^2}6
\end{align}
So we get:
\begin{align}
C_0(m^2,s,m^2,0,m^2,m^2) &=iC_\epsilon\frac 1{s\beta}\left(-\frac 2 {\epsilon} \ln(\chi)-2\ln(\chi)\ln(1-\chi)-2\DiLog(\chi)\right.\nonumber\\
 &\hspace{60pt}\left.+\frac 1 2\ln^2(\chi)-4\zeta(2)\right)
\end{align}
in accordance with \cite{Bojak:2000eu}\cite{PhysRevD.40.54}.

To compute $C_0(t,m^2,q^2,0,m^2,m^2)$ we use \cite{Denner:2005nn} and find immediatelty:
\begin{align}
C_0(t,m^2,q^2,0,m^2,m^2) &= \frac{i C_\epsilon}{\alpha}\left[-\zeta(2)+2\DiLog\left(\frac{t_1+\alpha}{t_1}\right)+\DiLog\left(\frac{q^2-t-m^2+\alpha}{q^2-t-m^2-\alpha}\right)\right.\nonumber\\
&\hspace{30pt}\DiLog\left(\frac{t_1-q^2\beta_q^2+\alpha}{t_1-q^2\beta_q^2-\beta_q\alpha}\right)-\DiLog\left(\frac{t_1-q^2\beta_q^2-\alpha}{t_1-q^2\beta_q^2+\beta_q\alpha}\right)\nonumber\\
&\hspace{30pt}\DiLog\left(\frac{t_1-q^2\beta_q^2+\alpha}{t_1+q^2\beta_q^2-\beta_q\alpha}\right)-\DiLog\left(\frac{t_1-q^2\beta_q^2-\alpha}{t_1+q^2\beta_q^2+\beta_q\alpha}\right)\nonumber\\
&\hspace{40pt}-\DiLog\left(\frac{t_1(q^2-t-m^2-\alpha)-2m^2\alpha}{t_1(q^2-t-m^2+\alpha)}\right)\nonumber\\
&\hspace{40pt}\left.-\DiLog\left(\frac{t_1(q^2-t-m^2-\alpha)-2m^2\alpha}{t_1(q^2-t-m^2-\alpha)}\right)\right]
\end{align}
with $\alpha=\kappa(t,q^2,m^2)$ and the Källén function (as defined in \cite[eq. (4.27)]{Denner:2005nn})
\begin{equation}
\kappa(x,y,z)=\sqrt{x^2+y^2+z^2-2(xy+xz+yx)}
\end{equation}
This is in accordance with \cite[eq. (A.8)]{Laenen1993162}(Note the typo there!).

Additionally, we find
\begin{align}
\lim_{q^2\rightarrow 0}C_0(t,m^2,q^2,0,m^2,m^2) = C_0(t,m^2,0,0,m^2,m^2) = C_0(m^2,0,t,0,m^2,m^2)
\end{align}
