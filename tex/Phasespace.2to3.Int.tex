at phase space integration there occur integrations over propagators\cite{Bojak:2000eu,PhysRevD.40.54,van_neerven_dimensional_1986}; the propagators can be decomposed in 2 types: [ab] and [ABC]; the needed integrals then reduce to the master formula:
\begin{align}
I_n^{(k,l)} &=\int\limits_0^\pi\!d\theta_1\,\sin^{n-3}(\theta_1)\int\limits_0^\pi\!d\theta_2\,\sin^{n-4}(\theta_2)(a+b\cos(\theta_1))^{-k}(A+B\cos(\theta_1)+C\sin(\theta_1)\cos(\theta_2))^{-l}\\
&=\int\!d\Omega_n\,(a+b\cos(\theta_1))^{-k}(A+B\cos(\theta_1)+C\sin(\theta_1)\cos(\theta_2))^{-l}
\end{align}

the integrals can be further destinguished by the range of $k,l$ and the type of collinearity (following the notation in \cite{Bojak:2000eu}):
\begin{itemize}
\item "non collinear": $a^2\neq b^2 \land A^2 \neq B^2 + C^2 \rightarrow I_{0,n}^{(k,l)}$
\item "single collinear a": $a=-b \land A^2 \neq B^2 + C^2 \rightarrow I_{a,n}^{(k,l)}$
\item "single collinear A": $a^2\neq b^2 \land A^2 = B^2 + C^2 \rightarrow I_{A,n}^{(k,l)}$
\item "double collinear": $a=-b \land A = -\sqrt{B^2 + C^2} \rightarrow I_{aA,n}^{(k,l)}$
\end{itemize}

Use $n=4+\epsilon$.

\subsubsection{integral helper}
define helper integral
\begin{equation}
\hat I^{(q)}(\nu) := \int\limits_0^\pi\!dt\,\sin^{\nu-3}(t)\cos^{q}(t)
\end{equation}

It is\cite[eq. 5.12.6]{NIST:DLMF}:
\begin{equation}
\int_{0}^{\pi}(\mathop{\sin\/}\nolimits t)^{\alpha-1}e^{i\beta t}dt=\frac{\pi}{2^{\alpha-1}} \frac{e^{i\pi \beta/2}}{\alpha\mathop{B}\left((\alpha+\beta+1)/2,(\alpha-\beta+1)/2\right)} \qquad\text{if}\,\Re(\alpha) > 0
\end{equation}
\begin{align}
\Rightarrow \hat I^{(0)}(n) &= \frac{\pi}{2^{n-3}(n-2)}\frac 1 {\mathop{B}((n-1)/2,(n-1)/2)}\\
\Rightarrow \hat I^{(0)}(n-1) &= \frac{\pi}{2^{n-4}(n-3)}\frac 1 {\mathop{B}((n-2)/2,(n-2)/2)}=\mathop{B}((n-3)/2,1/2)
\end{align}

If q is odd: $\hat I^{(q)}=0$, due to symetry of kernel; if q is even: $q=2p$ with $p\in\mathbbm N$:
\begin{align}
\hat I^{(2p)}(\nu) &= \frac 1 {2^{2p}}\sum\limits_{k=0}^{2p} \binom{2p}{k} \int\limits_0^\pi\sin^{\nu-3}(t)\exp(2i(k-p)t)\,dt\\
 &= \frac \pi{2^{2p+\nu-3} (\nu-2)}\sum\limits_{k=0}^{2p} \binom{2p}{k}\frac{\exp(i\pi(k-p))}{\mathop{B}((\nu-1)/2+(k-p),(\nu-1)/2-(k-p))}\\
 &= \frac \pi{2^{2p+\nu-3} (\nu-2)}\sum\limits_{l=-p}^{p} \binom{2p}{p+l}\frac{(-1)^l}{\mathop{B}((\nu-1)/2+l,(\nu-1)/2-l)}\\
 &= \frac {\pi\Gamma(\nu-1)(2p)!}{2^{2p+\nu-3} (\nu-2)\Gamma(\frac{n-1} 2+p)\Gamma(\frac{n-1} 2+p)}\left(\frac 1 {(p!)^2} \frac{\Gamma(\frac{\nu-1} 2+p)}{\Gamma(\frac{\nu-1} 2)}\frac{\Gamma(\frac{\nu-1} 2-p)}{\Gamma(\frac{\nu-1} 2)} \right. \nonumber\\
 &\hspace{40pt}\left. + 2\sum\limits_{l=1}^{p}\frac{(-1)^l}{(p+l)!(p-l)!}\frac{\Gamma(\frac{\nu-1} 2+p)}{\Gamma(\frac{\nu-1} 2+l)}\frac{\Gamma(\frac{\nu-1} 2-p)}{\Gamma(\frac{\nu-1} 2-l)}\right)\\
 &= \frac {2^{3-\nu}\pi\Gamma(\nu-1)}{(\nu-2)\Gamma(\frac{n-1} 2+p)\Gamma(\frac{n-1} 2+p)}\cdot \frac{\Gamma(\frac{\nu-1} 2-p)}{2^{p}\Gamma(\frac{\nu-1} 2)} \cdot \frac{(2p)!}{2^p p!} \cdot p! \left(\frac 1 {(p!)^2} \frac{\Gamma(\frac{\nu-1} 2+p)}{\Gamma(\frac{\nu-1} 2)} \right. \nonumber\\
 &\hspace{40pt}\left. + 2\sum\limits_{l=1}^{p}\frac{(-1)^l}{(p+l)!(p-l)!}\frac{\Gamma(\frac{\nu-1} 2+p)}{\Gamma(\frac{\nu-1} 2+l)}\frac{\Gamma(\frac{\nu-1} 2)}{\Gamma(\frac{\nu-1} 2-l)}\right)
\end{align}
TODO: prove \fxerror{prove}
\begin{align}
&p! \left(\frac 1 {(p!)^2} \frac{\Gamma(\frac{\nu-1} 2+p)}{\Gamma(\frac{\nu-1} 2)} + 2\sum\limits_{l=1}^{p}\frac{(-1)^l}{(p+l)!(p-l)!}\frac{\Gamma(\frac{\nu-1} 2+p)}{\Gamma(\frac{\nu-1} 2+l)}\frac{\Gamma(\frac{\nu-1} 2)}{\Gamma(\frac{\nu-1} 2-l)}\right)\\
&= \frac {1} {p!}\frac{\Gamma(-\frac{1}2+p)}{\Gamma(-\frac{1}2)} + 2\sum\limits_{l=1}^{p}\frac{(-1)^l p!}{(p+l)!(p-l)!}\frac{\Gamma(-\frac{1} 2+p)}{\Gamma(-\frac{1} 2+l)}\frac{\Gamma(-\frac{1} 2)}{\Gamma(-\frac{1} 2-l)}\\
&= 1
\end{align}
\begin{align}
\Rightarrow \hat I^{(2p)}(\nu) &=\frac {2^{3-\nu}\pi\Gamma(\nu-1)}{(\nu-2)\Gamma(\frac{n-1} 2+p)\Gamma(\frac{n-1} 2-p)} \cdot \frac{\Gamma(\frac{\nu-1} 2-p)}{2^{p}\Gamma(\frac{\nu-1} 2)} \cdot \frac{(2p!)}{2^p p!}\\
 &= \frac{\sqrt{\pi}(2p)!}{2^{2p}p!} \frac {\Gamma((\nu-2)/2)} {\Gamma(\frac{\nu-1}2+p)}
\end{align}

\subsubsection{any collinearity and $-k,-l\in\mathbbm N_0$}
If $-k,-l\in\mathbbm N_0$ $I_{n}^{(k,l)}$ can always be reduced in a straight forward manner to combinations of $\hat I^{(q)}(n)$ and this way one finds\cite[Ch. 5]{Bojak:2000eu}\cite[App. C]{PhysRevD.40.54}:
\begin{align}
I^{(0,0)}_{n} &= \hat I^{(0)}(n-1) \cdot \hat I^{(0)}(n) = \frac{2\pi}{n-3}\\
I^{(0,0)}_{4} &= 2\pi\\
I^{(-1,0)}_{n} &= \hat I^{(0)}(n-1) \cdot (a\hat I^{(0)}(n)+b\hat I^{(1)}(n)) = \frac{2\pi a}{n-3}\\
I^{(-1,0)}_{4} &= 2\pi a \\
I^{(0,-1)}_{n} &= \hat I^{(0)}(n-1) \cdot (A\hat I^{(0)}(n) + B\hat I^{(1)}(n)) + C\hat I^{(1)}(n-1)\hat I^{(0)}(n)\\
 &= \frac{2\pi A}{n-3}\\
I^{(0,-1)}_{4} &=2\pi A\\
I^{(-2,0)}_{n} &= \hat I^{(0)}(n-1) \cdot (a^2\hat I^{(0)}(n)+2ab\hat I^{(1)}(n) + b^2 \hat I^{(2)}(n))\\
 &= 2\pi\left(\frac{a^2(n-1)+b^2}{(n-1)(n-3)}\right)\\
I^{(-2,0)}_{4} &= 2\pi(a^2 + b^2/3) \\
I^{(0,-2)}_{n} &= \hat I^{(0)}(n-1) \cdot (A^2\hat I^{(0)}(n) + B^2\hat I^{(2)}(n)) + C^2\hat I^{(2)}(n-1)\hat I^{(0)}(n+2) \\
 &= 2\pi\left(\frac{A^2(n-1)+B^2+C^2}{(n-1)(n-3)}\right)\\
I^{(0,-2)}_{4} &= 2\pi(A^2+(B^2+C^2)/3) \\
I^{(-1,-1)}_{n} &= \hat I^{(0)}(n-1) \cdot (a A\hat I^{(0)}(n) + b B \hat I^{(2)}(n)) = 2\pi\left(\frac{aA(n-1)+bB}{(n-1)(n-3)}\right)\\
I^{(-1,-1)}_{4} &= 2\pi(aA + bB/3)
\end{align}

\subsubsection{single collinear a and $k,-l\in \mathbbm N_0$}
It is
\begin{align}
\hat I_{a}^{(k,q)}(\nu) &= \int\limits_0^\pi\!\frac{\sin^{\nu-3}t}{(1-\cos(t))^k}\cos^q(t)\, dt \\
 &= \int\limits_0^\pi\!\frac{\sin^{\nu-3}(t)}{(1-\cos^2(t))^k}\cos^q(t)(1+\cos(t))^k\, dt\\
 &=\int\limits_0^\pi\!\sin^{\nu-3-2k}(t)\cos^q(t)(1+\cos(t))^k\, dt\\
 &= \sum\limits_{l=0}^k\binom{k}{l}\hat I^{(q+l)}(\nu-2k)
\end{align}

this way one finds\cite[Ch. 5]{Bojak:2000eu}\cite[App. C]{PhysRevD.40.54}:
\begin{align}
I^{(1,0)}_{a,n} &= \frac 1 a\hat I^{(0)}(n-1) \cdot \hat I^{(0)}(n-2)\\
 &= \frac {2\pi}{a(n-4)}\\
I^{(2,0)}_{a,n} &= \frac 1 {a^2}\hat I^{(0)}(n-1) \cdot \left(\hat I^{(0)}(n-4) + \hat I^{(2)}(n-4)\right)\\
 &= \frac {2\pi}{a^2(n-6)} \approx -\frac {\pi}{a^2} + O(\epsilon)\\
I^{(1,-1)}_{a,n} &= \frac 1 a\hat I^{(0)}(n-1)\cdot \left(A\hat I^{(0)}(n-2)+B\hat I^{(2)}(n-2)\right)\\
 &= \frac {2\pi}{a}\frac{(A(n-3)+B)}{(n-3)(n-4)} \approx \frac {2\pi}{a}\left(\frac{A+B}\epsilon - 2B + O(\epsilon)\right)\\
I^{(1,-2)}_{a,n} &= \frac 1 a\left(\hat I^{(0)}(n-1)\cdot \left(A^2\hat I^{(0)}(n-2)+(B^2+2AB)\hat I^{(2)}(n-2)\right) + C^2\hat I^{(2)}(n-1)\hat I^{(0)}(n)\right)\\
 &= \frac {2\pi}{a}\left(\frac {A^2}{n-4} + \frac {2AB + B^2}{(n-4)(n-3)} + \frac {C^2}{(n-3)(n-2)}\right) \\
 &\approx \frac {2\pi}{a}\left(\frac{(A+B)^2}{\epsilon}+\frac{C^2}{2}-2AB-B^2+O(\epsilon)\right)\\
I^{(2,-2)}_{a,n} &= \frac 1 {a^2}\left(\hat I^{(0)}(n-1)\cdot \left(A^2(\hat I^{(0)}(n-4)+\hat I^{(2)}(n-4))+4AB\hat I^{(2)}(n-4) \right.\right.\nonumber\\
 &\hspace{20pt}\left.\left. + B^2(\hat I^{(2)}(n-4)+\hat I^{(4)}(n-4))\right) + C^2\hat I^{(2)}(n-1)(\hat I^{(0)}(n-2) + \hat I^{(2)}(n-2))\right)\\
 &= \frac {2\pi}{a^2}\left(\frac {A^2}{n-6}+\frac{4AB}{(n-6)(n-4)} + \frac{B^2 n}{(n-6)(n-4)(n-3)} + \frac{C^2}{(n-4)(n-3)} \right) \\
 &\approx \frac {2\pi}{a^2}\left(\frac{-2AB-2B^2+C^2}{\epsilon}+\frac{B^2-A^2}{2}-AB-C^2+O(\epsilon)\right)\\
\end{align}

\subsubsection{double collinear and $k,l\in\mathbbm N$}
as said in \cite[Ch. 5]{Bojak:2000eu}: if $0\leq -\frac{C}{A},\frac B A \leq 1$ use \cite[eq. A11]{van_neerven_dimensional_1986} with $\cos\kappa = -\frac B A$:
\begin{equation}
I_{aA,n}^{(k,l)} = \frac{2\pi 2^{-(k+l)}}{a^kA^l}\frac{\Gamma(1+\epsilon)}{\Gamma^2(1+\epsilon/2)} B(1+\frac \epsilon 2 - k, 1+ \frac \epsilon 2 -l) \pFq 2 1 \left(k,l;1+\frac \epsilon 2; \frac{A-B}{2A}\right)
\end{equation}

\subsubsection{non collinear and $k,-l\in\mathbbm N$}
Next we want to compute $I_{0,n}^{1,-3}$.

The $\theta_2$ integration can be performed using the integral helper and the problem reduces then to the following integral:
\begin{align}
&\hat I_{0,\cos}^{(k,q,p)}(\epsilon) \nonumber\\
&= \int\limits_0^\pi\!d\theta_1\, \frac{\sin^{1+\epsilon}(\theta_1)\sin^q(\theta_1) \cos^p(\theta_1)}{(a+b\cos(\theta_1))^k}\\
&= \frac 1 {2a^k}\left((1+(-1)^p)B\left(\frac{2+q+\epsilon}{2},\frac{1+p}{2}\right)\pFq 3 2 \left(\frac{1+k}{2},\frac k 2,\frac{1+p}{2};\frac{1}{2},\frac{3+q+p+\epsilon}{2};\frac{b^2}{a^2}\right)\right.\nonumber\\
 &\hspace{20pt}\left. \frac b a k(-1+(-1)^p)B\left(\frac{2+q+\epsilon}{2},\frac{2+p}{2}\right)\pFq 3 2\left(\frac{1+k} 2,\frac{2+k}{2},\frac{2+p}{2};\frac{3}{2},\frac{4+q+p+\epsilon}{2};\frac{b^2}{a^2}\right) \right)
\end{align}
for $k=1$ this simplifies to
\begin{align}
&\hat I_{0,\cos}^{(1,q,p)}(\epsilon) \nonumber\\
&= \int\limits_0^\pi\!d\theta_1\, \frac{\sin^{1+\epsilon}(\theta_1)\sin^q(\theta_1) \cos^p(\theta_1)}{(a+b\cos(\theta_1))}\\
&= \frac 1 {2a}\left((1+(-1)^p)B\left(\frac{2+q+\epsilon}{2},\frac{1+p}{2}\right)\pFq 2 1 \left(1,\frac{1+p}{2};\frac{3+q+p+\epsilon}{2};\frac{b^2}{a^2}\right)\right.\nonumber\\
 &\hspace{20pt}\left. \frac b a (-1+(-1)^p)B\left(\frac{2+q+\epsilon}{2},\frac{2+p}{2}\right)\pFq 2 1\left(1,\frac{2+p}{2};\frac{4+q+p+\epsilon}{2};\frac{b^2}{a^2}\right) \right)
\end{align}

\subsubsection{needed set}
we will need for $A_1$
\begin{equation}
\mathcal I_{A1}^{(j,k)} = \int\!d\Omega_n\,{t'}^{-2+j} {u_7}^{-2+k} \qquad j,k = \{0,1,2,3\}
\end{equation}
with
\begin{align}
a(t') = -b(t') &= -2\omega_1\omega_2 &&= -\frac{s_4(s_4-u_1)}{2(s_4+m^2)}\\
A(u_7) &= q^2-2q_0E_1 &&= q^2-\frac{(s_4+2m^2)(s+u_1)}{2(s_4+m^2)}\\
B(u_7) &= -2\omega_2(\abs{\vec p_2}\cos\psi-\omega_1) &&=\frac {s_4}{2}\left(1-\frac{s+u_1}{s_4+m^2}+\frac{s-q^2-t_1}{s_4-u_1}\right)\\
C(u_7) &= -2\omega_2\abs{\vec p_2}\sin\psi
\end{align}
%\begin{align}
%\int\!d\Omega_n &= \frac{2\pi}{n-3}\\
%\int\!d\Omega_n\,t' &= \frac{2\pi}{n-3} (-2\omega_1\omega_2) &&= -\frac{\pi}{n-3}\cdot\frac{s_4(s_4-u_1)}{s_4+m^2}\\
%\int\!d\Omega_n\,u_7 &= \frac{2\pi}{n-3}(q^2-2q_0E_1) &&=\frac{\pi}{n-3}\left(2q^2+\frac{(s_4+2m^2)(s+u_1)}{2(s_4+m^2)}\right)\\
%\int\!d\Omega_n\,\frac 1 {t'} &= \frac{2\pi}{n-4} \frac 1 {-2\omega_1\omega_2} &&= -\frac{4\pi}{n-4}\cdot\frac{s_4+m^2}{s_4(s_4-u_1)}\\
%\end{align}
