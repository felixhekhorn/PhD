\begin{align}
M_{\vec \kappa}^{(1),V}&=\hat {\mathcal P}_{\vec \kappa}^{\Pgg,\mu\mu'}\hat {\mathcal P}_{\kappa_2}^{\Pg,\nu\nu'}\sum_{j,j'}2\text{Re}\left[\Md^{(1),V}_{j,\mu\nu}\left(\Md^{(0)}_{j',\mu'\nu'}\right)^*\right] \nonumber\\
 &= 8g^4\mu_D^{-\epsilon}e^2e_H^2 N_C C_FC_\epsilon\left( C_A V_{\vec \kappa,\tOK} + 2C_F V_{\vec \kappa,\tQED}\right)
\end{align}
where $C_\epsilon = \exp(\epsilon/2(\gamma_E-\ln(4\pi)))/(16\pi^2)$ and $C_A$ is the second Casimir constant of the adjoint representation for the gluon (that introduces a non-abelian part).

The full expressions for the $V_{\vec \kappa,\tOK},V_{\vec \kappa,\tQED}$ are quite complicated and are too long to be presented here, nevertheless the arising poles are quite compact:
\begin{align}
V_{\vec \kappa,\tOK} &= -2B_{\vec \kappa,\tQED}\left(\frac 4 {\epsilon^2} + \left(\ln(-t_1/m^2) + \ln(-u_1/m^2) +\frac{s-2m^2}{s\beta}\ln(\chi)\right)\frac 2 \epsilon \right) + O(\epsilon^0)\\
V_{\vec \kappa,\tQED} &= -2B_{\vec \kappa,\tQED}\left(1-\frac{s-2m^2}{s\beta}\ln(\chi)\right)\frac 2 \epsilon + O(\epsilon^0)
\end{align}
The above results already include the mass renormalization that we have performed \textit{on-shell}, so all ultra-violet poles have been removed. For the renormalization of the strong coupling we use the $\MSbar_m$ scheme defined in \cite{Bojak:2000eu} and so the full (remaining) renormalization can be achieved by
\begin{align}
\frac{d^2\sigma_{\vec \kappa,\Pg}^{(1),V}}{dt_1du_1} &=\left.\frac{d^2\sigma_{\vec \kappa,\Pg}^{(1),V}}{dt_1du_1} \right|_{\text{bare}}+ 4\pi\alpha_s(\mu_R^2)C_\epsilon\left(\frac{\mu_D^2}{m^2}\right)^{-\epsilon/2}\left[\left(\frac 2 \epsilon +\ln(\mu_R^2/m^2)\right)\beta_0^f +\frac 2 3 \ln(\mu_R^2/m^2)\right]\frac{d^2\sigma_{\vec \kappa,\Pg}^{(0)}}{dt_1du_1}
\end{align}
with $\mu_R$ the renormalization scale introduced by the RGE, $\beta_0^f = (11C_A- 2n_{f})/3$ the first coefficient of the beta function and $n_f$ the number of \textit{total} flavours (i.e. $n_{lf}=n_f-1$ active (light) flavours and one heavy flavour). The double poles occuring in $V_{\vec \kappa,\tOK}$ are introduced by the diagrams \fxerror{do} when the soft and collinear singularities coincide.

The partonic cross section is given by
\begin{align}
d\sigma_{\vec \kappa,\Pg}^{(1),V} &= \frac 1 {2s'}\frac 1 2 E_{\kappa_2}(\epsilon) M_{\vec \kappa}^{(1),V} \dPSTwo
\end{align}
