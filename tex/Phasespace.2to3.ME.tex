define a shortcut:
\begin{equation}
\mathcal V_{a,b}(x,y) = \left(x^{k} y^{l}\right)_{k=0..a,l=0..b}
\end{equation}

It is
\begin{align}
A_{G,1} &= \sum_{k,l=0}^{3} \left(\mathcal C_{A_{G,1}}\right)_{(k,l)} {t'}^{-2+k} {u_7}^{-2+l} &&= \tr\left(\mathcal C_{A_{G,1}} \frac{\mathcal V_{3,3}(t',u_7)^t}{{t'}^2{u_7}^2}\right)\\
A_{L,1} &= \sum_{k=0}^{4}\sum_{l=0}^{2} \left(\mathcal C_{A_{L,1}}\right)_{(k,l)} {t'}^{-2+k} {u_7}^{-2+l} &&= \tr\left(\mathcal C_{A_{L,1}} \frac{\mathcal V_{4,2}(t',u_7)^t}{{t'}^2{u_7}^2}\right)
\end{align}
and we will thus need the integrals
\begin{align}
\left(\mathcal I_{A_{G,1}}\right)_{(k,l)} &= \int\!d\Omega_n\,\frac 1 {{t'}^2{u_7}^2}\left(\mathcal V_{3,3}(t',u_7)\right)_{(k,l)}\\
\left(\mathcal I_{A_{L,1}}\right)_{(k,l)} &= \int\!d\Omega_n\,\frac 1 {{t'}^2{u_7}^2}\left(\mathcal V_{4,2}(t',u_7)\right)_{(k,l)}
\end{align}
with
\begin{align}
a(t') = -b(t') &= -2\omega_1\omega_2 &&= -\frac{s_4(s_4-u_1)}{2(s_4+m^2)}\\
A(u_7) &= q^2-2q_0E_1 &&= q^2-\frac{(s_4+2m^2)(s+u_1)}{2(s_4+m^2)}\\
B(u_7) &= -2\omega_2(\abs{\vec p_2}\cos\psi-\omega_1) &&=\frac {s_4}{2}\left(1-\frac{s+u_1}{s_4+m^2}+\frac{s-q^2-t_1}{s_4-u_1}\right)\\
C(u_7) &= -2\omega_2\abs{\vec p_2}\sin\psi
\end{align}
that is $I_{a,n}^{(-2\ldots 2,-1\ldots 2)}$. With this we get
\begin{equation}
\int\!d\Omega_n\,A_{j,1} = \tr(\mathcal C_{A_{j,1}} \left(\mathcal I_{A_{j,1}}\right)^t) \qquad j=G,L
\end{equation}

It is
\begin{align}
A_{G,2} &= \sum_{k,l=0}^{3} \left(\mathcal C_{A_{G,2}}\right)_{(k,l)} {s_5}^{-2+k} {u'}^{-2+l} &&= \tr\left(\mathcal C_{A_{G,2}} \frac{\mathcal V_{3,3}(s_5,u')^t}{{s_5}^2{u'}^2}\right)\\
A_{L,2} &= \sum_{k=0}^{4}\sum_{l=0}^3 \left(\mathcal C_{A_{G,2}}\right)_{(k,l)} {s_5}^{-2+k} {u'}^{-2+l} &&= \tr\left(\mathcal C_{A_{L,2}} \frac{\mathcal V_{4,3}(s_5,u')^t}{{s_5}^2{u'}^2}\right)
\end{align}
and we will thus need the integrals
\begin{align}
\left(\mathcal I_{A_{G,2}}\right)_{(k,l)} &= \int\!d\Omega_n\,\frac 1 {{s_5}^2{u'}^2}\left(\mathcal V_{3,3}(s_5,u')\right)_{(k,l)}\\
\left(\mathcal I_{A_{L,2}}\right)_{(k,l)} &= \int\!d\Omega_n\,\frac 1 {{s_5}^2{u'}^2}\left(\mathcal V_{4,3}(s_5,u')\right)_{(k,l)}
\end{align}
but as both $s_5$ and $u'$ are of $[ABC]$ type we have to apply partial fractioning, following the ideas of \cite[Ch. 5]{Bojak:2000eu}. It is 
\begin{equation}
(\ref{eq:MonCon3s5up}): \qquad s_5 = s-q^2+t'+u'
\end{equation}
so end up with a form of $\mathcal V(p+q,q)$ where $p$ is $[ab]$ and $q$ and $p+q$ are $[ABC]$. The aim is then to get to a form with fractions of $\frac p q$ and/or $\frac {p+q} p$ and indeed this can be achieved. Define
\begin{equation}
\mathcal T =\!\left(\!\!
\begin{array}{cccc}
\{-2,1,2,1\} & \{1,\hphantom{-} 0,-1,-1\} & \{0,0,0,1\} & \{0,0,1,-1\}\\
\{-1,1,1,0\} & \{1,-1,\hphantom{-} 0,\hphantom{-} 0\}  & \{0,0,1,0\} & \{0,0,1,-1\}\\
\{\hphantom{-} 0,1,0,0\}  & \{1,\hphantom{-} 0,\hphantom{-} 0,\hphantom{-} 0\}   & \{1,0,0,0\} & \{0,0,1,-1\}\\
\{\hphantom{-} 1,1,0,0\}  & \{1,\hphantom{-} 1,\hphantom{-} 0,\hphantom{-} 0\}   & \{1,1,0,0\} & \{0,0,1,-1\}
\end{array}\!\!
\right)
\end{equation}
\begin{equation}
\left(\mathcal W(x,y)\right)_{(k,l)} = x^{-2+p_x(k+l)} y^{-2+p_y(k+l)}
\end{equation}
\begin{equation}
p_x(s) = \left\{\begin{array}{ll}s&\text{if }s\leq 3\\ 3 &\text{else}\end{array}\right.
\qquad
p_y(s) = 3-p_x(6-s)\left\{\begin{array}{ll}0&\text{if }s\leq 3\\ s-3 &\text{else}\end{array}\right.
\end{equation}
\begin{equation}
\Rightarrow \mathcal W(x,y) = \!\left(\!\!
\begin{array}{cccc}
x^{-2}y^{-2} & x^{-1}y^{-2} & y^{-2} & xy^{-2}\\
x^{-1}y^{-2} & y^{-2} & xy^{-2} & xy^{-1} \\
y^{-2} & xy^{-2} & xy^{-1} & x\\
xy^{-2} & xy^{-1} & x & xy
\end{array}\!\!
\right)
\end{equation}
\begin{equation}
\left(\mathcal{W}^*(p,q)\right)_{(k,l)} = \left\{\frac q p\mathcal W(p,q),\mathcal W(p,q),\frac {p+q} p\mathcal W(p,p+q),\mathcal{W}(p,p+q)\right\}_{(k,l)}
\end{equation}
we then find:
\begin{equation}
\left(\mathcal V(p+q,q)\right)_{(k,l)} = \mathcal T_{(k,l)}\cdot \mathcal W^*(p,q)_{(k,l)}
\end{equation}
where the equation has to be read separate for each $(k,l)$, i.e. ``$\cdot$`` applies to the scalar $4x4$-product of the $\{\}$s. As $\mathcal W^t = \mathcal W$, $\mathcal W$ has 10 unique entries and thus for $\mathcal W^*$ are 40 unique integrals needed: $I_0^{(-1\ldots 3,-2\ldots 2)}$.

