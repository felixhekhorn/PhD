use c.m.s. of recoiling heavy and light quark ($Q(p_1)$ and $q(k_2)$):
\begin{align}
k_2 &= (\omega_2,k_{2,x},\omega_2\sin\theta_1\cos\theta_2,\omega_2\cos\theta_1, \hat k_2)\\
p_1 &= (E_1,-k_{2,x},-\omega_2\sin\theta_1\cos\theta_2,-\omega_2\cos\theta_1, -\hat k_2)
\end{align}
where $k_{2,x}=k_{2,x}(\omega_2,\theta_1,\theta_2,\hat k_2)$ is such, that $k_2^2=0$.

The only remainding choice is then the alignment of the z-axis: either along $k_1$, called ``Set I``(\ref{sec:Phasespace.2to3.Framework.SetI}), along q, called ``Set II``(\ref{sec:Phasespace.2to3.Framework.SetII}) or along $p_1$, called ``Set III``(\ref{sec:Phasespace.2to3.Framework.SetIII}).

\subsubsection{Set I}
\label{sec:Phasespace.2to3.Framework.SetI}
align $k_1$ to z-axis:
\begin{align}
k_1 &= (\omega_1,0,0,\omega_1, \hat 0)\\
q &= (q_0,0,\abs{\vec p_2}\sin\psi,\abs{\vec p_2}\cos\psi-\omega_1, \hat 0)\\
p_2 &= (E_2,0,\abs{\vec p_2}\sin\psi,\abs{\vec p_2}\cos\psi, \hat 0)
\end{align}

constraints, from energy conservation, masses (light quark masses are already fixed: $k_1^2 = 0 = k_2^2$) and external Mandelstam variables:
\begin{align}
q_0+\omega_1 & &&=E_1+E_2+\omega_2 \label{eq:PSConIMomCon}\\
m^2 &= p_1^2 &&= E_1^2 - \omega_2^2 \label{eq:PSConIp1}\\
m^2 &= p_2^2 &&= E_2^2 - \abs{\vec p_2}^2 \label{eq:PSConIp2} \\
q^2 & &&= q_0^2 - \abs{\vec p_2}^2 + 2\abs{\vec p_2}\omega_1\cos\psi-\omega_1^2 \label{eq:PSConIq}\\
s &= (q+k_1)^2 &&= (q_0+\omega_1)^2 - \abs{\vec p_2}^2 \label{eq:PSConIs}\\
t &= (k_1-p_2)^2 &&= (\omega_1-E_2)^2-\abs{\vec p_2}^2+2\abs{\vec p_2}\omega_1\cos\psi-\omega_1^2 \label{eq:PSConIt}\\
u &= (q-p_2)^2 &&= (q_0-E_2)^2 - \omega_1^2 \label{eq:PSConIu}
\end{align}

solve:
\begin{align}
(\ref{eq:PSConIs}) - (\ref{eq:PSConIq}) + (\ref{eq:PSConIt}) - (\ref{eq:PSConIp2}) + (\ref{eq:PSConIu}) &= s-q^2+t-m^2+u\\
 &= s_4+m^2 = (E_1+\omega_2)^2 \label{eq:PSConIs4}\\
(\ref{eq:PSConIt}) + (\ref{eq:PSConIu}) - (\ref{eq:PSConIq}) &= t+u-q^2 = -2(E_1+\omega_2)E_2\\
\Rightarrow E_2 &= -\frac{t+u-q^2}{2\sqrt{s_4+m^2}} =\frac{s-s_4-2m^2}{2\sqrt{s_4+m^2}} \label{eq:PSConIE2}\\
(\ref{eq:PSConIs4})\land (\ref{eq:PSConIp1}) \Rightarrow \omega_2 &= \frac{s_4}{2\sqrt{s_4+m^2}} \label{eq:PSConIom2} \\
(\ref{eq:PSConIs4})\Rightarrow E_1 &= \frac{s_4+2m^2}{2\sqrt{s_4+m^2}} \label{eq:PSConIE1} \\
(\ref{eq:PSConIs})+(\ref{eq:PSConIu})-(\ref{eq:PSConIp2}) &= s+u-m^2 =2q_0(E_1+\omega_2)\\
\Rightarrow q_0 &= \frac{s+u_1}{2\sqrt{s_4+m^2}} \label{eq:PSConIq0}\\
(\ref{eq:PSConIt}) - (\ref{eq:PSConIq}) &= t-q^2 = (\omega_1-E_2)^2 - q_0^2\\
\Rightarrow \omega_1 &= \frac{s'+t_1}{2\sqrt{s_4+m^2}} = \frac{s_4-u_1}{2\sqrt{s_4+m^2}}\label{eq:PSConIom1}\\
(\ref{eq:PSConIp2}) \Rightarrow \abs{\vec p_2} &= \sqrt{E_2^2 - m^2} = \frac {\sqrt{(s-s_4)^2-4sm^2}} {2\sqrt{s_4+m^2}}\label{eq:PSConIabsp2}\\
(\ref{eq:PSConIq}) \Rightarrow \cos\psi &= \frac{q^2-q_0^2+\abs{\vec p_2}^2+\omega_1^2}{2\abs{\vec p_2}\omega_1}\\
 &= \frac {(u_1+m^2)(t_1-s')-(m^2-q^2-t_1)(s'+t_1)} {(s'+t_1)\sqrt{(s-s_4)^2-4sm^2}}\label{eq:PSConIcospsi}\\
\Rightarrow \sin\psi &= 2\frac{\sqrt{s_4+m^2}\sqrt{m^2s'^2+q^2t_1(s'+t_1)-s't_1u_1}}{(s'+t_1)\sqrt{(s-s_4)^2-4sm}}
\end{align}

\begin{align}
&&t' &= -2k_1k_2 = -2\omega_1\omega_2(1-\cos\theta_1)\\
&&u_6 &= - 2k_1p_1 = -2\omega_1(E_1 + \omega_2\cos\theta_1)\\
(\ref{eq:MomCon3wk1}): &&0 &= s+t_1+t'+u_6-q^2\quad\checkmark
\end{align}

$t'$ is the only variable that can get collinear (for $-q^2 > 0$).

\begin{align}
&&s_3 &= 2k_2p_2 = 2\omega_2(E_2-\abs{\vec p_2}(\cos\psi\cos\theta_1+\sin\psi\sin\theta_1\cos\theta_2)) \label{eq:PSConIs3}\\
&&s_5 &= (p_1+p_2)^2 = 2m^2+2p_1p_2\\
&& &= 2(m^2+E_1E_2+\omega_2\abs{\vec p_2}(\cos\psi\cos\theta_1+\sin\psi\sin\theta_1\cos\theta_2)) \label{eq:PSConIs5}\\
(\ref{eq:MomCon3wp2}): &&0 &= s_3+s_5+t_1+u_1-q^2\quad\checkmark
\end{align}

\begin{align}
&&u' &=(q-k_2)^2 = q^2-2qk_2\\
&& &= q^2-2(q_0\omega_2 -\omega_2(\abs{\vec p_2}\cos\psi-\omega_1)\cos\theta_1-\omega_2\abs{\vec p_2}\sin\psi\sin\theta_1\cos\theta_2)\\
&&u_7 &= q^2-2qp_1\\
&& &=q^2 - 2(q_0E_1 +\omega_2(\abs{\vec p_2}\cos\psi-\omega_1)\cos\theta_1 + \omega_2\abs{\vec p_2}\sin\psi\sin\theta_1\cos\theta_2)\\
(\ref{eq:MomCon3wq}): &&0 &=s+u_1+u_7+u'-2q^2\quad\checkmark
\end{align}


\subsubsection{Set II}
\label{sec:Phasespace.2to3.Framework.SetII}
align $q$ to z-axis:
\begin{align}
q &= (q_0,0,0,q_z,\hat 0)\\
k_1 &= (\omega_1,0,\abs{\vec p_2}\sin\psi,\abs{\vec p_2}\cos\psi-q_z, \hat 0)\\
p_2 &= (E_2,0,\abs{\vec p_2}\sin\psi,\abs{\vec p_2}\cos\psi, \hat 0)
\end{align}

constraints, from energy conservation, masses ($k_2^2 = 0$ is already fixed) and external Mandelstam variables:
\begin{align}
q_0+\omega_1 & &&=E_1+E_2+\omega_2 \label{eq:PSConIIMomCon}\\
m^2 &= p_1^2 &&= E_1^2 - \omega_2^2 \label{eq:PSConIIp1}\\
m^2 &= p_2^2 &&= E_2^2 - \abs{\vec p_2}^2 \label{eq:PSConIIp2} \\
q^2 & &&= q_0^2 - q_z^2 \label{eq:PSConIIq}\\
0 & &&= \omega_1^2 - \abs{\vec p_2}^2 + 2\abs{\vec p_2}q_z\cos\psi-q_z^2 \label{eq:PSConIIk1}\\
s &= (q+k_1)^2 &&= (q_0+\omega_1)^2 - \abs{\vec p_2}^2 \label{eq:PSConIIs}\\
t &= (k_1-p_2)^2 &&= (\omega_1-E_2)^2-q_z^2 \label{eq:PSConIIt}\\
u &= (q-p_2)^2 &&= (q_0-E_2)^2 - \abs{\vec p_2}^2+2\abs{\vec p_2}q_z\cos\psi - q_z^2 \label{eq:PSConIIu}
\end{align}

solve:
\begin{align}
(\ref{eq:PSConIIu})-(\ref{eq:PSConIIk1}) &= u=(q_0-E_2)^2-\omega_1^2 \label{eq:PSConIIu2}\\
(\ref{eq:PSConIIs}) - (\ref{eq:PSConIIq}) + (\ref{eq:PSConIIt}) - (\ref{eq:PSConIIp2}) + (\ref{eq:PSConIIu2}) &= s-q^2+t-m^2+u\\
 &= s_4+m^2 = (\omega_1+q_0-E_2)^2 \label{eq:PSConIIs4}\\
(\ref{eq:PSConIIt}) + (\ref{eq:PSConIIu}) - (\ref{eq:PSConIIq}) &= t+u-q^2 = 2(E_2-\omega_1-q_0)E_2\\
\Rightarrow E_2 &= -\frac{t+u-q^2}{2\sqrt{s_4+m^2}} =\frac{s-s_4-2m^2}{2\sqrt{s_4+m^2}} = (\ref{eq:PSConIE2}) \label{eq:PSConIIE2}\\
(\ref{eq:PSConIIs}) - (\ref{eq:PSConIIq}) + (\ref{eq:PSConIIt}) - (\ref{eq:PSConIIp2}) &= s-q^2+t-m^2 = -2(E_2-\omega_1-q_0)\omega_1\\
\Rightarrow \omega_1 &= \frac{s'+t_1}{2\sqrt{s_4+m^2}} =\frac{s_4-u_1}{2\sqrt{s_4+m^2}} = (\ref{eq:PSConIom1}) \label{eq:PSConIIom1}\\
(\ref{eq:PSConIIs4}) \land (\ref{eq:PSConIIE2}) \land (\ref{eq:PSConIIom1}) \Rightarrow q_0 &= \frac{s+u_1}{2\sqrt{s_4+m^2}} = (\ref{eq:PSConIq0}) \label{eq:PSConIIq0}\\
(\ref{eq:PSConIIs}) \land (\ref{eq:PSConIIq0}) \land (\ref{eq:PSConIIom1}) \Rightarrow \abs{\vec p_2} &= \frac{\sqrt{(s-s_4)^2-4m^2 s}}{2\sqrt{s_4+m^2}} = (\ref{eq:PSConIabsp2}) \label{eq:PSConIIabsp2}\\
(\ref{eq:PSConIIq}) \land (\ref{eq:PSConIIq0}) \Rightarrow q_z &= \frac{\sqrt{(s'+u_1')^2-4q^2t}}{2\sqrt{s_4+m^2}} \label{eq:PSConIIqz}\\
(\ref{eq:PSConIIp1}) \Rightarrow \omega_2 &= \frac{s_4}{2\sqrt{s_4+m^2}} = (\ref{eq:PSConIom2}) \label{eq:PSConIIom2}\\
(\ref{eq:PSConIIMomCon}) \Rightarrow E_1 &= \frac{s_4+2m^2}{2\sqrt{s_4+m^2}} = (\ref{eq:PSConIE1}) \label{eq:PSConIIE1}\\
(\ref{eq:PSConIIq}) \Rightarrow \cos\psi &= \frac{s(q^2-t_1-2m^2)+s_4 u_1' - q^2(s_4+2m^2)}{\sqrt{((s-s_4)^2-4m^2s)((s'+u_1')^2-4q^2 t)}} \neq (\ref{eq:PSConIcospsi}) \label{eq:PSConIIcospsi}
\end{align}

\begin{align}
&&u' &=(q-k_2)^2 = q^2-2qk_2 &&= q^2-2(q_0\omega_2 -q_z\omega_2\cos\theta_1\\
&&u_7 &= q^2-2qp_1 &&=q^2 - 2(q_0E_1 +q_z\omega_2\cos\theta_1)\\
(\ref{eq:MomCon3wq}): &&0 &=s'+u_1'+u_7+u'\quad\checkmark
\end{align}

\begin{align}
&&s_3 &= 2k_2p_2 = 2\omega_2(E_2-\abs{\vec p_2}(\cos\psi\cos\theta_1+\sin\psi\sin\theta_1\cos\theta_2)) = (\ref{eq:PSConIs3})\\
&&s_5 &= (p_1+p_2)^2 = 2m^2+2p_1p_2\\
&& &= 2(m^2+E_1E_2+\omega_2\abs{\vec p_2}(\cos\psi\cos\theta_1+\sin\psi\sin\theta_1\cos\theta_2)) = (\ref{eq:PSConIs5})\\
(\ref{eq:MomCon3wp2}): &&0 &= s_3+s_5+t_1+u_1'\quad\checkmark
\end{align}


\begin{align}
&&t' &= -2k_1k_2\\
 &&&= -2\omega_2(\omega_1 - (\abs{\vec p_2}\cos\psi-q_z)\cos\theta_1 - \abs{\vec p_2}\sin\psi\sin\theta_1\cos\theta_2)\\
&&u_6 &= - 2k_1p_1\\
 &&&= -2(\omega_1 E_1 + \omega_2(\abs{\vec p_2}\cos\psi-q_z)\cos\theta_1 + \omega_2\abs{\vec p_2}\sin\psi\sin\theta_1\cos\theta_2)\\
(\ref{eq:MomCon3wk1}): &&0 &= s'+t_1+t'+u_6\quad\checkmark
\end{align}

same as before: $t'$ is the only variable that can get collinear (for $-q^2 > 0$) - now collinear as [ABC] variable.


\subsubsection{Set III}
\label{sec:Phasespace.2to3.Framework.SetIII}
align $p_2$ to z-axis:
\begin{align}
q &= (q_0,0,\abs{\vec q\,}\sin\psi,\abs{\vec q\,}\cos\psi,\hat 0)\\
k_1 &= (\omega_1,0,-\abs{\vec q\,}\sin\psi,-\abs{\vec q\,}\cos\psi+p_{2,z}, \hat 0)\\
p_2 &= (E_2,0,0,p_{2,z}, \hat 0)
\end{align}

constraints, from energy conservation, masses ($k_2^2 = 0$ is already fixed) and external Mandelstam variables:
\begin{align}
q_0+\omega_1 & &&=E_1+E_2+\omega_2 \label{eq:PSConIIIMomCon}\\
m^2 &= p_1^2 &&= E_1^2 - \omega_2^2 \label{eq:PSConIIIp1}\\
m^2 &= p_2^2 &&= E_2^2 - p_{2,z}^2 \label{eq:PSConIIIp2} \\
q^2 & &&= q_0^2 - \abs{\vec q\,}^2 \label{eq:PSConIIIq}\\
0 & &&= \omega_1^2 - \abs{\vec q\,}^2 + 2\abs{\vec q\,}p_{2,z}\cos\psi-p_{2,z}^2 \label{eq:PSConIIIk1}\\
s &= (q+k_1)^2 &&= (q_0+\omega_1)^2 - p_{2,z}^2 \label{eq:PSConIIIs}\\
t &= (k_1-p_2)^2 &&= (\omega_1-E_2)^2- \abs{\vec q\,}^2 \label{eq:PSConIIIt}\\
u &= (q-p_2)^2 &&= (q_0-E_2)^2 - \abs{\vec q\,}^2 + 2\abs{\vec q\,}p_{2,z}\cos\psi-p_{2,z}^2 \label{eq:PSConIIIu}
\end{align}

solve:
\begin{align}
(\ref{eq:PSConIIIu})-(\ref{eq:PSConIIIk1}) &= u=(q_0-E_2)^2-\omega_1^2 \label{eq:PSConIIIu2}\\
(\ref{eq:PSConIIIs}) - (\ref{eq:PSConIIIq}) + (\ref{eq:PSConIIIt}) - (\ref{eq:PSConIIIp2}) + (\ref{eq:PSConIIIu2}) &= s-q^2+t-m^2+u\\
 &= s_4+m^2 = (\omega_1+q_0-E_2)^2 =(\ref{eq:PSConIIs4}) \label{eq:PSConIIIs4}\\
(\ref{eq:PSConIIIt}) + (\ref{eq:PSConIIIu}) - (\ref{eq:PSConIIIq}) &= t+u-q^2 = 2(E_2-\omega_1-q_0)E_2\\
\Rightarrow E_2 &= -\frac{t+u-q^2}{2\sqrt{s_4+m^2}} =\frac{s-s_4-2m^2}{2\sqrt{s_4+m^2}} = (\ref{eq:PSConIE2}) = (\ref{eq:PSConIIE2}) \label{eq:PSConIIIE2}\\
(\ref{eq:PSConIIIs}) - (\ref{eq:PSConIIIq}) + (\ref{eq:PSConIIIt}) - (\ref{eq:PSConIIIp2}) &= s-q^2+t-m^2 = -2(E_2-\omega_1-q_0)\omega_1\\
\Rightarrow \omega_1 &= \frac{s'+t_1}{2\sqrt{s_4+m^2}} =\frac{s_4-u_1}{2\sqrt{s_4+m^2}} = (\ref{eq:PSConIom1}) = (\ref{eq:PSConIIom1})\label{eq:PSConIIIom1}\\
(\ref{eq:PSConIIIs4}) \land (\ref{eq:PSConIIIE2}) \land (\ref{eq:PSConIIIom1}) \Rightarrow q_0 &= \frac{s+u_1}{2\sqrt{s_4+m^2}} = (\ref{eq:PSConIq0}) = (\ref{eq:PSConIIq0}) \label{eq:PSConIIIq0}\\
(\ref{eq:PSConIIIq}) \land (\ref{eq:PSConIIIq0}) \Rightarrow \abs{\vec q\,} &= \frac{\sqrt{(s'+u_1')^2-4q^2t}}{2\sqrt{s_4+m^2}} =(\ref{eq:PSConIIqz}) \label{eq:PSConIIabsq}\\
(\ref{eq:PSConIIIp1})\land(\ref{eq:PSConIIIMomCon}) \Rightarrow E_1 &= \frac{s_4+2m^2}{2\sqrt{s_4+m^2}} = (\ref{eq:PSConIE1}) = (\ref{eq:PSConIIE1}) \label{eq:PSConIIIE1}\\
(\ref{eq:PSConIIIMomCon}) \Rightarrow \omega_2 &= \frac{s_4}{2\sqrt{s_4+m^2}} = (\ref{eq:PSConIom2}) = (\ref{eq:PSConIIom2}) \label{eq:PSConIIIom2}\\
(\ref{eq:PSConIIIp2}) \land (\ref{eq:PSConIIIE2}) \Rightarrow p_{2,z} &= \frac{\sqrt{(s-s_4)^2-4m^2 s}}{2\sqrt{s_4+m^2}} = (\ref{eq:PSConIabsp2}) = (\ref{eq:PSConIIabsp2}) \label{eq:PSConIIIabsp2}\\
(\ref{eq:PSConIIIq}) \Rightarrow \cos\psi &= \frac{s(q^2-t-m^2)- 2q^2(s_4+m^2)+s_4 u_1}{\sqrt{((s-s_4)^2-4m^2s)((s'+u_1')^2-4q^2 t)}} \label{eq:PSConIIIcospsi}\\
 &=(\ref{eq:PSConIIcospsi}) \neq (\ref{eq:PSConIcospsi}) \nonumber
\end{align}


\begin{align}
&&s_3 &= 2k_2p_2 = 2\omega_2(E_2-p_{2,z}\cos\theta_1)\\
&&s_5 &= (p_1+p_2)^2 = 2m^2+2p_1p_2\\
&& &= 2(m^2+E_1E_2+\omega_2p_{2,z}\cos\theta_1)\\
(\ref{eq:MomCon3wp2}): &&0 &= s_3+s_5+t_1+u_1'\quad\checkmark
\end{align}

\begin{align}
&&u' &=(q-k_2)^2 = q^2-2qk_2 \\
 &&&= q^2-2\omega_2(q_0 -\abs{\vec q\,}(\cos\psi\cos\theta_1 + \sin\psi\sin\theta_1\cos\theta_2))\\
&&u_7 &= q^2-2qp_1\\
 &&&=q^2 - 2(q_0E_1 +\abs{\vec q\,}\omega_2(\cos\psi\cos\theta_1 + \sin\psi\sin\theta_1\cos\theta_2)))\\
(\ref{eq:MomCon3wq}): &&0 &=s'+u_1'+u_7+u'\quad\checkmark
\end{align}

\begin{align}
&&t' &= -2k_1k_2\\
 &&&= -2\omega_2(\omega_1 + (\abs{\vec q\,}\cos\psi-p_{2,z})\cos\theta_1 + \abs{\vec q\,}\sin\psi\sin\theta_1\cos\theta_2)\\
&&u_6 &= - 2k_1p_1\\
 &&&= -2(\omega_1 E_1 - \omega_2(\abs{\vec q\,}\cos\psi-p_{2,z})\cos\theta_1 - \omega_2\abs{\vec q\,}\sin\psi\sin\theta_1\cos\theta_2)\\
(\ref{eq:MomCon3wk1}): &&0 &= s'+t_1+t'+u_6\quad\checkmark
\end{align}

same as before: $t'$ is the only variable that can get collinear (for $-q^2 > 0$) - now collinear as [ABC] variable.

