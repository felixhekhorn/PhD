% LaTeX Vorlage
\documentclass[
  ngerman,		% Sprache
  a4paper,		% Papierformat
  11pt,			% Schriftgröße (default 10pt)
  DIV=12,		% Seiteneinteilung
  parskip=half  	% Absätze (full,half,false -+*)
]{scrartcl}
%\documentclass[a4paper,10pt]{article}


\usepackage[status=draft]{fixme}

\usepackage[utf8]{inputenc}
\usepackage[ngerman]{babel} % Sprache 
\usepackage{amsmath, amssymb}
\usepackage{graphicx} % Grafiken einbinden
\usepackage[left=2cm,right=2cm,top=2.5cm,bottom=3cm]{geometry}
\usepackage{color}
%\usepackage{bbm}

%\usepackage{pdflscape}
\usepackage{ulem}
\usepackage{simplewick}
\usepackage{array}
\usepackage{feynmf}
\usepackage{slashed}

%\usepackage{mathpazo}
%\usepackage{breqn}

\providecommand{\abs}[1]{\left|#1\right|}
\providecommand{\VektorV}[3]{
\!\left(\!\!
\begin{array}{c}
#1 \\ #2 \\ #3
\end{array}\!\!
\right)\!
}

\providecommand{\Det}[9]{
	\begin{vmatrix}
	    #1 & #2 & #3 \\
	    #4 & #5 & #6 \\
	    #7 & #8 & #9 \\	    
	\end{vmatrix}
}

\DeclareMathOperator{\Grad}{\text{grad}}
\DeclareMathOperator{\Div}{\text{div}}
\DeclareMathOperator{\Rot}{\text{rot}}
\DeclareMathOperator{\tr}{\text{tr}}

\DeclareMathOperator{\acos}{\text{arccos}}
\DeclareMathOperator{\asin}{\text{arcsin}}
\DeclareMathOperator{\atanh}{\text{artanh}}
\DeclareMathOperator{\x}{\times}
\DeclareMathOperator{\cdt}{\!\cdot\!}
\DeclareMathOperator{\del}{\partial}
\DeclareMathOperator{\EqualClaim}{\stackrel{!}{=}}
\DeclareMathOperator{\equivals}{\mathrel{\widehat{=}}}
\providecommand{\Nabla}[0]{\vec\nabla}
\providecommand{\ex}[1]{e^{#1}}
\providecommand{\EE}[1]{\cdot 10^{#1}}
\providecommand{\FT}[1]{\mathcal{FT}\left[#1\right]}
\providecommand{\Mel}[1]{\mathcal{M}\left[#1\right]}
\providecommand{\invMel}[1]{\mathcal{M}^{-1}\left[#1\right]}

\providecommand{\dt}[0]{\Derive t}
\providecommand{\dx}[0]{\Derive x}
\providecommand{\Derive}[1]{\DeriveN{#1}{}}
\providecommand{\DeriveN}[2]{\DeriveNF {#1}{#2}{}}
\providecommand{\DeriveF}[2]{\DeriveNF {#1}{}{#2}}
\providecommand{\DeriveNF}[3]{\frac {d^{#2}#3} {d #1^{#2}}}
\providecommand{\dtP}[0]{\DeriveP t}
\providecommand{\dxP}[0]{\DeriveP x}
\providecommand{\DeriveP}[1]{\DerivePN{#1}{}}
\providecommand{\DerivePF}[2]{\DerivePNF {#1} {} {#2}}
\providecommand{\DerivePN}[2]{\DerivePNF {#1} {#2} {} }
\providecommand{\DerivePNF}[3]{\frac {\partial^{#2}#3} {\partial #1^{#2}}}
\providecommand{\DerivePMF}[3]{\frac {\partial^{2}#3} {\partial #1 \partial #2}}
\providecommand{\e}[1]{\hat{e}_{#1}}
\providecommand{\pFq}[2]{{}_{#1}F_{#2}}

\providecommand{\bra}[1]{\langle#1\rvert}
\providecommand{\ket}[1]{\lvert#1\rangle}
\providecommand{\bracket}[2]{\langle#1\vert#2\rangle}
\providecommand{\normOrd}[1]{\,:\!#1\!:\,}
\providecommand{\wContr}[3]{\contraction{}{#1}{#2}{#3}#1#2#3}

\DeclareMathOperator{\und}{\text{UND}}
\DeclareMathOperator{\oder}{\text{ODER}}

\DeclareMathOperator{\Md}{\mathcal M}
\DeclareMathOperator{\Ld}{\mathcal L}
\DeclareMathOperator{\Hd}{\mathcal H}
\DeclareMathOperator{\Nd}{\hat {\mathcal N}}
\DeclareMathOperator{\To}{\hat {\mathcal T}}

\DeclareMathOperator{\ijI}{\mathit{ij},\mathbf{I}}
\DeclareMathOperator{\MSbar}{\overline{\text{MS}}}


\DeclareRobustCommand{\PQ}{\HepGenParticle{Q}{}{}\xspace} % quark
\DeclareRobustCommand{\PaQ}{\HepGenAntiParticle{Q}{}{}\xspace} % anti-quark


\begin{document}

\section{2 to 2 phase space}

process:
\begin{equation}
\gamma^*(q) + g(k_1) \rightarrow Q(p_1)+\bar{Q}(p_2)
\end{equation}

use c.m.s. of incoming particles:
\begin{align}
q &= \left(\frac {s+q^2}{2\sqrt s},0,0,\ldots,-\frac{s-q^2}{2\sqrt s}\right) \\
k_1 &= \frac {s-q^2}{2\sqrt s}\left(1,0,0,\ldots,1\right)
\end{align}
such that
\begin{equation}
q+k_1 = (\sqrt s,\vec 0)\qquad k_1^2 = 0
\end{equation}
for the outgoing particles it follows
\begin{align}
p_1 &= \frac{\sqrt s} 2 \left(1,0,\beta\sin\theta,\ldots,\beta\cos\theta\right)\\
p_2 &= \frac{\sqrt s} 2 \left(1,0,-\beta\sin\theta,\ldots,-\beta\cos\theta\right)
\end{align}
with $\beta = \sqrt{1-4m^2/s}$ such that
\begin{equation}
p_1+p_2 = (\sqrt s,\vec 0)\qquad p_1^2 = p_2^2 = m^2
\end{equation}

use n-sphere:
\begin{equation}
d^Dx = \Omega_D x^{D-1} dx = \frac{2\pi^{D/2}}{\Gamma(D/2)}x^{D-1} dx= \frac{\pi^{D/2}}{\Gamma(D/2)}(x^2)^{(D-2)/2} dx^2
\end{equation}

compute phase space:
\begin{align}
PS_2 &= \int\!\!\frac{d^np_1}{(2\pi)^{n-1}}\frac{d^np_1}{(2\pi)^{n-1}} (2\pi)^n\delta^{(n)}(q+k_1-p_1-p_2)\delta(p_1^2-m^2)\delta(p_2^2 - m^2)\\
 &= \frac 1 {(2\pi)^{n-2}}\int\!\!d^np_1\,\delta((q+k_1-p_2)^2 -m^2) \delta(p_1^2-m^2)\\
 &= \frac 1 {(2\pi)^{n-2}}\int\!\!dp_{1,0}dp_{1,||}d^2p_{1,\perp}d^{n-4}\hat p_1\,\delta(s-2p_{1,0}\sqrt s) \delta(p_{1,0}^2-p_{1,||}^2 - p_{1,\perp}^2 - \hat p_1^2-m^2)\\
 &= \frac \pi {(2\pi)^{n-2} 2\sqrt s}\int\!\!dp_{1,||}dp^2_{1,\perp}d^{n-4}\hat p_1\,\delta(s/4-p_{1,||}^2 - p_{1,\perp}^2 - \hat p_1^2-m^2)\\
 &= \frac \pi {(2\pi)^{n-2} 2\sqrt s}\int\!\!dp_{1,||}d\hat p_1^2\,\frac{\pi^{(n-4)/2}}{\Gamma((n-4)/2)}(\hat p_1^2)^{(n-6)/2}\\
 &= \frac 1 {2\sqrt s\Gamma((n-4)/2)(4\pi)^{(n-2)/2}}\int\!\!dp_{1,||}d\hat p_1^2\,(\hat p_1^2)^{(n-6)/2}
\end{align}
Integration borders are
\begin{equation}
p_{1,||} \in \frac {\sqrt s} 2\beta \cdot [-1,1] \qquad \hat p_1^2 \in \left(\frac{s \beta^2}{4}-p_{1,||}^2\right)\cdot [0,1]
\end{equation}

if cross section does not depend on hat-space:
\begin{align}
\int\!\!d\hat p_1^2\,(\hat p_1^2)^{(n-6)/2} &= \frac 2 {n-4}\left(\frac{s \beta^2}{4}-p_{1,||}^2\right)^{(n-4)/2}\\
\Rightarrow PS_2 &= \frac 1 {2\sqrt s\Gamma((n-2)/2)(4\pi)^{(n-2)/2}}\int\!\!dp_{1,||}\left(\frac{s \beta^2}{4}-p_{1,||}^2\right)^{(n-4)/2}
\end{align}

rewrite $p_{1,||}$ to $\cos\theta$:
\begin{equation}
p_{1,||} = \frac{\sqrt s} 2 \beta\cos\theta \Rightarrow\quad dp_{1,||} = \frac{\sqrt s} 2 \beta\,d\!\cos\theta,\quad \cos\theta\in[-1,1],\, \hat p_1^2 \in \frac{s \beta^2}{4}\left(1-\cos^2\theta\right)\cdot[0,1]
\end{equation}
rewrite $\cos\theta$ to $t_1 = (k_1-p_2)^2-m^2$:
\begin{equation}
\cos\theta = t_1 \Rightarrow\quad d\!\cos\theta = dt_1,\quad t_1\in [,],\, \hat p_1^2 \in \cdot[0,1]
\end{equation}


\section{2 to 3 phase space}

process:
\begin{equation}
\gamma^*(q) + q(k_1) \rightarrow Q(p_1)+\bar{Q}(p_2) + q(k_2)
\end{equation}

\subsection{kinematic constraints}

definitions of kinematic variables:
\begin{align}
s &= (q+k_1)^2 &\Rightarrow& &2qk_1 &= s-q^2\\
s_3 &= (k_2+p_2)^2-m^2  &\Rightarrow& &2k_2p_2 &= s_3\\
s_4 &= (k_2+p_1)^2-m^2  &\Rightarrow& &2k_2p_1 &= s_4\\
s_5 &= (p_1+p_2)^2 = -u_5  &\Rightarrow& &2p_1p_2 &= s_5-2m^2\\
t_1 &= (k_1-p_2)^2-m^2 = t - m^2  &\Rightarrow& &2k_1p_2 &=-t_1\\
t' &= (k_1-k_2)^2  &\Rightarrow& &2k_1k_2 &= -t'\\
u_1 &= (q-p_2)^2-m^2 = u - m^2  &\Rightarrow& &2qp_2 &=-u_1+q^2\\
u_6 &= (k_1-p_1)^2 - m^2  &\Rightarrow& &2k_1p_1 &=-u_6\\
u_7 &= (q-p_1)^2 - m^2  &\Rightarrow& &2qp_1 &=-u_7+q^2\\
u' &= (q-k_2)^2  &\Rightarrow& &2qk_2 &=-u'+q^2
\end{align}

impose momentum conservation:
\begin{equation}
q+k_1 = p_1+p_2+k_2
\end{equation}

contract with 2 times momentum:
\begin{align}
2q^2 && +s-q^2 &=-u_7+q^2 &-& u_1+q^2 &-& u'+q^2 &&\Leftrightarrow &0 &=s+u_1+u_7+u'-2q^2 \label{eq:MomCon3wq} \\
s-q^2 && +0 &= -u_6 &-& t_1 &-& t' &&\Leftrightarrow &0 &= s+t_1+t'+u_6-q^2 \label{eq:MomCon3wk1}\\
-u_7+q^2 && -u_6 &= 2m^2 &+& s_5-2m^2 &+& s_4 &&\Leftrightarrow &0 &= s_4+s_5+u_6+u_7-q^2 \label{eq:MomCon3wp1}\\
-u_1+q^2 && -t_1 &= s_5-2m^2 &+& 2m^2 &+& s_3 &&\Leftrightarrow &0 &= s_3+s_5+t_1+u_1-q^2 \label{eq:MomCon3wp2}\\
-u'+q^2 && -t' &= s_4 &+& s_3 &+& 0 &&\Leftrightarrow &0 &= s_3+s_4+t'+u'-q^2 \label{eq:MomCon3wk2}
\end{align}

\begin{align}
\frac 1 2 \left((\ref{eq:MomCon3wq}) + (\ref{eq:MomCon3wk1}) + (\ref{eq:MomCon3wp2}) - (\ref{eq:MomCon3wp1}) - (\ref{eq:MomCon3wk2})\right)=&0= s - q^2 + t_1 + u_1 - s_4
\end{align}

\subsection{computation}
use c.m.s. of recoiling heavy and light quark ($Q(p_1)$ and $q(k_2)$):
\begin{align}
k_2 &= (\omega_2,0,\ldots,0,\omega_2\sin\theta\sin\phi,\omega_2\sin\theta\cos\phi,\omega_2\cos\theta)\\
p_1 &= (E_1,0,\ldots,0,-\omega_2\sin\theta\sin\phi,-\omega_2\sin\theta\cos\phi,-\omega_2\cos\theta)\\
k_1 &= (\omega_1,0,\ldots,0,0,\omega_1)\\
q &= (q_0,0,\ldots,0,\abs{\vec p_2}\sin\psi,\abs{\vec p_2}\cos\psi-\omega_1)\\
p_2 &= (E_2,0,\ldots,0,\abs{\vec p_2}\sin\psi,\abs{\vec p_2}\cos\psi)
\end{align}

light quark masss already taken into account: $k_1^2 = 0 = k_2^2$

constraints:
\begin{align}
q_0+\omega_1 & &&=E_1+E_2+\omega_2 \label{eq:PSConMomCon}\\
m^2 &= p_1^2 &&= E_1^2 - \omega_1^2 \label{eq:PSConp1}\\
m^2 &= p_2^2 &&= E_2^2 - \abs{\vec p_2}^2 \label{eq:PSConp2} \\
q^2 & &&= q_0^2 - \abs{\vec p_2}^2 + 2\abs{\vec p_2}\omega_1\cos\psi-\omega_1^2 \label{eq:PSConq}\\
s &= (q+k_1)^2 &&= (q_0-\omega_1)^2 - \abs{\vec p_2}^2 \label{eq:PSCons}\\
t &= (k_1-p_2)^2 &&= (\omega_1-E_2)^2-\abs{\vec p_2}^2+2\abs{\vec p_2}\omega_1\cos\psi-\omega_1^2 \label{eq:PSCont}\\
u &= (q-p_2)^2 &&= (q_0-E_2)^2 - \omega_1^2 \label{eq:PSConu}
\end{align}

\begin{align}
(\ref{eq:PSCons}) - (\ref{eq:PSConq}) + (\ref{eq:PSCont}) - (\ref{eq:PSConp2}) + (\ref{eq:PSConu}) &= s-q^2+t-m^2+u\\
 &= s_4+m^2 = (E_1+\omega)^2 \label{eq:PSCons4}\\
(\ref{eq:PSCont}) + (\ref{eq:PSConu}) - (\ref{eq:PSConq}) &= t+u-q^2 = -2(E_1+\omega_2)E_2\\
\Rightarrow E_2 &= -\frac{t+u-q^2}{2\sqrt{s_4+m^2}} =\frac{s_4+2m^2}{2\sqrt{s_4+m^2}} \label{eq:PSConE2}\\
\Rightarrow \abs{\vec p_2} &= \sqrt{E_2^2 - m^2} = \frac {s_4} {2\sqrt{s_4+m^2}}\label{eq:PSConabsp2}\\
(\ref{eq:PSCons})+(\ref{eq:PSConu})-(\ref{eq:PSConp2}) &= s+u-m^2 =2q_0(E_1+\omega_2)\\
\Rightarrow q_0 &= \frac{s+u-m^2}{2\sqrt{s_4+m^2}} \label{eq:PSConq0}\\
(\ref{eq:PSCont}) - (\ref{eq:PSConq}) &= t-q^2 = (\omega_1-E_2)^2 - q_0^2\\
\Rightarrow \omega_1 &= \frac{s_4+m^2-u}{2\sqrt{s_4+m^2}} \label{eq:PSConom1}
\end{align}


\end{document}
