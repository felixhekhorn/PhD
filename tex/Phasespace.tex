% @author: Felix Hekhorn <felix.hekhorn@student.uni-tuebingen.de>
\documentclass[
  english,		% Sprache
  a4paper,		% Papierformat
  11pt,			% Schriftgröße (default 10pt)
  %DIV=12,		% Seiteneinteilung
  DIV=12,
  titlepage,
  toc=bibnumbered,
  parskip=full,  	% Absätze (full,half,false -+*)
  headings=normal,
  BCOR=12mm,
  numbers=noenddot
]{scrartcl}
%\documentclass[a4paper,10pt]{article}
\usepackage{scrtime,scrlfile,scrpage2}

\usepackage[status=draft]{fixme}
%\usepackage[status=final]{fixme}

\usepackage[utf8]{inputenc}
%\usepackage[ngerman]{babel} % Sprache 
\usepackage[ngerman,english,main=english]{babel} % Sprache 
\selectlanguage{english}
\usepackage{amsmath, amssymb}

%\usepackage{graphicx} % Grafiken einbinden
% The following is needed in order to make the code compatible
% with both latex/dvips and pdflatex.
\ifx\pdftexversion\undefined
\usepackage[dvips]{graphicx}
\else
\usepackage[pdftex]{graphicx}
\DeclareGraphicsRule{*}{mps}{*}{}
\fi

%\usepackage[left=2cm,right=2cm,top=2.5cm,bottom=3cm]{geometry}
\usepackage{color}
\usepackage{bbm}

\usepackage{pdflscape}
\usepackage{ulem}
\usepackage{url}
\usepackage{caption}
\usepackage{subcaption}
\usepackage{array}
\usepackage{multirow}
\usepackage{listings}
\usepackage{placeins}

\usepackage{siunitx} % SI Einheiten
\usepackage[version=3]{mhchem}
%\sisetup{
%	exponent-product = \!\cdot\!,
%	output-product = \cdot,
%	list-final-separator =  { und } ,
%	list-pair-separator = { und } ,
%	range-phrase = { bis },
%	output-decimal-marker = {,},
%	separate-uncertainty = true,
%	group-digits = false
%}

\usepackage{simplewick}
\usepackage{feynmf}
\usepackage{slashed}

%\usepackage{biblatex}
\usepackage[numbers]{natbib}
%\bibliographystyle{natdin}
%\bibliographystyle{kp}
\bibliographystyle{utphys}

\usepackage{hyperref}
%\hypersetup{colorlinks=false}
\usepackage{tabularx}

\providecommand{\abs}[1]{\left|#1\right|}
\providecommand{\VektorV}[3]{
\!\left(\!\!
\begin{array}{c}
#1 \\ #2 \\ #3
\end{array}\!\!
\right)\!
}

\providecommand{\Det}[9]{
	\begin{vmatrix}
	    #1 & #2 & #3 \\
	    #4 & #5 & #6 \\
	    #7 & #8 & #9 \\	    
	\end{vmatrix}
}

\DeclareMathOperator{\Grad}{\text{grad}}
\DeclareMathOperator{\Div}{\text{div}}
\DeclareMathOperator{\Rot}{\text{rot}}
\DeclareMathOperator{\tr}{\text{tr}}

\DeclareMathOperator{\acos}{\text{arccos}}
\DeclareMathOperator{\asin}{\text{arcsin}}
\DeclareMathOperator{\atanh}{\text{artanh}}
\DeclareMathOperator{\DiLog}{\text{Li}_2}
\DeclareMathOperator{\x}{\times}
\DeclareMathOperator{\cdt}{\!\cdot\!}
\DeclareMathOperator{\del}{\partial}
\DeclareMathOperator{\EqualClaim}{\stackrel{!}{=}}
\DeclareMathOperator{\equivals}{\mathrel{\widehat{=}}}
\providecommand{\Nabla}[0]{\vec\nabla}
\providecommand{\ex}[1]{e^{#1}}
\providecommand{\EE}[1]{\cdot 10^{#1}}
\providecommand{\FT}[1]{\mathcal{FT}\left[#1\right]}
\providecommand{\Mel}[1]{\mathcal{M}\left[#1\right]}
\providecommand{\invMel}[1]{\mathcal{M}^{-1}\left[#1\right]}

\providecommand{\dt}[0]{\Derive t}
\providecommand{\dx}[0]{\Derive x}
\providecommand{\Derive}[1]{\DeriveN{#1}{}}
\providecommand{\DeriveN}[2]{\DeriveNF {#1}{#2}{}}
\providecommand{\DeriveF}[2]{\DeriveNF {#1}{}{#2}}
\providecommand{\DeriveNF}[3]{\frac {d^{#2}#3} {d #1^{#2}}}
\providecommand{\dtP}[0]{\DeriveP t}
\providecommand{\dxP}[0]{\DeriveP x}
\providecommand{\DeriveP}[1]{\DerivePN{#1}{}}
\providecommand{\DerivePF}[2]{\DerivePNF {#1} {} {#2}}
\providecommand{\DerivePN}[2]{\DerivePNF {#1} {#2} {} }
\providecommand{\DerivePNF}[3]{\frac {\partial^{#2}#3} {\partial #1^{#2}}}
\providecommand{\DerivePMF}[3]{\frac {\partial^{2}#3} {\partial #1 \partial #2}}
\providecommand{\e}[1]{\hat{e}_{#1}}
\providecommand{\pFq}[2]{{}_{#1}F_{#2}}

\providecommand{\bra}[1]{\langle#1\rvert}
\providecommand{\ket}[1]{\lvert#1\rangle}
\providecommand{\bracket}[2]{\langle#1\vert#2\rangle}
\providecommand{\normOrd}[1]{\,:\!#1\!:\,}
\providecommand{\wContr}[3]{\contraction{}{#1}{#2}{#3}#1#2#3}

\DeclareMathOperator{\Md}{\mathcal M}
\DeclareMathOperator{\Ld}{\mathcal L}
\DeclareMathOperator{\Hd}{\mathcal H}
\DeclareMathOperator{\Nd}{\hat {\mathcal N}}
\DeclareMathOperator{\To}{\hat {\mathcal T}}

\DeclareMathOperator{\MSbar}{\overline{\text{MS}}}

\DeclareRobustCommand{\PQ}{\HepGenParticle{Q}{}{}\xspace} % heavy quark
\DeclareRobustCommand{\PaQ}{\HepGenAntiParticle{Q}{}{}\xspace} % heavy anti-quark

\def\MMa{{\texttt{Mathematica}}}
\def\HEPMath{\texttt{HEPMath}}
\def\FeynCalc{\texttt{FeynCalc}}
\def\LoopTools{\texttt{LoopTools}}
\def\QCDLoop{\texttt{QCDLoop}}


\begin{document}

\section{2 to 2 phase space}
following \cite{Marco}:

process:
\begin{equation}
\gamma^*(q) + g(k_1) \rightarrow Q(p_1)+\bar{Q}(p_2)
\end{equation}

kinematics:
\begin{align}
s &= (q+k_1)^2 &s' &=s-q^2\\
t &= (k_1-p_1)^2 &t_1 &= t-m^2\\
u &= (k_1-p_2)^2 &u_1 &= u-m^2
\end{align}

use c.m.s. of incoming particles:
\begin{align}
q &= \left(\frac {s+q^2}{2\sqrt s},0,0,\ldots,-\frac{s-q^2}{2\sqrt s}\right) \\
k_1 &= \frac {s-q^2}{2\sqrt s}\left(1,0,0,\ldots,1\right)
\end{align}
such that
\begin{equation}
q+k_1 = (\sqrt s,\vec 0)\qquad k_1^2 = 0
\end{equation}
for the outgoing particles it follows
\begin{align}
p_1 &= \frac{\sqrt s} 2 \left(1,0,\beta\sin\theta,\ldots,\beta\cos\theta\right)\\
p_2 &= \frac{\sqrt s} 2 \left(1,0,-\beta\sin\theta,\ldots,-\beta\cos\theta\right)
\end{align}
with $\beta = \sqrt{1-4m^2/s}$ such that
\begin{equation}
p_1+p_2 = (\sqrt s,\vec 0)\qquad p_1^2 = p_2^2 = m^2
\end{equation}

use n-sphere:
\begin{equation}
d^Dx = \Omega_D x^{D-1} dx = \frac{2\pi^{D/2}}{\Gamma(D/2)}x^{D-1} dx= \frac{\pi^{D/2}}{\Gamma(D/2)}(x^2)^{(D-2)/2} dx^2
\end{equation}

compute phase space:
\begin{align}
PS_2 &= \int\!\!\frac{d^np_1}{(2\pi)^{n-1}}\frac{d^np_1}{(2\pi)^{n-1}} (2\pi)^n\delta^{(n)}(q+k_1-p_1-p_2)\delta(p_1^2-m^2)\delta(p_2^2 - m^2)\\
 &= \frac 1 {(2\pi)^{n-2}}\int\!\!d^np_1\,\delta((q+k_1-p_2)^2 -m^2) \delta(p_1^2-m^2)\\
 &= \frac 1 {(2\pi)^{n-2}}\int\!\!dp_{1,0}dp_{1,||}d^2p_{1,\perp}d^{n-4}\hat p_1\,\delta(s-2p_{1,0}\sqrt s) \delta(p_{1,0}^2-p_{1,||}^2 - p_{1,\perp}^2 - \hat p_1^2-m^2)\\
 &= \frac \pi {(2\pi)^{n-2} 2\sqrt s}\int\!\!dp_{1,||}dp^2_{1,\perp}d^{n-4}\hat p_1\,\delta(s/4-p_{1,||}^2 - p_{1,\perp}^2 - \hat p_1^2-m^2)\\
 &= \frac \pi {(2\pi)^{n-2} 2\sqrt s}\int\!\!dp_{1,||}d\hat p_1^2\,\frac{\pi^{(n-4)/2}}{\Gamma((n-4)/2)}(\hat p_1^2)^{(n-6)/2}\\
 &= \frac 1 {2\sqrt s\Gamma((n-4)/2)(4\pi)^{(n-2)/2}}\int\!\!dp_{1,||}d\hat p_1^2\,(\hat p_1^2)^{(n-6)/2}
\end{align}
Integration borders are
\begin{equation}
p_{1,||} \in \frac {\sqrt s} 2\beta \cdot [-1,1] \qquad \hat p_1^2 \in \left(\frac{s \beta^2}{4}-p_{1,||}^2\right)\cdot [0,1]
\end{equation}

if cross section does not depend on hat-space:
\begin{align}
\int\!\!d\hat p_1^2\,(\hat p_1^2)^{(n-6)/2} &= \frac 2 {n-4}\left(\frac{s \beta^2}{4}-p_{1,||}^2\right)^{(n-4)/2}\\
\Rightarrow PS_2 &= \frac 1 {2\sqrt s\Gamma((n-2)/2)(4\pi)^{(n-2)/2}}\int\!\!dp_{1,||}\left(\frac{s \beta^2}{4}-p_{1,||}^2\right)^{(n-4)/2}
\end{align}

rewrite $p_{1,||}$ to $\cos\theta$:
\begin{equation}
p_{1,||} = \frac{\sqrt s} 2 \beta\cos\theta \Rightarrow\quad dp_{1,||} = \frac{\sqrt s} 2 \beta\,d\!\cos\theta,\quad \cos\theta\in[-1,1],\, \hat p_1^2 \in \frac{s \beta^2}{4}\left(1-\cos^2\theta\right)\cdot[0,1]
\end{equation}
rewrite $\cos\theta$ to $t_1 = (k_1-p_2)^2-m^2$:
\begin{equation}
\cos\theta = \frac{2t_1/s'+1}{\beta} \Rightarrow\quad d\!\cos\theta = \frac 2 {\beta s'} dt_1,\quad t_1\in \frac {s'} 2[-\beta -1,\beta -1],\, \hat p_1^2 \in (-m^2-\frac{st_1}{s'^2}(s'+t_1))\cdot[0,1]
\end{equation}
\begin{equation}
p_{1,||} = \sqrt s\left(\frac{t_1}{s'}+\frac{1} 2\right) \Rightarrow\quad  dp_{1,||} = \frac{\sqrt s}{s'} dt_1
\end{equation}
\begin{equation}
\Rightarrow PS_2 = \frac 1 {2s'\Gamma((n-2)/2)(4\pi)^{(n-2)/2}}\int\!\!dt_1\left(\frac{(t_1(u_1-q^2)-s'm^2)s' - q^2t_1^2}{s'^2}\right)^{(n-4)/2}
\end{equation}


\section{2 to 3 phase space}
following \cite{PhysRevD.40.54,van_neerven_dimensional_1986,Marco}:

process:
\begin{equation}
\gamma^*(q) + q(k_1) \rightarrow Q(p_1)+\bar{Q}(p_2) + q(k_2)
\end{equation}

\subsection{kinematic constraints}

definitions of kinematic variables:
\begin{align}
s &= (q+k_1)^2 &\Rightarrow& &2qk_1 &= s-q^2\\
s_3 &= (k_2+p_2)^2-m^2  &\Rightarrow& &2k_2p_2 &= s_3\\
s_4 &= (k_2+p_1)^2-m^2  &\Rightarrow& &2k_2p_1 &= s_4\\
s_5 &= (p_1+p_2)^2 = -u_5  &\Rightarrow& &2p_1p_2 &= s_5-2m^2\\
t_1 &= (k_1-p_2)^2-m^2 = t - m^2  &\Rightarrow& &2k_1p_2 &=-t_1\\
t' &= (k_1-k_2)^2  &\Rightarrow& &2k_1k_2 &= -t'\\
u_1 &= (q-p_2)^2-m^2 = u - m^2  &\Rightarrow& &2qp_2 &=-u_1+q^2\\
u_6 &= (k_1-p_1)^2 - m^2  &\Rightarrow& &2k_1p_1 &=-u_6\\
u_7 &= (q-p_1)^2 - m^2  &\Rightarrow& &2qp_1 &=-u_7+q^2\\
u' &= (q-k_2)^2  &\Rightarrow& &2qk_2 &=-u'+q^2
\end{align}

impose momentum conservation:
\begin{equation}
q+k_1 = p_1+p_2+k_2
\end{equation}

contract with 2 times momentum:
\begin{align}
2q^2 && +s-q^2 &=-u_7+q^2 &-& u_1+q^2 &-& u'+q^2 &&\Leftrightarrow &0 &=s+u_1+u_7+u'-2q^2 \label{eq:MomCon3wq} \\
s-q^2 && +0 &= -u_6 &-& t_1 &-& t' &&\Leftrightarrow &0 &= s+t_1+t'+u_6-q^2 \label{eq:MomCon3wk1}\\
-u_7+q^2 && -u_6 &= 2m^2 &+& s_5-2m^2 &+& s_4 &&\Leftrightarrow &0 &= s_4+s_5+u_6+u_7-q^2 \label{eq:MomCon3wp1}\\
-u_1+q^2 && -t_1 &= s_5-2m^2 &+& 2m^2 &+& s_3 &&\Leftrightarrow &0 &= s_3+s_5+t_1+u_1-q^2 \label{eq:MomCon3wp2}\\
-u'+q^2 && -t' &= s_4 &+& s_3 &+& 0 &&\Leftrightarrow &0 &= s_3+s_4+t'+u'-q^2 \label{eq:MomCon3wk2}
\end{align}

\begin{align}
\frac 1 2 \left((\ref{eq:MomCon3wq}) + (\ref{eq:MomCon3wk1}) + (\ref{eq:MomCon3wp2}) - (\ref{eq:MomCon3wp1}) - (\ref{eq:MomCon3wk2})\right)=&0= s - q^2 + t_1 + u_1 - s_4
\end{align}

\subsection{choose framework}
use c.m.s. of recoiling heavy and light quark ($Q(p_1)$ and $q(k_2)$):
\begin{align}
k_2 &= (\omega_2,k_{2,x},\omega_2\sin\theta_1\cos\theta_2,\omega_2\cos\theta_1, \hat k_2)\\
p_1 &= (E_1,-k_{2,x},-\omega_2\sin\theta_1\cos\theta_2,-\omega_2\cos\theta_1, -\hat k_2)\\
k_1 &= (\omega_1,0,0,\omega_1, \hat 0)\\
q &= (q_0,0,\abs{\vec p_2}\sin\psi,\abs{\vec p_2}\cos\psi-\omega_1, \hat 0)\\
p_2 &= (E_2,0,\abs{\vec p_2}\sin\psi,\abs{\vec p_2}\cos\psi, \hat 0)
\end{align}

light quark masses are already fixed: $k_1^2 = 0 = k_2^2$

constraints:
\begin{align}
q_0+\omega_1 & &&=E_1+E_2+\omega_2 \label{eq:PSConMomCon}\\
m^2 &= p_1^2 &&= E_1^2 - \omega_2^2 \label{eq:PSConp1}\\
m^2 &= p_2^2 &&= E_2^2 - \abs{\vec p_2}^2 \label{eq:PSConp2} \\
q^2 & &&= q_0^2 - \abs{\vec p_2}^2 + 2\abs{\vec p_2}\omega_1\cos\psi-\omega_1^2 \label{eq:PSConq}\\
s &= (q+k_1)^2 &&= (q_0+\omega_1)^2 - \abs{\vec p_2}^2 \label{eq:PSCons}\\
t &= (k_1-p_2)^2 &&= (\omega_1-E_2)^2-\abs{\vec p_2}^2+2\abs{\vec p_2}\omega_1\cos\psi-\omega_1^2 \label{eq:PSCont}\\
u &= (q-p_2)^2 &&= (q_0-E_2)^2 - \omega_1^2 \label{eq:PSConu}
\end{align}

solve:
\begin{align}
(\ref{eq:PSCons}) - (\ref{eq:PSConq}) + (\ref{eq:PSCont}) - (\ref{eq:PSConp2}) + (\ref{eq:PSConu}) &= s-q^2+t-m^2+u\\
 &= s_4+m^2 = (E_1+\omega_2)^2 \label{eq:PSCons4}\\
(\ref{eq:PSCont}) + (\ref{eq:PSConu}) - (\ref{eq:PSConq}) &= t+u-q^2 = -2(E_1+\omega_2)E_2\\
\Rightarrow E_2 &= -\frac{t+u-q^2}{2\sqrt{s_4+m^2}} =\frac{s-s_4-2m^2}{2\sqrt{s_4+m^2}} \label{eq:PSConE2}\\
(\ref{eq:PSCons4})\land (\ref{eq:PSConp1}) \Rightarrow \omega_2 &= \frac{s_4}{2\sqrt{s_4+m^2}} \label{eq:PSConom2} \\
(\ref{eq:PSCons4})\Rightarrow E_1 &= \frac{s_4+2m^2}{2\sqrt{s_4+m^2}} \label{eq:PSConE1} \\
(\ref{eq:PSCons})+(\ref{eq:PSConu})-(\ref{eq:PSConp2}) &= s+u-m^2 =2q_0(E_1+\omega_2)\\
\Rightarrow q_0 &= \frac{s+u_1}{2\sqrt{s_4+m^2}} \label{eq:PSConq0}\\
(\ref{eq:PSCont}) - (\ref{eq:PSConq}) &= t-q^2 = (\omega_1-E_2)^2 - q_0^2\\
\Rightarrow \omega_1 &= \frac{s_4-u_1}{2\sqrt{s_4+m^2}} \label{eq:PSConom1}\\
(\ref{eq:PSConp2}) \Rightarrow \abs{\vec p_2} &= \sqrt{E_2^2 - m^2} = \frac {\sqrt{(s-s_4)^2-4sm^2}} {2\sqrt{s_4+m^2}}\label{eq:PSConabsp2}\\
(\ref{eq:PSConq}) \Rightarrow \cos\psi &= \frac{q^2-q_0^2+\abs{\vec p_2}^2+\omega_1^2}{2\abs{\vec p_2}\omega_1}\\
 &= \frac {2u(q^2-s+t_1)-(m^2-q^2-t_1)(s_4-u_1)} {(s_4-u_1)\sqrt{(s-s_4)^2-4sm^2}}\label{eq:PSConabcos}
\end{align}

\begin{align}
&&t' &= -2k_1k_2 = -2\omega_1\omega_2(1-\cos\theta_1)\\
&&u_6 &= - 2k_1p_1 = -2\omega_1(E_1 + \omega_2\cos\theta_1)\\
(\ref{eq:MomCon3wk1}): &&0 &= s+t_1+t'+u_6-q^2\quad\checkmark
\end{align}

\begin{align}
&&s_3 &= 2k_2p_2 = 2\omega_2(E_2-\abs{\vec p_2}(\cos\psi\cos\theta_1+\sin\psi\sin\theta_1\cos\theta_2))\\
&&s_5 &= (p_1+p_2)^2 = 2m^2+2p_1p_2\\
&& &= 2(m^2+E_1E_2+\omega_2\abs{\vec p_2}(\cos\psi\cos\theta_1+\sin\psi\sin\theta_1\cos\theta_2))\\
(\ref{eq:MomCon3wp2}): &&0 &= s_3+s_5+t_1+u_1-q^2\quad\checkmark
\end{align}

\begin{align}
&&u' &=(q-k_2)^2 = q^2-2qk_2\\
&& &= q^2-2(q_0\omega_2 -\omega_2(\abs{\vec p_2}\cos\psi-\omega_1)\cos\theta_1-\omega_2\abs{\vec p_2}\sin\psi\sin\theta_1\cos\theta_2)\\
&&u_7 &= q^2-2qp_1\\
&& &=q^2 - 2(q_0E_1 +\omega_2(\abs{\vec p_2}\cos\psi-\omega_1)\cos\theta_1 + \omega_2\abs{\vec p_2}\sin\psi\sin\theta_1\cos\theta_2)\\
(\ref{eq:MomCon3wq}): &&0 &=s+u_1+u_7+u'-2q^2\quad\checkmark
\end{align}

\subsection{phase space integrals}
at phase space integration there occur integrations over propagators\cite{Bojak:2000eu,PhysRevD.40.54,van_neerven_dimensional_1986}; the propagators can be decomposed in 2 types: [ab] and [ABC]; the needed integrals then reduce to the master formula:
\begin{align}
I_n^{(k,l)} &=\int\limits_0^\pi\!d\theta_1\,\sin^{n-3}(\theta_1)\int\limits_0^\pi\!d\theta_2\,\sin^{n-4}(\theta_2)(a+b\cos(\theta_1))^{-k}(A+B\cos(\theta_1)+C\sin(\theta_1)\cos(\theta_2))^{-l}\\
&=\int\!d\Omega_n\,(a+b\cos(\theta_1))^{-k}(A+B\cos(\theta_1)+C\sin(\theta_1)\cos(\theta_2))^{-l}
\end{align}

the integrals can be further destinguished by the range of $k,l$ and the type of collinearity (following the notation in \cite{Bojak:2000eu}):
\begin{itemize}
\item "non collinear": $a^2\neq b^2 \land A^2 \neq B^2 + C^2 \rightarrow I_{0,n}^{(k,l)}$
\item "single collinear a": $a=-b \land A^2 \neq B^2 + C^2 \rightarrow I_{a,n}^{(k,l)}$
\item "single collinear A": $a^2\neq b^2 \land A^2 = B^2 + C^2 \rightarrow I_{A,n}^{(k,l)}$
\item "double collinear": $a=-b \land A = -\sqrt{B^2 + C^2} \rightarrow I_{aA,n}^{(k,l)}$
\end{itemize}

Use $n=4+\epsilon$.

\subsubsection{integral helper}
define helper integral
\begin{equation}
\hat I^{(q)}(\nu) := \int\limits_0^\pi\!dt\,\sin^{\nu-3}(t)\cos^{q}(t)
\end{equation}

It is\cite[eq. 5.12.6]{NIST:DLMF}:
\begin{equation}
\int_{0}^{\pi}(\mathop{\sin\/}\nolimits t)^{\alpha-1}e^{i\beta t}dt=\frac{\pi}{2^{\alpha-1}} \frac{e^{i\pi \beta/2}}{\alpha\mathop{B}\left((\alpha+\beta+1)/2,(\alpha-\beta+1)/2\right)} \qquad\text{if}\,\Re(\alpha) > 0
\end{equation}
\begin{align}
\Rightarrow \hat I^{(0)}(n) &= \frac{\pi}{2^{n-3}(n-2)}\frac 1 {\mathop{B}((n-1)/2,(n-1)/2)}\\
\Rightarrow \hat I^{(0)}(n-1) &= \frac{\pi}{2^{n-4}(n-3)}\frac 1 {\mathop{B}((n-2)/2,(n-2)/2)}=\mathop{B}((n-3)/2,1/2)
\end{align}

If q is odd: $\hat I^{(q)}=0$, due to symetry of kernel; if q is even: $q=2p$ with $p\in\mathbbm N$:
\begin{align}
\hat I^{(2p)}(\nu) &= \frac 1 {2^{2p}}\sum\limits_{k=0}^{2p} \binom{2p}{k} \int\limits_0^\pi\sin^{\nu-3}(t)\exp(2i(k-p)t)\,dt\\
 &= \frac \pi{2^{2p+\nu-3} (\nu-2)}\sum\limits_{k=0}^{2p} \binom{2p}{k}\frac{\exp(i\pi(k-p))}{\mathop{B}((\nu-1)/2+(k-p),(\nu-1)/2-(k-p))}\\
 &= \frac \pi{2^{2p+\nu-3} (\nu-2)}\sum\limits_{l=-p}^{p} \binom{2p}{p+l}\frac{(-1)^l}{\mathop{B}((\nu-1)/2+l,(\nu-1)/2-l)}\\
 &= \frac {\pi\Gamma(\nu-1)(2p)!}{2^{2p+\nu-3} (\nu-2)\Gamma(\frac{n-1} 2+p)\Gamma(\frac{n-1} 2+p)}\left(\frac 1 {(p!)^2} \frac{\Gamma(\frac{\nu-1} 2+p)}{\Gamma(\frac{\nu-1} 2)}\frac{\Gamma(\frac{\nu-1} 2-p)}{\Gamma(\frac{\nu-1} 2)} \right. \nonumber\\
 &\hspace{40pt}\left. + 2\sum\limits_{l=1}^{p}\frac{(-1)^l}{(p+l)!(p-l)!}\frac{\Gamma(\frac{\nu-1} 2+p)}{\Gamma(\frac{\nu-1} 2+l)}\frac{\Gamma(\frac{\nu-1} 2-p)}{\Gamma(\frac{\nu-1} 2-l)}\right)\\
 &= \frac {2^{3-\nu}\pi\Gamma(\nu-1)}{(\nu-2)\Gamma(\frac{n-1} 2+p)\Gamma(\frac{n-1} 2+p)}\cdot \frac{\Gamma(\frac{\nu-1} 2-p)}{2^{p}\Gamma(\frac{\nu-1} 2)} \cdot \frac{(2p)!}{2^p p!} \cdot p! \left(\frac 1 {(p!)^2} \frac{\Gamma(\frac{\nu-1} 2+p)}{\Gamma(\frac{\nu-1} 2)} \right. \nonumber\\
 &\hspace{40pt}\left. + 2\sum\limits_{l=1}^{p}\frac{(-1)^l}{(p+l)!(p-l)!}\frac{\Gamma(\frac{\nu-1} 2+p)}{\Gamma(\frac{\nu-1} 2+l)}\frac{\Gamma(\frac{\nu-1} 2)}{\Gamma(\frac{\nu-1} 2-l)}\right)
\end{align}
TODO: prove \fxerror{prove}
\begin{align}
&p! \left(\frac 1 {(p!)^2} \frac{\Gamma(\frac{\nu-1} 2+p)}{\Gamma(\frac{\nu-1} 2)} + 2\sum\limits_{l=1}^{p}\frac{(-1)^l}{(p+l)!(p-l)!}\frac{\Gamma(\frac{\nu-1} 2+p)}{\Gamma(\frac{\nu-1} 2+l)}\frac{\Gamma(\frac{\nu-1} 2)}{\Gamma(\frac{\nu-1} 2-l)}\right)\\
&= \frac {1} {p!}\frac{\Gamma(-\frac{1}2+p)}{\Gamma(-\frac{1}2)} + 2\sum\limits_{l=1}^{p}\frac{(-1)^l p!}{(p+l)!(p-l)!}\frac{\Gamma(-\frac{1} 2+p)}{\Gamma(-\frac{1} 2+l)}\frac{\Gamma(-\frac{1} 2)}{\Gamma(-\frac{1} 2-l)}\\
&= 1
\end{align}
\begin{align}
\Rightarrow \hat I^{(2p)}(\nu) &=\frac {2^{3-\nu}\pi\Gamma(\nu-1)}{(\nu-2)\Gamma(\frac{n-1} 2+p)\Gamma(\frac{n-1} 2-p)} \cdot \frac{\Gamma(\frac{\nu-1} 2-p)}{2^{p}\Gamma(\frac{\nu-1} 2)} \cdot \frac{(2p!)}{2^p p!}\\
 &= \frac{\sqrt{\pi}(2p)!}{2^{2p}p!} \frac {\Gamma((\nu-2)/2)} {\Gamma(\frac{\nu-1}2+p)}
\end{align}

\subsubsection{any collinearity and $-k,-l\in\mathbbm N_0$}
If $-k,-l\in\mathbbm N_0$ $I_{n}^{(k,l)}$ can always be reduced in a straight forward manner to combinations of $\hat I^{(q)}(n)$ and this way one finds\cite[Ch. 5]{Bojak:2000eu}\cite[App. C]{PhysRevD.40.54}:
\begin{align}
I^{(0,0)}_{n} &= \hat I^{(0)}(n-1) \cdot \hat I^{(0)}(n) = \frac{2\pi}{n-3}\\
I^{(0,0)}_{4} &= 2\pi\\
I^{(-1,0)}_{n} &= \hat I^{(0)}(n-1) \cdot (a\hat I^{(0)}(n)+b\hat I^{(1)}(n)) = \frac{2\pi a}{n-3}\\
I^{(-1,0)}_{4} &= 2\pi a \\
I^{(0,-1)}_{n} &= \hat I^{(0)}(n-1) \cdot (A\hat I^{(0)}(n) + B\hat I^{(1)}(n)) + C\hat I^{(1)}(n-1)\hat I^{(0)}(n)\\
 &= \frac{2\pi A}{n-3}\\
I^{(0,-1)}_{4} &=2\pi A\\
I^{(-2,0)}_{n} &= \hat I^{(0)}(n-1) \cdot (a^2\hat I^{(0)}(n)+2ab\hat I^{(1)}(n) + b^2 \hat I^{(2)}(n))\\
 &= 2\pi\left(\frac{a^2(n-1)+b^2}{(n-1)(n-3)}\right)\\
I^{(-2,0)}_{4} &= 2\pi(a^2 + b^2/3) \\
I^{(0,-2)}_{n} &= \hat I^{(0)}(n-1) \cdot (A^2\hat I^{(0)}(n) + B^2\hat I^{(2)}(n)) + C^2\hat I^{(2)}(n-1)\hat I^{(0)}(n+2) \\
 &= 2\pi\left(\frac{A^2(n-1)+B^2+C^2}{(n-1)(n-3)}\right)\\
I^{(0,-2)}_{4} &= 2\pi(A^2+(B^2+C^2)/3) \\
I^{(-1,-1)}_{n} &= \hat I^{(0)}(n-1) \cdot (a A\hat I^{(0)}(n) + b B \hat I^{(2)}(n)) = 2\pi\left(\frac{aA(n-1)+bB}{(n-1)(n-3)}\right)\\
I^{(-1,-1)}_{4} &= 2\pi(aA + bB/3)
\end{align}

we will need:
\begin{align}
\int\!d\Omega_n\,t' &= -2\pi \omega_1\omega_2 &&= -\pi\frac{s_4(s_4-u_1)}{2(s_4+m^2)}\\
\int\!d\Omega_n\,u_7 &= \frac{2\pi}{n-3}(q^2-2q_0E_1)
\end{align}

\subsubsection{single collinear a and $k,-l\in \mathbbm N_0$}
It is
\begin{align}
\hat I_{a}^{(k,q)}(\nu) &= \int\limits_0^\pi\!\frac{\sin^{\nu-3}t}{(1-\cos(t))^k}\cos^q(t)\, dt \\
 &= \int\limits_0^\pi\!\frac{\sin^{\nu-3}(t)}{(1-\cos^2(t))^k}\cos^q(t)(1+\cos(t))^k\, dt\\
 &=\int\limits_0^\pi\!\sin^{\nu-3-2k}(t)\cos^q(t)(1+\cos(t))^k\, dt\\
 &= \sum\limits_{l=0}^k\binom{k}{l}\hat I^{(q+l)}(\nu-2k)
\end{align}

this way one finds\cite[Ch. 5]{Bojak:2000eu}\cite[App. C]{PhysRevD.40.54}:
\begin{align}
I^{(1,0)}_{a,n} &= \frac 1 a\hat I^{(0)}(n-1) \cdot \hat I^{(0)}(n-2)\\
 &= \frac {2\pi}{a(n-4)}\\
I^{(2,0)}_{a,n} &= \frac 1 a^2\hat I^{(0)}(n-1) \cdot \left(\hat I^{(0)}(n-4) + \hat I^{(2)}(n-4)\right)\\
 &= \frac {2\pi}{a^2(n-6)} \approx -\frac {\pi}{a^2} + O(\epsilon)\\
I^{(1,-1)}_{a,n} &= \frac 1 a\hat I^{(0)}(n-1)\cdot \left(A\hat I^{(0)}(n-2)+B\hat I^{(2)}(n-2)\right)\\
 &= \frac {2\pi}{a}\frac{(A(n-3)+B)}{(n-3)(n-4)} \approx \frac {2\pi}{a}\left(\frac{A+B}\epsilon - 2B + O(\epsilon)\right)\\
I^{(1,-2)}_{a,n} &= \frac 1 a\left(\hat I^{(0)}(n-1)\cdot \left(A^2\hat I^{(0)}(n-2)+(B^2+2AB)\hat I^{(2)}(n-2)\right) + C^2\hat I^{(2)}(n-1)\hat I^{(0)}(n)\right)\\
 &= \frac {2\pi}{a}\left(\frac {A^2}{n-4} + \frac {2AB + B^2}{(n-4)(n-3)} + \frac {C^2}{(n-3)(n-2)}\right) \\
 &\approx \frac {2\pi}{a}\left(\frac{(A+B)^2}{\epsilon}+\frac{C^2}{2}-2AB-B^2+O(\epsilon)\right)\\
I^{(2,-2)}_{a,n} &= \frac 1 {a^2}\left(\hat I^{(0)}(n-1)\cdot \left(A^2(\hat I^{(0)}(n-4)+\hat I^{(2)}(n-4))+4AB\hat I^{(2)}(n-4) \right.\right.\nonumber\\
 &\hspace{20pt}\left.\left. + B^2(\hat I^{(2)}(n-4)+\hat I^{(4)}(n-4))\right) + C^2\hat I^{(2)}(n-1)(\hat I^{(0)}(n-2) + \hat I^{(2)}(n-2))\right)\\
 &= \frac {2\pi}{a^2}\left(\frac {A^2}{n-6}+\frac{4AB}{(n-6)(n-4)} + \frac{B^2 n}{(n-6)(n-4)(n-3)} + \frac{C^2}{(n-4)(n-3)} \right) \\
 &\approx \frac {2\pi}{a^2}\left(\frac{-2AB-2B^2+C^2}{\epsilon}+\frac{B^2-A^2}{2}-AB-C^2+O(\epsilon)\right)\\
\end{align}

\subsubsection{double collinear and $k,l\in\mathbbm N$}
as said in \cite[Ch. 5]{Bojak:2000eu}: if $0\leq -\frac{C}{A},\frac B A \leq 1$ use \cite[eq. A11]{van_neerven_dimensional_1986} with $\cos\kappa = -\frac B A$:
\begin{equation}
I_{aA,n}^{(k,l)} = \frac{2\pi 2^{-(k+l)}}{a^kA^l}\frac{\Gamma(1+\epsilon)}{\Gamma^2(1+\epsilon/2)} B(1+\frac \epsilon 2 - k, 1+ \frac \epsilon 2 -l) \pFq 2 1 \left(k,l;1+\frac \epsilon 2; \frac{A-B}{2A}\right)
\end{equation}

\subsubsection{non collinear and $k,-l\in\mathbbm N$}
Next we want to compute $I_{0,n}^{1,-3}$.

The $\theta_2$ integration can be performed using the integral helper and the problem reduces then to the following integral:
\begin{align}
&\hat I_{0,\cos}^{(k,q,p)}(\epsilon) \nonumber\\
&= \int\limits_0^\pi\!d\theta_1\, \frac{\sin^{1+\epsilon}(\theta_1)\sin^q(\theta_1) \cos^p(\theta_1)}{(a+b\cos(\theta_1))^k}\\
&= \frac 1 {2a^k}\left((1+(-1)^p)B\left(\frac{2+q+\epsilon}{2},\frac{1+p}{2}\right)\pFq 3 2 \left(\frac{1+k}{2},\frac k 2,\frac{1+p}{2};\frac{1}{2},\frac{3+q+p+\epsilon}{2};\frac{b^2}{a^2}\right)\right.\nonumber\\
 &\hspace{20pt}\left. \frac b a k(-1+(-1)^p)B\left(\frac{2+q+\epsilon}{2},\frac{2+p}{2}\right)\pFq 3 2\left(\frac{1+k} 2,\frac{2+k}{2},\frac{2+p}{2};\frac{3}{2},\frac{4+q+p+\epsilon}{2};\frac{b^2}{a^2}\right) \right)
\end{align}
for $k=1$ this simplifies to
\begin{align}
&\hat I_{0,\cos}^{(1,q,p)}(\epsilon) \nonumber\\
&= \int\limits_0^\pi\!d\theta_1\, \frac{\sin^{1+\epsilon}(\theta_1)\sin^q(\theta_1) \cos^p(\theta_1)}{(a+b\cos(\theta_1))}\\
&= \frac 1 {2a}\left((1+(-1)^p)B\left(\frac{2+q+\epsilon}{2},\frac{1+p}{2}\right)\pFq 2 1 \left(1,\frac{1+p}{2};\frac{3+q+p+\epsilon}{2};\frac{b^2}{a^2}\right)\right.\nonumber\\
 &\hspace{20pt}\left. \frac b a (-1+(-1)^p)B\left(\frac{2+q+\epsilon}{2},\frac{2+p}{2}\right)\pFq 2 1\left(1,\frac{2+p}{2};\frac{4+q+p+\epsilon}{2};\frac{b^2}{a^2}\right) \right)
\end{align}

\appendix

\bibliography{ref.bib}
\listoffixmes

\end{document}
