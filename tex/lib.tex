\providecommand{\abs}[1]{\left|#1\right|}
\providecommand{\VektorV}[3]{
\!\left(\!\!
\begin{array}{c}
#1 \\ #2 \\ #3
\end{array}\!\!
\right)\!
}

\providecommand{\Det}[9]{
	\begin{vmatrix}
	    #1 & #2 & #3 \\
	    #4 & #5 & #6 \\
	    #7 & #8 & #9 \\	    
	\end{vmatrix}
}

\DeclareMathOperator{\Grad}{\text{grad}}
\DeclareMathOperator{\Div}{\text{div}}
\DeclareMathOperator{\Rot}{\text{rot}}
\DeclareMathOperator{\tr}{\text{tr}}

\DeclareMathOperator{\acos}{\text{arccos}}
\DeclareMathOperator{\asin}{\text{arcsin}}
\DeclareMathOperator{\atanh}{\text{artanh}}
\DeclareMathOperator{\DiLog}{\text{Li}_2}
\DeclareMathOperator{\x}{\times}
\DeclareMathOperator{\cdt}{\!\cdot\!}
\DeclareMathOperator{\del}{\partial}
\DeclareMathOperator{\EqualClaim}{\stackrel{!}{=}}
\DeclareMathOperator{\equivals}{\mathrel{\widehat{=}}}
\providecommand{\Nabla}[0]{\vec\nabla}
\providecommand{\ex}[1]{e^{#1}}
\providecommand{\EE}[1]{\cdot 10^{#1}}
\providecommand{\FT}[1]{\mathcal{FT}\left[#1\right]}
\providecommand{\Mel}[1]{\mathcal{M}\left[#1\right]}
\providecommand{\invMel}[1]{\mathcal{M}^{-1}\left[#1\right]}

\providecommand{\dt}[0]{\Derive t}
\providecommand{\dx}[0]{\Derive x}
\providecommand{\Derive}[1]{\DeriveN{#1}{}}
\providecommand{\DeriveN}[2]{\DeriveNF {#1}{#2}{}}
\providecommand{\DeriveF}[2]{\DeriveNF {#1}{}{#2}}
\providecommand{\DeriveNF}[3]{\frac {d^{#2}#3} {d #1^{#2}}}
\providecommand{\dtP}[0]{\DeriveP t}
\providecommand{\dxP}[0]{\DeriveP x}
\providecommand{\DeriveP}[1]{\DerivePN{#1}{}}
\providecommand{\DerivePF}[2]{\DerivePNF {#1} {} {#2}}
\providecommand{\DerivePN}[2]{\DerivePNF {#1} {#2} {} }
\providecommand{\DerivePNF}[3]{\frac {\partial^{#2}#3} {\partial #1^{#2}}}
\providecommand{\DerivePMF}[3]{\frac {\partial^{2}#3} {\partial #1 \partial #2}}
\providecommand{\e}[1]{\hat{e}_{#1}}
\providecommand{\pFq}[2]{{}_{#1}F_{#2}}

\providecommand{\bra}[1]{\langle#1\rvert}
\providecommand{\ket}[1]{\lvert#1\rangle}
\providecommand{\bracket}[2]{\langle#1\vert#2\rangle}
\providecommand{\normOrd}[1]{\,:\!#1\!:\,}
\providecommand{\wContr}[3]{\contraction{}{#1}{#2}{#3}#1#2#3}

\DeclareMathOperator{\Md}{\mathcal M}
\DeclareMathOperator{\Ld}{\mathcal L}
\DeclareMathOperator{\Hd}{\mathcal H}
\DeclareMathOperator{\Nd}{\hat {\mathcal N}}
\DeclareMathOperator{\To}{\hat {\mathcal T}}

\DeclareMathOperator{\MSbar}{\overline{\text{MS}}}
\DeclareMathOperator{\dPSTwo}{{d\text{PS}_2}}
\DeclareMathOperator{\dPSThree}{{d\text{PS}_3}}

\DeclareMathOperator{\tVV}{\text{VV}}
\DeclareMathOperator{\tVA}{\text{VA}}
\DeclareMathOperator{\tAA}{\text{AA}}
\DeclareMathOperator{\tOK}{\text{OK}}
\DeclareMathOperator{\tQED}{\text{QED}}
\DeclareMathOperator{\tThr}{\text{thr}}

\DeclareRobustCommand{\PQ}{\HepGenParticle{Q}{}{}\xspace} % heavy quark
\DeclareRobustCommand{\PaQ}{\HepGenAntiParticle{Q}{}{}\xspace} % heavy anti-quark
\DeclareRobustCommand{\PZx}{\HepGenParticle{Z}{}{*}\xspace} % virtual Z

\def\MMa{{\texttt{Mathematica}}}
\def\HEPMath{\texttt{HEPMath}}
\def\FeynCalc{\texttt{FeynCalc}}
\def\LoopTools{\texttt{LoopTools}}
\def\QCDLoop{\texttt{QCDLoop}}
