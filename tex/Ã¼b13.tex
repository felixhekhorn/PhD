% LaTeX Vorlage
\documentclass[
  ngerman,		% Sprache
  a4paper,		% Papierformat
  11pt,			% Schriftgröße (default 10pt)
  DIV=12,		% Seiteneinteilung
  parskip=half  	% Absätze (full,half,false -+*)
]{scrartcl}
%\documentclass[a4paper,10pt]{article}

\usepackage[utf8]{inputenc}
\usepackage[ngerman]{babel} % Sprache 
\usepackage{amsmath, amssymb}
\usepackage{graphicx} % Grafiken einbinden
\usepackage[left=2cm,right=2cm,top=2.5cm,bottom=3cm]{geometry}
\usepackage{color}
\usepackage{bbm}

%\usepackage{pdflscape}
\usepackage{ulem}
\usepackage{simplewick}
\usepackage{array}
\usepackage{feynmf}
\usepackage{slashed}

\providecommand{\abs}[1]{\left|#1\right|}
\providecommand{\VektorV}[3]{
\!\left(\!\!
\begin{array}{c}
#1 \\ #2 \\ #3
\end{array}\!\!
\right)\!
}

\providecommand{\Det}[9]{
	\begin{vmatrix}
	    #1 & #2 & #3 \\
	    #4 & #5 & #6 \\
	    #7 & #8 & #9 \\	    
	\end{vmatrix}
}

\DeclareMathOperator{\Grad}{\text{grad}}
\DeclareMathOperator{\Div}{\text{div}}
\DeclareMathOperator{\Rot}{\text{rot}}
\DeclareMathOperator{\tr}{\text{tr}}

\DeclareMathOperator{\acos}{\text{arccos}}
\DeclareMathOperator{\asin}{\text{arcsin}}
\DeclareMathOperator{\atanh}{\text{artanh}}
\DeclareMathOperator{\DiLog}{\text{Li}_2}
\DeclareMathOperator{\x}{\times}
\DeclareMathOperator{\cdt}{\!\cdot\!}
\DeclareMathOperator{\del}{\partial}
\DeclareMathOperator{\EqualClaim}{\stackrel{!}{=}}
\DeclareMathOperator{\equivals}{\mathrel{\widehat{=}}}
\providecommand{\Nabla}[0]{\vec\nabla}
\providecommand{\ex}[1]{e^{#1}}
\providecommand{\EE}[1]{\cdot 10^{#1}}
\providecommand{\FT}[1]{\mathcal{FT}\left[#1\right]}
\providecommand{\Mel}[1]{\mathcal{M}\left[#1\right]}
\providecommand{\invMel}[1]{\mathcal{M}^{-1}\left[#1\right]}

\providecommand{\dt}[0]{\Derive t}
\providecommand{\dx}[0]{\Derive x}
\providecommand{\Derive}[1]{\DeriveN{#1}{}}
\providecommand{\DeriveN}[2]{\DeriveNF {#1}{#2}{}}
\providecommand{\DeriveF}[2]{\DeriveNF {#1}{}{#2}}
\providecommand{\DeriveNF}[3]{\frac {d^{#2}#3} {d #1^{#2}}}
\providecommand{\dtP}[0]{\DeriveP t}
\providecommand{\dxP}[0]{\DeriveP x}
\providecommand{\DeriveP}[1]{\DerivePN{#1}{}}
\providecommand{\DerivePF}[2]{\DerivePNF {#1} {} {#2}}
\providecommand{\DerivePN}[2]{\DerivePNF {#1} {#2} {} }
\providecommand{\DerivePNF}[3]{\frac {\partial^{#2}#3} {\partial #1^{#2}}}
\providecommand{\DerivePMF}[3]{\frac {\partial^{2}#3} {\partial #1 \partial #2}}
\providecommand{\e}[1]{\hat{e}_{#1}}
\providecommand{\pFq}[2]{{}_{#1}F_{#2}}

\providecommand{\bra}[1]{\langle#1\rvert}
\providecommand{\ket}[1]{\lvert#1\rangle}
\providecommand{\bracket}[2]{\langle#1\vert#2\rangle}
\providecommand{\normOrd}[1]{\,:\!#1\!:\,}
\providecommand{\wContr}[3]{\contraction{}{#1}{#2}{#3}#1#2#3}

\DeclareMathOperator{\Md}{\mathcal M}
\DeclareMathOperator{\Ld}{\mathcal L}
\DeclareMathOperator{\Hd}{\mathcal H}
\DeclareMathOperator{\Nd}{\hat {\mathcal N}}
\DeclareMathOperator{\To}{\hat {\mathcal T}}

\DeclareMathOperator{\MSbar}{\overline{\text{MS}}}

\DeclareRobustCommand{\PQ}{\HepGenParticle{Q}{}{}\xspace} % heavy quark
\DeclareRobustCommand{\PaQ}{\HepGenAntiParticle{Q}{}{}\xspace} % heavy anti-quark

\def\MMa{{\texttt{Mathematica}}}
\def\HEPMath{\texttt{HEPMath}}
\def\FeynCalc{\texttt{FeynCalc}}
\def\LoopTools{\texttt{LoopTools}}
\def\QCDLoop{\texttt{QCDLoop}}


\title{QFT - Übungsblatt Nr. 13}
\author{Felix Hekhorn, Benedikt Anlauf}
\subtitle{ Gruppe A: Felix Ringer }
\date{\today}

\begin{document}
\maketitle

\setcounter{section}{38}
\def\thesubsection {\thesection.\alph{subsection}}

\section{Chirale Darstellung}
\subsection{}
\[
\left.\begin{array}{c}
(\gamma^0)^2 = \mathbbm 1 \quad \{\gamma^0,\gamma^k\} = 0 \\
\gamma^j\gamma^k =
\left(\begin{array}{cc}
-\sigma^j\sigma^k & 0\\
0 & \sigma^j\sigma^k
\end{array}\right) \quad \sigma^j\sigma^k = \delta^{jk} + i\epsilon^{jkl}\sigma_l \Rightarrow \{\gamma^j,\gamma^k\} = -2\delta^{jk}
\end{array}\right\} \Rightarrow \{\gamma^\mu,\gamma^\nu\}=2g^{\mu\nu} \checkmark
 \]
\[ \gamma^0\gamma^1\gamma^2\gamma^3 = \gamma^5 \checkmark \]
\subsection{}
\begin{align*}
\Ld &= \bar\psi(\slashed \partial - m)\psi\\
 &=
(\phi^\dag, \chi^\dag)
\left(\begin{array}{cc}
-m &\partial_0 + \sigma^j\partial_j\\
\partial_0 - \sigma^j\partial_j & -m
\end{array}\right)
\left(\begin{array}{c}\phi\\\chi\end{array}\right) \\
 &=
(\phi^\dag, \chi^\dag)
\left(\begin{array}{c}
-m\phi + (\partial_0 + \sigma^j\partial_j)\chi\\
(\partial_0 - \sigma^j\partial_j)\phi - m\chi
\end{array}\right) \\
 &= \phi^\dag(\partial_0 - \sigma^j\partial_j)\phi +  \chi^\dag(\partial_0 + \sigma^j\partial_j)\chi -m\phi^\dag\phi -m\chi^\dag\chi - m\phi^\dag\chi - m\chi^\dag\phi\\
&= \phi^\dag(\partial_0 - \sigma^j\partial_j)\phi +  \chi^\dag(\partial_0 + \sigma^j\partial_j)\chi - m\phi^\dag\chi - m\chi^\dag\phi
\end{align*}
\[\text{denn:}\quad \phi^\dag\phi = \bar\psi_L\psi_L = \frac 1 4\bar\psi(\mathbbm 1 + \gamma^5)(\mathbbm 1 - \gamma^5)\psi = 0\]

\section{Impulsoperator}
\begin{align*}
\Nabla\psi &= \sum\limits_r\int\!\!\frac{d^3p}{(2\pi)^{3/2}2E_{p}}
 \hat c^{(r)}(\vec p)u^{(r)}(p)\Nabla\exp(-ipx) + \hat d^{(r)\dag}(\vec p)v^{(r)}(p)\Nabla\exp(ipx) \\
 &= i\sum\limits_r\int\!\!\frac{d^3p\,\vec p}{(2\pi)^{3/2}2E_{p}}
 \hat c^{(r)}(\vec p)u^{(r)}(p)\exp(-ipx) - \hat d^{(r)\dag}(\vec p)v^{(r)}(p)\exp(ipx)
\end{align*}
\begin{align*}
\hat {\vec P} &= -i\int\!\!d^3x\,\psi^\dag(x)\Nabla\psi\\
 &= \sum\limits_{r,s}\int\!\!\frac{d^3pd^3q\,\vec q}{2E_{p}2E_q}\!\int\!\!\frac{d^3x}{(2\pi)^3}
 \left( \hat c^{(r)\dag}(\vec p)u^{(r)\dag}(p)\exp(+ipx) + \hat d^{(r)}(\vec p)v^{(r)^\dag}(p)\exp(-ipx) \right)\\
 &\hspace{120pt}\left( \hat c^{(s)}(\vec q)u^{(s)}(q)\exp(-iqx) - \hat d^{(s)\dag}(\vec q)v^{(s)}(q)\exp(iqx) \right)\\
 &= \sum\limits_{r}\int\!\!\frac{d^3p}{2E_{p}} \vec p\left( \hat c^{(r)\dag}\hat c^{(r)} - \hat d^{(r)}\hat d^{(r)\dag}\right)
\end{align*}
mit selben Argumenten wie immer: $\exp(\ldots) d^3x$ liefert $\delta^3(\vec p\pm \vec q)$; die $\delta^3(\vec p+ \vec q)$ werden durch die $(u,v)$-Produkte eliminiert; bei den $\delta^3(\vec p- \vec q)$-Beiträgen werden durch die $(u,v)$-Produkte die Spins gleichgesetzt;
\[ \Rightarrow \normOrd{\hat {\vec P}} = \sum\limits_{r}\int\!\!\frac{d^3p}{2E_{p}} \vec p\left( \hat c^{(r)\dag}\hat c^{(r)} + \hat d^{(r)\dag}\hat d^{(r)}\right) \]

\section{Wick-Theorem}
\begin{align*}
&\bra 0 \To[\psi_a(x_1)\psi_b(x_2)\psi_c(x_3)\hat\psi_d(x_4)\hat\psi_e(x_5)\hat\psi_f(x_6)] \ket 0\\
 &=
  (-)^{2+1} S^F_{ad}(x_1-x_4)S^F_{be}(x_2-x_5)S^F_{cf}(x_3-x_6)
+ (-)^{2}   S^F_{ad}(x_1-x_4)S^F_{bf}(x_2-x_6)S^F_{ce}(x_3-x_5) \\
&+(-)^{3+1} S^F_{ae}(x_1-x_5)S^F_{bd}(x_2-x_4)S^F_{cf}(x_3-x_6)
+ (-)^{3}   S^F_{ae}(x_1-x_5)S^F_{bf}(x_2-x_6)S^F_{cd}(x_3-x_4) \\
&+(-)^{4+1} S^F_{af}(x_1-x_6)S^F_{bd}(x_2-x_4)S^F_{ce}(x_3-x_5)
+ (-)^{4}   S^F_{af}(x_1-x_6)S^F_{be}(x_2-x_5)S^F_{cd}(x_3-x_4) \\
\end{align*}

\section{Yukawa}
\subsection{}
\[\bra f \hat S \ket i = \frac 1 2(-ig)^2\int\!\!d^3xd^3y \bra f \To[\bar\psi_a(x)\psi_a(x)\phi(x)\bar\psi_b(y)\psi_b(y)\phi(y)] \ket i\]
\begin{align*}
&\bra f \To[\bar\psi_a(x)\psi_a(x)\phi(x)\bar\psi_b(y)\psi_b(y)\phi(y)] \ket i\\
 &= \bra f \normOrd{\bar\psi_a(x)\psi_b(y)}\normOrd{\phi(x)\phi(y)} \ket i S^F_{ab}(x-y) +
	\bra f \normOrd{\psi_a(x)\bar\psi_b(y)}\normOrd{\phi(x)\phi(y)} \ket i S^F_{ab}(x-y)
\end{align*}
\begin{align*}
%&\bra 0 \hat a_1\hat c_1\hat c^{\dag}_2\hat c_3\hat a^\dag_2\hat a_3 \hat c^\dag_4\hat a^\dag_4 \ket 0\\
% &=[\hat a_1,\hat a^\dag_2][\hat a_3,\hat a^\dag_4]\\
&\bra 0 \hat a(\vec k_1)\hat c^{(r)}(\vec k_2)\hat c^{(w_1)\dag}(\vec q_1)\hat c^{(w_2)}(\vec q_2)\hat a^\dag(\vec \kappa_1)\hat a(\vec \kappa_2) \hat c^{(s)\dag}(\vec p_2)\hat a^\dag(\vec p_1)\ket 0\\
&= \bra 0 \hat c^{(r)}(\vec k_2)\hat c^{(w_1)\dag}(\vec q_1)\ket 0 \bra 0\hat c^{(w_2)}(\vec q_2) \hat c^{(s)\dag}(\vec p_2)\ket 0 \bra 0 \hat a(\vec k_1)\hat a^\dag(\vec \kappa_1)\ket 0 \bra 0\hat a(\vec \kappa_2)\hat a^\dag(\vec p_1)\ket 0\\
 &= \{\hat c^{(r)}(\vec k_2),\hat c^{(w_1)\dag}(\vec q_1)\} \{\hat c^{(w_2)}(\vec q_2),\hat c^{(s)}(\vec p_2)\} [\hat a(\vec k_1),\hat a^\dag(\vec \kappa_1)] [\hat a(\vec \kappa_2),\hat a^\dag(\vec p_1)]
\end{align*}
\begin{align*}
& \frac {(-ig)^2} 2\!\int\!\!d^3xd^3y \bra f\normOrd{\bar\psi_a(x)\psi_b(y)}\normOrd{\phi(x)\phi(y)} \ket i S^F_{ab}(x-y)\\
 &= \frac {(-ig)^2} 2\!\int\!\!d^3xd^3y \left( \prod\limits_{j=1}^2 \sum\limits_{w_j}\int\!\!\frac {d^3q_jd^3\kappa_j}{(2\pi)^{3/2}2E_q2E_\kappa} \right) S^F_{ab}(x-y)\\
&\hspace{80pt} \exp(+i\kappa_1x - i\kappa_2y)2E_{\kappa_1}\delta^3(\vec \kappa_1 - \vec k_1)2E_{\kappa_2}\delta^3(\vec \kappa_2 - \vec p_1)\\
&\hspace{80pt} \exp(+iq_1x - iq_2y) \bar u^{(w_1)}_a(q_1) u^{(w_2)}_b(q_2)2E_{q_1}\delta^3(\vec q_1 - \vec k_2)\delta^{w_1,s}2E_{q_2}\delta^3(\vec q_2 - \vec p_2)\delta^{w_2,r}\\
 &= \frac {(-ig)^2} 2 ((2\pi)^{3/2})^{-4} 2\!\int\!\!d^3xd^3y \!\int\!\! \frac {d^4k}{(2\pi)^4}\frac {i\bar u^{(s)}(p_2)(\slashed k + M)u^{(r)}(k_2)}{k^2 - M^2 + i\epsilon}\exp(i(k_1 + k_2 - k)x - i(p_1 + p_2 + k)y)\\
 &=\frac {(-ig)^2} 2 \bar u^{(s)}(p_2)\frac {i(\slashed k_1 + \slashed k_2 + M)}{(k_1+k_2)^2 - M^2}u^{(r)}(k_2) \cdot \frac {(2\pi)^4\delta(k_1 + k_2 - p_1- p_2)}{((2\pi)^{3/2})^{4}}
\end{align*}
anderere Beitrag analog

\subsection{}
\begin{center}\begin{tabular}{
>{\centering\vspace{40pt}}m{180pt}
>{\centering\vspace{40pt}}m{180pt}
}
\begin{fmffile}{A42-a-1}
	\begin{fmfgraph*}(60,50)
	\fmfleft{Ni,Pi}
	\fmfright{Po,No}

	\fmf{dashes}{Pi,v2}
	\fmf{dashes}{v1,Po}
	\fmf{fermion}{Ni,v1,v2,No}

	\fmfforce{.5w,.2h}{v1}
	\fmfforce{.5w,.8h}{v2}

	\fmfdot{v1,v2}
	\fmflabel{$k_1$}{Pi}
	\fmflabel{$k_2,r$}{Ni}
	\fmflabel{$p_1$}{Po}
	\fmflabel{$p_2,s$}{No}

	\end{fmfgraph*}
\end{fmffile}
&
\begin{fmffile}{A42-a-2}
	\begin{fmfgraph*}(60,50)
	\fmfleft{Ni,Pi}
	\fmfright{Po,No}

	\fmf{dashes}{Pi,v1}
	\fmf{dashes}{v2,Po}
	\fmf{fermion}{Ni,v1,v2,No}

	\fmfforce{.2w,.5h}{v1}
	\fmfforce{.8w,.5h}{v2}

	\fmfdot{v1,v2}
	\fmflabel{$k_1$}{Pi}
	\fmflabel{$k_2,r$}{Ni}
	\fmflabel{$p_1$}{Po}
	\fmflabel{$p_2,s$}{No}

	\end{fmfgraph*}
\end{fmffile}
\end{tabular}\end{center}
\[ iM_{fi} = (-ig)^2\bar u^{(s)}(p_2) i\left(
\frac {\slashed k_2 - \slashed p_1 + M}{(k_2-p_1)^2 - M^2} + \frac {\slashed k_1 + \slashed k_2 + M}{(k_1+k_2)^2 - M^2}
\right) u^{(r)}(k_2) \]

\end{document}
