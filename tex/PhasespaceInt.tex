% LaTeX Vorlage
\documentclass[
  english,		% Sprache
  a4paper,		% Papierformat
  11pt,			% Schriftgröße (default 10pt)
  DIV=12,		% Seiteneinteilung
  parskip=half  	% Absätze (full,half,false -+*)
]{scrartcl}
%\documentclass[a4paper,10pt]{article}


\usepackage[status=draft]{fixme}

\usepackage[utf8]{inputenc}
%\usepackage[ngerman]{babel} % Sprache 
\usepackage[ngerman,english,main=english]{babel} % Sprache 
\usepackage{amsmath, amssymb}
\usepackage{graphicx} % Grafiken einbinden
\usepackage[left=2cm,right=2cm,top=2.5cm,bottom=3cm]{geometry}
\usepackage{color}
\usepackage{bbm}

%\usepackage{pdflscape}
\usepackage{ulem}
\usepackage{simplewick}
\usepackage{array}
\usepackage{feynmf}
\usepackage{slashed}

%\usepackage{mathpazo}
%\usepackage{breqn}

\providecommand{\abs}[1]{\left|#1\right|}
\providecommand{\VektorV}[3]{
\!\left(\!\!
\begin{array}{c}
#1 \\ #2 \\ #3
\end{array}\!\!
\right)\!
}

\providecommand{\Det}[9]{
	\begin{vmatrix}
	    #1 & #2 & #3 \\
	    #4 & #5 & #6 \\
	    #7 & #8 & #9 \\	    
	\end{vmatrix}
}

\DeclareMathOperator{\Grad}{\text{grad}}
\DeclareMathOperator{\Div}{\text{div}}
\DeclareMathOperator{\Rot}{\text{rot}}
\DeclareMathOperator{\tr}{\text{tr}}

\DeclareMathOperator{\acos}{\text{arccos}}
\DeclareMathOperator{\asin}{\text{arcsin}}
\DeclareMathOperator{\atanh}{\text{artanh}}
\DeclareMathOperator{\x}{\times}
\DeclareMathOperator{\cdt}{\!\cdot\!}
\DeclareMathOperator{\del}{\partial}
\DeclareMathOperator{\EqualClaim}{\stackrel{!}{=}}
\DeclareMathOperator{\equivals}{\mathrel{\widehat{=}}}
\providecommand{\Nabla}[0]{\vec\nabla}
\providecommand{\ex}[1]{e^{#1}}
\providecommand{\EE}[1]{\cdot 10^{#1}}
\providecommand{\FT}[1]{\mathcal{FT}\left[#1\right]}
\providecommand{\Mel}[1]{\mathcal{M}\left[#1\right]}
\providecommand{\invMel}[1]{\mathcal{M}^{-1}\left[#1\right]}

\providecommand{\dt}[0]{\Derive t}
\providecommand{\dx}[0]{\Derive x}
\providecommand{\Derive}[1]{\DeriveN{#1}{}}
\providecommand{\DeriveN}[2]{\DeriveNF {#1}{#2}{}}
\providecommand{\DeriveF}[2]{\DeriveNF {#1}{}{#2}}
\providecommand{\DeriveNF}[3]{\frac {d^{#2}#3} {d #1^{#2}}}
\providecommand{\dtP}[0]{\DeriveP t}
\providecommand{\dxP}[0]{\DeriveP x}
\providecommand{\DeriveP}[1]{\DerivePN{#1}{}}
\providecommand{\DerivePF}[2]{\DerivePNF {#1} {} {#2}}
\providecommand{\DerivePN}[2]{\DerivePNF {#1} {#2} {} }
\providecommand{\DerivePNF}[3]{\frac {\partial^{#2}#3} {\partial #1^{#2}}}
\providecommand{\DerivePMF}[3]{\frac {\partial^{2}#3} {\partial #1 \partial #2}}
\providecommand{\e}[1]{\hat{e}_{#1}}
\providecommand{\pFq}[2]{{}_{#1}F_{#2}}

\providecommand{\bra}[1]{\langle#1\rvert}
\providecommand{\ket}[1]{\lvert#1\rangle}
\providecommand{\bracket}[2]{\langle#1\vert#2\rangle}
\providecommand{\normOrd}[1]{\,:\!#1\!:\,}
\providecommand{\wContr}[3]{\contraction{}{#1}{#2}{#3}#1#2#3}

\DeclareMathOperator{\und}{\text{UND}}
\DeclareMathOperator{\oder}{\text{ODER}}

\DeclareMathOperator{\Md}{\mathcal M}
\DeclareMathOperator{\Ld}{\mathcal L}
\DeclareMathOperator{\Hd}{\mathcal H}
\DeclareMathOperator{\Nd}{\hat {\mathcal N}}
\DeclareMathOperator{\To}{\hat {\mathcal T}}

\DeclareMathOperator{\ijI}{\mathit{ij},\mathbf{I}}
\DeclareMathOperator{\MSbar}{\overline{\text{MS}}}


\DeclareRobustCommand{\PQ}{\HepGenParticle{Q}{}{}\xspace} % quark
\DeclareRobustCommand{\PaQ}{\HepGenAntiParticle{Q}{}{}\xspace} % anti-quark


\begin{document}


\section{2 to 3 phase space}
\fxerror{Quote Phys.Rev.D40.54}

\begin{align}
I_n^{(k,l)} &=: I_{abABC}^{(k,l)}(n)\\
 &= \int\limits_0^\pi\!d\theta_1\,\sin^{n-3}(\theta_1)\int\limits_0^\pi\!d\theta_2\,\sin^{n-4}(\theta_2)(a+b\cos(\theta_1))^{-k}(A+B\cos(\theta_1)+C\sin(\theta_1)\cos(\theta_2))^{-l}\\
%I_{ab}^{(k)}(n) &:= \int\limits_0^\pi\!d\theta_1\,\sin^{n-3}(\theta_1)(a+b\cos(\theta_1))^{-k}\\
I_{b}^{(q)}(\nu) &:= \int\limits_0^\pi\!dt\,\sin^{\nu-3}(t)\cos^{q}(t)
\end{align}
It is:\fxerror{Quote DLMF 5.12.6}
\begin{equation}
\int_{0}^{\pi}(\mathop{\sin\/}\nolimits t)^{\alpha-1}e^{i\beta t}dt=\frac{\pi}{2^{\alpha-1}} \frac{e^{i\pi \beta/2}}{\alpha\mathop{\mathrm{B}\/}\nolimits\!\left((\alpha+\beta+1)/2,(\alpha-\beta+1)/2\right)} \qquad\text{if}\,\Re(\alpha) > 0
\end{equation}
\begin{align}
\Rightarrow I_{b}^{(0)}(n) &= \frac{\pi}{2^{n-3}(n-2)}\frac 1 {\mathrm{B}((n-1)/2,(n-1)/2)}\\
\Rightarrow I_{b}^{(0)}(n-1) &= \frac{\pi}{2^{n-4}(n-3)}\frac 1 {\mathrm{B}((n-2)/2,(n-2)/2)}=\mathrm{B}((n-3)/2,1/2)
\end{align}

If q is odd: $I_b^{(q)}=0$, due to symetry of kernel; if q is even: $q=2p$ with $p\in\mathbbm N$:
\begin{align}
I_{b}^{(2p)}(\nu) &= \frac 1 {2^{2p}}\sum\limits_{k=0}^{2p} \binom{2p}{k} \int\limits_0^\pi\sin^{\nu-3}(t)\exp(2i(k-p)t)\,dt\\
 &= \frac \pi{2^{2p+\nu-3} (\nu-2)}\sum\limits_{k=0}^{2p} \binom{2p}{k}\frac{\exp(i\pi(k-p))}{\mathrm{B}((\nu-1)/2+(k-p),(\nu-1)/2-(k-p))}\\
 &= \frac \pi{2^{2p+\nu-3} (\nu-2)}\sum\limits_{l=-p}^{p} \binom{2p}{p+l}\frac{(-1)^l}{\mathrm{B}((\nu-1)/2+l,(\nu-1)/2-l)}\\
 &= \frac {\pi\Gamma(\nu-1)(2p)!}{2^{2p+\nu-3} (\nu-2)\Gamma(\frac{n-1} 2+p)\Gamma(\frac{n-1} 2+p)}\left(\frac 1 {(p!)^2} \frac{\Gamma(\frac{\nu-1} 2+p)}{\Gamma(\frac{\nu-1} 2)}\frac{\Gamma(\frac{\nu-1} 2-p)}{\Gamma(\frac{\nu-1} 2)} \right. \nonumber\\
 &\hspace{40pt}\left. + 2\sum\limits_{l=1}^{p}\frac{(-1)^l}{(p+l)!(p-l)!}\frac{\Gamma(\frac{\nu-1} 2+p)}{\Gamma(\frac{\nu-1} 2+l)}\frac{\Gamma(\frac{\nu-1} 2-p)}{\Gamma(\frac{\nu-1} 2-l)}\right)\\
 &= \frac {2^{3-\nu}\pi\Gamma(\nu-1)(2p)!}{(\nu-2)\Gamma(\frac{n-1} 2+p)\Gamma(\frac{n-1} 2+p)}\cdot \frac{\Gamma(\frac{\nu-1} 2-p)}{2^{p}\Gamma(\frac{\nu-1} 2)} \cdot \frac{(2p)!}{2^p}\left(\frac 1 {(p!)^2} \frac{\Gamma(\frac{\nu-1} 2+p)}{\Gamma(\frac{\nu-1} 2)} \right. \nonumber\\
 &\hspace{40pt}\left. + 2\sum\limits_{l=1}^{p}\frac{(-1)^l}{(p+l)!(p-l)!}\frac{\Gamma(\frac{\nu-1} 2+p)}{\Gamma(\frac{\nu-1} 2+l)}\frac{\Gamma(\frac{\nu-1} 2)}{\Gamma(\frac{\nu-1} 2-l)}\right)
\end{align}
TODO: prove the brackets resolve to $\mathcal N(p)$ ... \fxerror{prove}
\begin{align}
I_{b}^{(2p)}(\nu) &=\frac {2^{3-\nu}\pi\Gamma(\nu-1)}{(\nu-2)\Gamma(\frac{n-1} 2+p)\Gamma(\frac{n-1} 2-p)} \cdot \frac{\Gamma(\frac{\nu-1} 2-p)}{2^{p}\Gamma(\frac{\nu-1} 2)} \cdot \mathcal{N}(p)\\
 &= \frac{\mathcal{N}(p)\sqrt{\pi}}{2^{p-1}} \frac {\Gamma(\nu/2)} {(\nu-2)\Gamma(\frac{\nu-1}2+p)}
\end{align}
with
\begin{equation}
\mathcal{N}(p) = \frac {(2p)!} {2^{p}(p!)^2}\frac{\Gamma(-\frac{1}2+p)}{\Gamma(-\frac{1}2)} + 2\sum\limits_{l=1}^{p}\frac{(-1)^l(2p)!}{2^p(p+l)!(p-l)!}\frac{\Gamma(-\frac{1} 2+p)}{\Gamma(-\frac{1} 2+l)}\frac{\Gamma(-\frac{1} 2)}{\Gamma(-\frac{1} 2-l)}
\end{equation}

If $-k,-l\in\mathbbm N_0$ $I_{abABC}^{(k,l)}(n)$ can always be reduced to combination of $I_b^{(q)}(n)$ and this way one for example finds: \fxerror{quote Phys.Rev.D40.54}
\begin{align}
I^{(0,0)}_{abABC}(4) &= I_b^{(0)}(3) \cdot I_b^{(0)}(4) = 2\pi \\
I^{(-1,0)}_{abABC}(4) &= I_b^{(0)}(3) \cdot (aI_b^{(0)}(4)+bI_b^{(1)}(4)) = 2\pi a \\
I^{(0,-1)}_{abABC}(4) &= I_b^{(0)}(3) \cdot (AI_b^{(0)}(4) + BI_b^{(1)}(4)) + CI_b^{(1)}(3)I_b^{(0)}(4) = 2\pi A\\
I^{(-2,0)}_{abABC}(4) &= I_b^{(0)}(3) \cdot (a^2I_b^{(0)}(4)+2abI_b^{(1)}(4) + b^2 I_b^{(2)}(4)) = 2\pi(a^2 + b^2/3) \\
I^{(0,-2)}_{abABC}(4) &= I_b^{(0)}(3) \cdot (A^2I_b^{(0)}(4) + B^2I_b^{(2)}(4)) + C^2I_b^{(2)}(3)I_b^{(0)}(6) \\
 &= 2\pi(A^2+(B^2+C^2)/3) \\
I^{(-1,-1)}_{abABC}(4) &= I_b^{(0)}(3) \cdot (a AI_b^{(0)}(4) + b B I_b^{(2)}(4)) = 2\pi(aA + bB/3) 
\end{align}

\end{document}
