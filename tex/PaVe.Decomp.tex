\subsection{Definitions}
\cite{Passarino:1978jh}:
\begin{align}
A(m) &= \frac 1 {i\pi^2}\int d^nq\frac 1 {q^2+m^2}\\
B_0(p,m_1,m_2) &= \frac 1 {i\pi^2}\int d^nq\frac 1 {(q^2+m_1^2)((q+p)^2+m_2^2)}
\end{align}
and apart from their pole term (called $\Delta$ - see \cite[eq. D.1]{Passarino:1978jh}), they keep $n=4$.

\cite{PhysRevD4054,Bojak:2000eu}:
\begin{align}
A(m) &= \mu^{-\epsilon}\int\frac{d^nq}{(2\pi)^n} \frac 1 {q^2-m^2}\\
B(q_1,m_1,m_2) &= \mu^{-\epsilon}\int\frac{d^nq}{(2\pi)^n} \frac 1 {(q^2-m_1^2)((q+q_1)^2-m_2^2)}
\end{align}
and $n=4+\epsilon$. (\cite{PhysRevD4054} writes \textquote{The notations for the one-, two-, three-, and four-point functions have been taken over from Ref. \cite{Passarino:1978jh}.} - obviously they do not.)

\HEPMath\cite{wiebusch_hepmath_2015} and \FeynCalc\cite{Mertig:1990an,Shtabovenko:2016sxi} refer to \LoopTools\cite{Hahn:1998yk,LoopTools212Guide}. \cite[eq. (1.1)]{LoopTools212Guide} and \cite[eq. (2.6)]{Ellis:2011cr}:
\begin{align}
T_{\mu_1\ldots\mu_P}^N &=
\frac{\mu^{4 - D}}{i\pi^{D/2}\,r_\Gamma}
%\frac{(2\pi\mu)^{4 - D}}{\i\pi^2}
\int d^Dq\,
\frac{q_{\mu_1}\cdots q_{\mu_P}}
  {\bigl[q^2 - m_1^2\bigr]\,
   \bigl[(q + k_1)^2 - m_2^2\bigr] \cdots
   \bigl[(q + k_{N - 1})^2 - m_N^2\bigr]} \\[1ex]
\notag
r_\Gamma &= \frac{\Gamma^2(1 - \varepsilon)\Gamma(1+\varepsilon)}
  {\Gamma(1 - 2\varepsilon)}\,,
\quad D = 4 - 2\varepsilon\,
\end{align}
later in the code they use a different signature (to avoid any vector structure):
\begin{align}
A(m^2), B_0(p^2,m_1^2,m_2^2), C_0(p_1^2,p_2^2,(p_1+p_2)^2,m_1^2,m_2^2,m_3^2)\nonumber\\
D_0(p_1^2,p_2^2,p_3^2,p_4^2,(p_1+p_2)^2,(p_2+p_3)^2,m_1^2,m_2^2,m_3^2,m_4^2)
\end{align}

\cite{Denner:2005nn}:
\begin{align}
T_{\mu_1\ldots\mu_P}^N(p_1,\ldots,p_{N-1},m_0,\ldots,m_{N-1}) &= \frac{(2\pi\mu)^{4-D}}{i\pi^2}\int d^Dq \frac{q_{\mu_1}\cdots q_{\mu_P}}{L_0 L_1 \cdots L_{N-1}}\\
L_0 &= q^2-m_0^2 +i\varepsilon\\
L_i &= (q+p_i)^2-m_i^2+i\varepsilon \, i=1,\ldots,N-1
\end{align}

I will stick to the integrals of \cite{Bojak:2000eu} as it is the most natural form, I think, and to the non-vector signature, if possible.

\subsection{Decomposition Labeling}
\cite{Passarino:1978jh,Bojak:2000eu}:
\begin{align}
B_\mu(p,m_1,m_2) &=p_{\mu} B_1(p,m_1,m_2)\\
B_{\mu\nu} &= p_{\mu}p_{\nu} B_{21}+g_{\mu\nu}B_{22}\\
C_{\mu}(p_1,p_2,m_1,m_2,m_3) &= p_{1,\mu}C_{11}+p_{2,\mu}C_{12}\\
C_{\mu\nu} &= p_{1,\mu}p_{1,\nu}C_{21}+p_{2,\mu}p_{2,\nu}C_{22}+(p_{1,\mu}p_{2,\nu}+p_{1,\nu}p_{2,\mu})C_{23}+g_{\mu\nu}C_{24}
\end{align}
The arguments of the functions are always inherited.

\HEPMath, \FeynCalc, \LoopTools, \cite{Ellis:2011cr}:
\begin{align}
B_\mu(p,m_1,m_2) &=p_{\mu} B_1(p,m_1,m_2)\\
B_{\mu\nu} &= g_{\mu\nu}B_{00}+p_{\mu}p_{\nu} B_{11}\\
C_{\mu}(p_1,p_2,m_1,m_2,m_3) &= p_{1,\mu}C_{1}+p_{2,\mu}C_{2}=\sum_{j=1}^2 p_{j,\mu}C_{j}\\
C_{\mu\nu} &= p_{1,\mu}p_{1,\nu}C_{11}+p_{2,\mu}p_{2,\nu}C_{22}+(p_{1,\mu}p_{2,\nu}+p_{1,\nu}p_{2,\mu})C_{12}+g_{\mu\nu}C_{00}\\
 &=g_{\mu\nu}C_{00} + \sum_{j,k=1}^2 p_{j,\mu}p_{k,\nu}C_{jk}
\end{align}
The arguments of the functions are always inherited.

I will stick to \HEPMath{} as it is the more generic and extensible form, I think.

\subsection{B Decomposition}
define
\begin{align}
f_1 = m_1^2-m_0^2-p^2
\end{align}
then one finds easily
\begin{align}
B_1(p^2,m_0^2,m_1^2) &= \frac 1 {2p^2}\left(f_1B_0(p^2,m_0^2,m_1^2)+A_0(m_0^2)-A_0(m_1^2)\right) \label{eq:B1}\\
B_{00}(p^2,m_0^2,m_1^2) &= \frac 1 {2(n-1)}\left(2m_0^2B_0(p^2,m_0^2,m_1^2)+A_0(m_1^2)-f_1B_1(p^2,m_0^2,m_1^2)\right)\\
B_{11}(p^2,m_0^2,m_1^2) &= \frac 1 {2p^2}\left(f_1B_0(p^2,m_0^2,m_1^2)+A_0(m_1^2)-2B_{00}(p^2,m_0^2,m_1^2)\right)
\end{align}
in accordance with \cite{Bojak:2000eu,Ellis:2011cr}.

Concering $B_1$ \cite{Passarino:1978jh} and \LoopTools{} use the following identity
\begin{equation}
A_0(m_0^2)-A_0(m_1^2)=(m_0^2-m_1^2)B_0(0,m_0^2,m_1^2)
\end{equation}
that might help away with
\begin{displayquote}[{\cite[below eq. D.6]{Passarino:1978jh}}]
In case $m_1$ and/or $m_2$ are very large the expression on the right-hand side of eq. (\ref{eq:B1}) suffers very strong cancellations: the total is very much smaller than the individual terms. For this reason we have not used these algebraic relations, except to rewrite self-energy diagrams as much as possible in a form most suitable for numerical evaluation.
\end{displayquote}

To compare the other results to \cite{Passarino:1978jh} and \LoopTools{} one has to use the \textit{strict} $n\rightarrow 4$ limit and the following identities\cite{Denner:2005nn}:
\begin{align}
(n-4) B_{00}(p^2,m_0^2,m_1^2) &= \frac 1 6 (p^2-3m_0^2-3m_1^2)\\
(n-4) B_{11}(p^2,m_0^2,m_1^2) &= -\frac 2 3
\end{align}
