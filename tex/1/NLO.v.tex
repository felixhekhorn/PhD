Virtual contributions have the same initial and final state as the Born process, but have a looping particle. All contributing Feynman diagrams are depicted in figure \fxerror{do}.
The result can then be written as
\begin{equation}
\hat\sum_{k,\sigma}\mathcal P_{k}^{\mu\mu'}\sum_{j}\left[\Md^{(1),v}_{j,\mu}\left(\Md^{(0)}_{1,\mu'}+\Md^{(0)}_{2,\mu'}\right)^*+c.c.\right] = 8g^4e^2e_H^2 N_C C_F\left( C_A V_{k,OK} + 2C_F V_{k,QED}\right)
\end{equation}
where $C_A$ is the second Casimir constant of the adjoint representation for the gluon.

For the computation of the loops the Passarino-Veltman-decomposition\cite{Passarino:1978jh} is used as far as possible. The decomposition is based on Lorentz invariance and a good explanation is for example given in \cite{Bojak:2000eu}. The needed scalar integrals are given in \cite{PhysRevD4054} and \cite{Laenen1993162}, but there is also one wrong integral: we find with \cite[Box 16]{Ellis:2007qk}:
\begin{align}
&D_0(m^2,0,q^2,m^2,t,s,0,m^2,m^2,m^2)\nonumber\\
 &=\frac{i C_\epsilon}{\beta s t_1}\left[-\frac{2\ln(\chi)}{\epsilon} -2\ln(\chi)\ln(\tilde t)+\DiLog(1-\chi^2)-4\zeta(2)+\ln^2(\chi_q) + 2\DiLog(-\chi\chi_q)\right.\nonumber\\
 &\hspace{40pt}\left. +2\DiLog(-\chi/\chi_q)+2\ln(\chi\chi_q)\ln(1+\chi\chi_q)+2\ln(\chi/\chi_q)\ln(1+\chi/\chi_q)\right]
\end{align}
where we used the argument ordering of \LoopTools\cite{Hahn:1998yk,LoopTools212Guide} (and also checked it against \LoopTools).
