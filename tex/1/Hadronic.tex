The hadronic reaction to study is deep-inelastic lepton-proton scattering:
\begin{equation}
\Pleptonminus(l_1) + \Pp(p) \rightarrow \Pleptonminus(l_2) + \PQ(p_1)(\PaQ(p_2)) + X
\end{equation}
where one either detects the heavy quark $\PQ$ \textit{or} the heavy antiquark $\PaQ$ and $X$ stands for any final hadronic state allowed by quantum-number conservation. We define then the hadronic Bjorken variables
\begin{equation}
q=l_1-l_2, \quad Q^2=-q^2, \quad x=\frac{Q^2}{2p\cdot q}, \quad z = \frac{p\cdot q}{p\cdot l_1}
\end{equation}

We can then define the measurable deep-inelastic hadron structure functions
\begin{align}
F_k(x,Q^2,m^2) &= \sum_{n=0}^\infty F_{k}^{(n)}(x,Q^2,m^2)\\
F_{k}^{(n)}(x,Q^2,m^2) &= \frac{Q^2}{4\pi^2\alpha} \sum_{j\in\{\Pg,\Pq,\Paq\}}\int\limits_x^{z_{max}} \frac{dz}{z} f_j(x/z,\mu_F^2) \sigma^{(n)}_{k,j}(s,q^2,m^2)
\end{align}
where $k\in\{G,L,P\}$ denotes as usual projection, $z=Q^2/s'$, $z_{max} = Q^2/(4m^2+Q^2)$ and $f_j(x/z,\mu_F^2)$ denotes parton momentum density functions\cite{Martin:2009iq,PhysRevLett.113.012001}.
We can then split the contributions whether there is a gluon in the initial state $F_{k,g}$ or a (anti)quark $F_{k,q}$. In leading order we find:
\begin{align}
F_{k,g}^{(0)}(x,Q^2,m^2) &= \frac{\alpha_sQ^2}{4\pi^2m^2}e_H^2\int\limits_x^{z_{max}}\frac{dz}{z} f_g(x/z,\mu_F^2) c^{(0)}_{k,g}(\eta,\xi)
\end{align}
We find for the gluonic part in next-to-leading order:
\begin{align}
&F_{k,g}^{(1)}(x,Q^2,m^2) \nonumber\\
 &= \frac{\alpha_s^2Q^2}{\pi m^2}e_H^2\int\limits_x^{z_{max}}\frac{dz}{z} f_g(x/z,\mu_F^2) \left(c_{k,g}^{(1)}(\eta,\xi) + \ln(\mu_F^2/m^2)\bar c_{k,g}^{F,(1)}(\eta,\xi) + \ln(\mu_R^2/m^2)\bar c_{k,g}^{R,(1)}(\eta,\xi)\right)
\end{align}
We find for the quark part in next-to-leading order:
\begin{align}
&F_{k,q}^{(1)}(x,Q^2,m^2) \nonumber\\
 &= \frac{\alpha_s^2Q^2}{\pi m^2}e_H^2\int\limits_x^{z_{max}}\frac{dz}{z} \left(\sum_{j=1}^{n_{lf}}f_{q(j)}(x/z,\mu_F^2)+f_{q(-j)}(x/z,\mu_F^2)\right) \nonumber\\
 &\hspace{120pt} \cdot\left(c_{k,q}^{(1)}(\eta,\xi) + \ln(\mu_F^2/m^2)\bar c_{k,q}^{F,(1)}(\eta,\xi)\right) \nonumber\\
 &\hspace{20pt} + \frac{\alpha_s^2Q^2}{\pi m^2}\int\limits_x^{z_{max}}\frac{dz}{z} \left(\sum_{j=1}^{n_{lf}}e_{q(j)}^2\left(f_{q(j)}(x/z,\mu_F^2)+f_{q(-j)}(x/z,\mu_F^2)\right)\right) d_{k,q}(\eta,\xi)
\end{align}
where we used the PDG particle labeling\cite{Hagiwara:2002fs}: $q(1)=u,q(-1)=\bar u,q(2)=d,q(-2)=\bar d$, \ldots\,and $e_u = e_c = e_t = 2/3, e_d=e_s=e_b=-1/3$. 

We can then also define some more practical functions:
\begin{align}
F_{2}(x,Q^2,m^2) &= F_{T}(x,Q^2,m^2) + F_{L}(x,Q^2,m^2)\\
 &= F_{G}(x,Q^2,m^2) + \frac 3 2 F_{L}(x,Q^2,m^2)\\
F_1(x,Q^2,m^2) &= (F_{2}(x,Q^2,m^2)-F_L(x,Q^2,m^2))/(2x)\\
 &= \left(F_{G}(x,Q^2,m^2)+\frac 1 2 F_{L}(x,Q^2,m^2)\right)/(2x)\\
g_1(x,Q^2,m^2) &= F_{P}(x,Q^2,m^2)/(2x)
\end{align}
and we define
\begin{align}
R_{k'}(x,Q^2,m^2) &= \frac{F_{k'}^{(0)}(x,Q^2,m^2)+F_{k'}^{(1)}(x,Q^2,m^2)}{F_{k'}^{(0)}(x,Q^2,m^2)}
\end{align}
with $k'\in\{2,L,P\}$ to better observe next-to-leading order effects.

We define the spin asymmetry by
\begin{align}
A_1(x,Q^2,m^2) &= \frac{g_1(x,Q^2,m^2)}{F_1(x,Q^2,m^2)} = \frac{F_P(x,Q^2,m^2)}{F_2(x,Q^2,m^2)-F_L(x,Q^2,m^2)}
\end{align}

For the plots we focused on charm production ($n_{lf}=3$) with $m_c=\SI{1.5}{\GeV}$ and we used the two-loop running coupling of \cite{Altarelli:1988qr}:
\begin{align}
\alpha_s(\mu_R^2) &= \frac{1}{\beta_{0}^4\ln(\mu_R^2/\Lambda_4)}\left(1-\frac{\beta_1^4}{(\beta_0^4)^2} \ln(\ln(\mu_R^2/\Lambda_4))\right)
\end{align}
with the first two coefficients of the QCD beta function $\beta_0^f=(33-2f)/(12\pi)$ and $\beta_1^f=(306-38f)/(48\pi^2)$\fxerror{find CA?} and $\Lambda_4=\SI{0.194}{\GeV^2}$. We set $\mu_F^2=\mu_R^2=4m^2-q^2$ in analogy to \cite{Laenen1993162}. We used the PDF set MSTW2008nlo90cl\cite{Martin:2009iq,Martin:2009bu,Martin:2010db} provided by LHAPDF\cite{LHAPDF6} for the unpolarized structure functions ($F_2,F_1,F_G,F_L$) and DSSV2014\cite{PhysRevLett.113.012001} for the polarized structure function ($g_1,F_G$).

\fxerror{shift images to appendix?}

\begin{figure}[ht!]
\centering
\input{img/A}
\caption{spin asymmetry $A_{1}(x,Q^2,m_c^2)$ plotted as function of $x$ for different values of $Q^2$ in units of $\si{\GeV^2}$}\label{fig:A}
\end{figure}

\pagebreak
\begin{figure}[ht!]
\centering
\begin{subfigure}[t]{\textwidth}
	\input{img/F0012}
\end{subfigure}\\%
\begin{subfigure}[t]{\textwidth}
	\input{img/F001L}
\end{subfigure}\\%
\begin{subfigure}[t]{\textwidth}
	\input{img/F001P}
\end{subfigure}
\caption{hadronic structure functions $F_{k}(x,Q^2,m_c^2)$ plotted as function of $x$ for different values of $Q^2$ in units of $\si{\GeV^2}$}\label{fig:F001}
\end{figure}

\pagebreak
\begin{figure}[ht!]
\centering
\begin{subfigure}[t]{\textwidth}
	\input{img/Fg12}
\end{subfigure}\\%
\begin{subfigure}[t]{\textwidth}
	\input{img/Fg1L}
\end{subfigure}\\%
\begin{subfigure}[t]{\textwidth}
	\input{img/Fg1P}
\end{subfigure}
\caption{next-to-leading order hadronic structure functions $F_{k,g}^{(1)}(x,Q^2,m_c^2)$ plotted as function of $x$ for different values of $Q^2$ in units of $\si{\GeV^2}$}\label{fig:Fg1}
\end{figure}

\pagebreak
\begin{figure}[ht!]
\centering
\begin{subfigure}[t]{\textwidth}
	\input{img/Fq12}
\end{subfigure}\\%
\begin{subfigure}[t]{\textwidth}
	\input{img/Fq1L}
\end{subfigure}\\%
\begin{subfigure}[t]{\textwidth}
	\input{img/Fq1P}
\end{subfigure}
\caption{next-to-leading order hadronic structure functions $F_{k,q}^{(1)}(x,Q^2,m_c^2)$ plotted as function of $x$ for different values of $Q^2$ in units of $\si{\GeV^2}$}\label{fig:Fq1}
\end{figure}

\pagebreak
\begin{figure}[ht!]
\centering
\begin{subfigure}[t]{\textwidth}
	\input{img/R2}
\end{subfigure}\\%
\begin{subfigure}[t]{\textwidth}
	\input{img/RL}
\end{subfigure}\\%
\begin{subfigure}[t]{\textwidth}
	\input{img/RP}
\end{subfigure}
\caption{ratio of hadronic structure functions $R_{k'}^{(1)}(x,Q^2,m_c^2)$ plotted as function of $x$ for different values of $Q^2$ in units of $\si{\GeV^2}$}\label{fig:R}
\end{figure}

\clearpage
\pagebreak
