To collect all soft and collinear poles we have to calculate in $n=4+\epsilon$ dimension. Unfortunaly the extension for \textit{polarized} processes is nontrivial, because the occuring Levi-Civita tensors $\varepsilon_{\mu\nu\rho\sigma}$ and $\gamma_5$. A common choice to deal with these objects is the HVBM prescription\cite{breitenlohner1977} that keeps those two objects four dimensional at the price for splitting the full $n$-dimensional space into a $(n-4)$-dimensional space, called \enquote{hat-space}, and a four-dimensional space (that is actually never used).

In leading order (LO) we have to consider the following processes
\begin{equation}
\Pggx(q;\sigma_q) + \Pg(k_1;\sigma_{k_1}) \rightarrow \PQ(p_1)+\PaQ(p_2)
\end{equation}
The corresponding parton structure tensor $W_{\mu\mu'}^{(0)}$ can then be written as
\fxerror{avoid all order expr?}
\begin{align}
&W_{\mu\mu'}^{(0)}(k_1,q;s,t_1,u_1,q^2;\sigma_{k_1}\sigma_{q})\nonumber\\
 &= \frac 1 2 E_\epsilon K_{\Pg\Pgg}\!\int\!\frac{d^{n-1}p_1}{2E_1(2\pi)^{n-1}}\!\int\!\frac{d^{n-1}p_2}{2E_2(2\pi)^{n-1}}\delta(p_1^2-m^2) \delta(p_2^2-m^2)\nonumber\\
 &\hspace{20pt}(2\pi)^n\delta^{(n)}(k_1+q-p_1-p_2) \Md_{\mu}^{(0)}(\sigma_{k_1},\sigma_q)\Md_{\mu'}^{(0)}(\sigma_{k_1},\sigma_q)
\label{eq:PartonicStructureTensor0}\end{align}
where the initial $1/2$ is the initial state spin average, $K_{\Pg\Pgg}$ is the color average,
\begin{align}
E_\epsilon := \left\{\begin{array}{ll}
1/(1+\epsilon/2) &\text{unpolarized}\\
1 &\text{polarized}
\end{array}\right.
\end{align}
accounts for initial freedom in $n$ dimensions for bosons and we defined the following Mandelstam variables:
\begin{equation}
s = (q+k_1)^2, \quad t_1=t-m^2=(k_1-p_2)^2-m^2, \quad u_1 = u - m^2 = (q-p_2)^2 -m^2
\end{equation}
\begin{equation}
s' = s-q^2,\quad u_1' = u_1 - q^2
\end{equation}
\fxerror{move to LO?}
The Lorentz indices $\mu$ and $\mu'$ refer to the virtual photon that is exchanged with the scattering lepton.

By using Lorentz covariance, hermiticity, parity invariance and current conservation the parton structure tensor can be decomposed into several parts:
\begin{align}
W_{\mu\mu'}(k_1,q;s,t_1,u_1,q^2;\sigma_{k_1},\sigma_{q}) &= \left(-g_{\mu\mu'} + \frac{q_\mu q_{\mu'}}{q^2}\right)\frac{d^2\sigma_T(s,t_1,u_1,q^2)}{dt_1du_1} \nonumber\\
 &\hspace{20pt} +\left(k_{1,\mu}-\frac{k_1\cdot q}{q^2}q_\mu\right)\left(k_{1,\mu'}-\frac{k_1\cdot q}{q^2}q_{\mu'}\right)\left(\frac{-4q^2}{{s'}^2}\right) \nonumber\\
 &\hspace{30pt} \cdot\left(\frac{d^2\sigma_T(s,t_1,u_1,q^2)}{dt_1du_1}+\frac{d^2\sigma_L(s,t_1,u_1,q^2)}{dt_1du_1}\right)
\end{align}
\fxerror{extend}
We can then define appropiate projection operators\cite{Laenen1993162,Vogelsang:1993eg}
\begin{align}
\hat {\mathcal P}^{\Pgg}_{G,\mu\mu'} &= -g_{\mu\mu'} &b_G(\epsilon) &= \frac 1 {2(1+\epsilon/2)}\\
\hat {\mathcal P}^{\Pgg}_{L,\mu\mu'} &= -\frac{4q^2}{{s'}^2} k_{1,\mu}k_{1,\mu'} &b_L(\epsilon) &= 1\\
\hat {\mathcal P}^{\Pgg}_{P,\mu\mu'} &= i\varepsilon_{\mu\mu'\rho\rho'}\frac{q^{\rho}k_1^{\rho'}}{s'} &b_P(\epsilon) &= 1
\end{align}
\fxerror{justify avoidance of $\Delta$?} and we then find
\begin{align}
\frac{d^2\sigma_{k}(s,t_1,u_1,q^2)}{dt_1tu_1} &= b_k(\epsilon)\hat {\mathcal P}^{\Pgg}_{k,\mu\mu'}W^{\mu\mu'}
\end{align}
with $k\in\{G,L,P\}$ denoting (here and mostly ever after) the projection type. The transverse partonic cross section $d\sigma_T$ can be reconstructed from the above definitions by using
\begin{align}
d\sigma_T &= d\sigma_G + b_G(\epsilon)d\sigma_L
\end{align}
We also define accordingly
\begin{equation}
E_G(\epsilon) = E_L(\epsilon) = \frac 1 {1+\epsilon/2} \qquad E_P(\epsilon) = 1
\end{equation}

The final state spins are always summed over, but the initial spins have to be treated seperately: for unpolarized projections $k\in\{G,L\}$ they are also summed over, but for the polarized projection $k=P$ they are projected on their asymmetric part:
\begin{align}
\hat {\mathcal P}^{\Pg}_{G}\varepsilon^{(\sigma_k)}_\nu(k_1)\varepsilon^{*(\sigma_k)}_{\nu'}(k_1) = \hat {\mathcal P}^{\Pg}_{L}\varepsilon^{(\sigma_k)}_\nu(k_1)\varepsilon^{*(\sigma_k)}_{\nu'}(k_1) &= -g_{\nu\nu'} \label{eq:PgG} \\
\hat {\mathcal P}^{\Pg}_{P}\varepsilon^{(\sigma_k)}_\nu(k_1)\varepsilon^{*(\sigma_k)}_{\nu'}(k_1) &= 2i\epsilon_{\nu\nu'\rho\rho'}\frac{k_1^{\rho}q^{\rho}}{s'}
\end{align}
By writing eq. (\ref{eq:PgG}) we decided to introduce Fadeev-Popov ghosts\cite{Bojak:2000eu} as we got a single diagram in next-to-leading order with a triple-gluon vertex\fxerror{explain ghosts?}. As we can consider all quarks in the initial state as massless, we further find
\begin{align}
\hat {\mathcal P}^{\Pq}_{G} u(k_1)\bar u(k_1) = \hat {\mathcal P}^{\Pq}_{L} u(k_1)\bar u(k_1) &= \slashed k_1
&\hat {\mathcal P}^{\Pq}_{P} u(k_1)\bar u(k_1) &= -\gamma_5\slashed k_1\\
\hat {\mathcal P}^{\Paq}_{G} v(k_1)\bar v(k_1) = \hat {\mathcal P}^{\Paq}_{L} v(k_1)\bar v(k_1) &= \slashed k_1
&\hat {\mathcal P}^{\Paq}_{P} v(k_1)\bar v(k_1) &= \gamma_5\slashed k_1
\end{align}

We define a set of partonic variables:
\begin{align}
0\leq&\,\rho = \frac {4m^2} s\leq 1 &0\leq&\,\beta = \sqrt{1-\rho}\leq 1 &0\leq&\,\chi = \frac{1-\beta}{1+\beta}\leq 1\\
&\rho_q = \frac {4m^2} {q^2}\leq 0 &1\leq&\,\beta_q = \sqrt{1-\rho_q} &0\leq&\,\chi_q = -\frac{1-\beta_q}{1+\beta_q}\leq 1
\end{align}

When computing Feynman diagrams a computer algebra system (CAS) is almost obligatory: common choices are \texttt{FORM}\cite{Vermaseren:2000nd} or \MMa\cite{Mathematica} - for the later the most common choice is \texttt{TRACER}\cite{Tracer}, but we have chosen \HEPMath\cite{wiebusch_hepmath_2015}. We used the Feynman rules given by \cite{Leader}.
