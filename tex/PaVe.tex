% @author: Felix Hekhorn <felix.hekhorn@student.uni-tuebingen.de>
\documentclass[
  english,		% Sprache
  a4paper,		% Papierformat
  11pt,			% Schriftgröße (default 10pt)
  DIV=12,		% Seiteneinteilung
  titlepage,
  toc=bibnumbered,
  parskip=full,  	% Absätze (full,half,false -+*)
  headings=normal,
  BCOR=12mm,
  numbers=noenddot
]{scrartcl}
%\documentclass[a4paper,10pt]{article}
\usepackage{scrtime,scrlfile,scrpage2}

\usepackage[status=draft]{fixme}
%\usepackage[status=final]{fixme}

\usepackage[utf8]{inputenc}
%\usepackage[ngerman]{babel} % Sprache 
\usepackage[ngerman,english,main=english]{babel} % Sprache 
\selectlanguage{english}
\usepackage{amsmath, amssymb}

%\usepackage{graphicx} % Grafiken einbinden
% The following is needed in order to make the code compatible
% with both latex/dvips and pdflatex.
\ifx\pdftexversion\undefined
\usepackage[dvips]{graphicx}
\else
\usepackage[pdftex]{graphicx}
\DeclareGraphicsRule{*}{mps}{*}{}
\fi
\usepackage{pdfpages}

%\usepackage[left=2cm,right=2cm,top=2.5cm,bottom=3cm]{geometry}
\usepackage{color}
\usepackage{bbm}
\usepackage{csquotes}

\usepackage{pdflscape}
\usepackage{ulem}
\usepackage{url}
\usepackage{caption}
\usepackage{subcaption}
\usepackage{array}
\usepackage{multirow}
\usepackage{listings}
\usepackage{placeins}

\usepackage{siunitx} % SI Einheiten
\usepackage[version=3]{mhchem}
%\sisetup{
%	exponent-product = \!\cdot\!,
%	output-product = \cdot,
%	list-final-separator =  { und } ,
%	list-pair-separator = { und } ,
%	range-phrase = { bis },
%	output-decimal-marker = {,},
%	separate-uncertainty = true,
%	group-digits = false
%}

%\usepackage{simplewick}
%\usepackage{feynmf}
\usepackage{slashed}
\usepackage{hepnames}

%\usepackage{biblatex}
\usepackage[numbers]{natbib}
%\bibliographystyle{natdin}
%\bibliographystyle{kp}
\bibliographystyle{utphys}

\usepackage{hyperref}
%\hypersetup{colorlinks=false}
\usepackage{tabularx}

\providecommand{\abs}[1]{\left|#1\right|}
\providecommand{\VektorV}[3]{
\!\left(\!\!
\begin{array}{c}
#1 \\ #2 \\ #3
\end{array}\!\!
\right)\!
}

\providecommand{\Det}[9]{
	\begin{vmatrix}
	    #1 & #2 & #3 \\
	    #4 & #5 & #6 \\
	    #7 & #8 & #9 \\	    
	\end{vmatrix}
}

\DeclareMathOperator{\Grad}{\text{grad}}
\DeclareMathOperator{\Div}{\text{div}}
\DeclareMathOperator{\Rot}{\text{rot}}
\DeclareMathOperator{\tr}{\text{tr}}

\DeclareMathOperator{\acos}{\text{arccos}}
\DeclareMathOperator{\asin}{\text{arcsin}}
\DeclareMathOperator{\atanh}{\text{artanh}}
\DeclareMathOperator{\DiLog}{\text{Li}_2}
\DeclareMathOperator{\x}{\times}
\DeclareMathOperator{\cdt}{\!\cdot\!}
\DeclareMathOperator{\del}{\partial}
\DeclareMathOperator{\EqualClaim}{\stackrel{!}{=}}
\DeclareMathOperator{\equivals}{\mathrel{\widehat{=}}}
\providecommand{\Nabla}[0]{\vec\nabla}
\providecommand{\ex}[1]{e^{#1}}
\providecommand{\EE}[1]{\cdot 10^{#1}}
\providecommand{\FT}[1]{\mathcal{FT}\left[#1\right]}
\providecommand{\Mel}[1]{\mathcal{M}\left[#1\right]}
\providecommand{\invMel}[1]{\mathcal{M}^{-1}\left[#1\right]}

\providecommand{\dt}[0]{\Derive t}
\providecommand{\dx}[0]{\Derive x}
\providecommand{\Derive}[1]{\DeriveN{#1}{}}
\providecommand{\DeriveN}[2]{\DeriveNF {#1}{#2}{}}
\providecommand{\DeriveF}[2]{\DeriveNF {#1}{}{#2}}
\providecommand{\DeriveNF}[3]{\frac {d^{#2}#3} {d #1^{#2}}}
\providecommand{\dtP}[0]{\DeriveP t}
\providecommand{\dxP}[0]{\DeriveP x}
\providecommand{\DeriveP}[1]{\DerivePN{#1}{}}
\providecommand{\DerivePF}[2]{\DerivePNF {#1} {} {#2}}
\providecommand{\DerivePN}[2]{\DerivePNF {#1} {#2} {} }
\providecommand{\DerivePNF}[3]{\frac {\partial^{#2}#3} {\partial #1^{#2}}}
\providecommand{\DerivePMF}[3]{\frac {\partial^{2}#3} {\partial #1 \partial #2}}
\providecommand{\e}[1]{\hat{e}_{#1}}
\providecommand{\pFq}[2]{{}_{#1}F_{#2}}

\providecommand{\bra}[1]{\langle#1\rvert}
\providecommand{\ket}[1]{\lvert#1\rangle}
\providecommand{\bracket}[2]{\langle#1\vert#2\rangle}
\providecommand{\normOrd}[1]{\,:\!#1\!:\,}
\providecommand{\wContr}[3]{\contraction{}{#1}{#2}{#3}#1#2#3}

\DeclareMathOperator{\Md}{\mathcal M}
\DeclareMathOperator{\Ld}{\mathcal L}
\DeclareMathOperator{\Hd}{\mathcal H}
\DeclareMathOperator{\Nd}{\hat {\mathcal N}}
\DeclareMathOperator{\To}{\hat {\mathcal T}}

\DeclareMathOperator{\MSbar}{\overline{\text{MS}}}

\DeclareRobustCommand{\PQ}{\HepGenParticle{Q}{}{}\xspace} % heavy quark
\DeclareRobustCommand{\PaQ}{\HepGenAntiParticle{Q}{}{}\xspace} % heavy anti-quark

\def\MMa{{\texttt{Mathematica}}}
\def\HEPMath{\texttt{HEPMath}}
\def\FeynCalc{\texttt{FeynCalc}}
\def\LoopTools{\texttt{LoopTools}}
\def\QCDLoop{\texttt{QCDLoop}}


\begin{document}

\section{Passarino-Veltman decomposition}
to cut a long story short: there are lots of different settings und conventions and one has really to be carefull

\subsection{definitions}
\cite{Passarino:1978jh}:
\begin{align}
A(m) &= \frac 1 {i\pi^2}\int d^nq\frac 1 {q^2+m^2}\\
B_0(p,m_1,m_2) &= \frac 1 {i\pi^2}\int d^nq\frac 1 {(q^2+m_1^2)((q+p)^2+m_2^2)}
\end{align}
and apart from their pole term (called $\Delta$ - see \cite[eq. D.1]{Passarino:1978jh}), they keep $n=4$.

\cite{PhysRevD4054,Bojak:2000eu}:
\begin{align}
A(m) &= \mu^{-\epsilon}\int\frac{d^nq}{(2\pi)^n} \frac 1 {q^2-m^2}\\
B(q_1,m_1,m_2) &= \mu^{-\epsilon}\int\frac{d^nq}{(2\pi)^n} \frac 1 {(q^2-m_1^2)((q+q_1)^2-m_2^2)}\\
\end{align}
and $n=4+\epsilon$. (\cite{PhysRevD4054} writes \textquote{The notations for the one-, two-, three-, and four-point functions have been taken over from Ref. \cite{Passarino:1978jh}.} - obviously they do not.)

\HEPMath\cite{wiebusch_hepmath_2015} and \FeynCalc\cite{Mertig:1990an,Shtabovenko:2016sxi} refer to \LoopTools\cite{Hahn:1998yk,LoopTools212Guide}. \cite[eq. (1.1)]{LoopTools212Guide} and \cite[eq. (2.6)]{Ellis:2011cr}:
\begin{align}
T_{\mu_1\ldots\mu_P}^N &=
\frac{\mu^{4 - D}}{i\pi^{D/2}\,r_\Gamma}
%\frac{(2\pi\mu)^{4 - D}}{\i\pi^2}
\int d^Dq\,
\frac{q_{\mu_1}\cdots q_{\mu_P}}
  {\bigl[q^2 - m_1^2\bigr]\,
   \bigl[(q + k_1)^2 - m_2^2\bigr] \cdots
   \bigl[(q + k_{N - 1})^2 - m_N^2\bigr]} \\[1ex]
\notag
r_\Gamma &= \frac{\Gamma^2(1 - \varepsilon)\Gamma(1+\varepsilon)}
  {\Gamma(1 - 2\varepsilon)}\,,
\quad D = 4 - 2\varepsilon\,
\end{align}
later in the code they use a different signature (to avoid any vector structure):
\begin{equation}
A(m^2), B_0(p^2,m_1^2,m_2^2), C_0(p_1^2,p_2^2,(p_1+p_2)^2,m_1^2,m_2^2,m_3^2)
\end{equation}

I will stick to \cite{Bojak:2000eu} as it is the most natural form, I think.

\subsection{decomposition labeling}
\cite{Passarino:1978jh,Bojak:2000eu}:
\begin{align}
B_\mu(p,m_1,m_2) &=p_{\mu} B_1(p,m_1,m_2)\\
B_{\mu\nu} &= p_{\mu}p_{\nu} B_{21}+g_{\mu\nu}B_{22}\\
C_{\mu}(p_1,p_2,m_1,m_2,m_3) &= p_{1,\mu}C_{11}+p_{2,\mu}C_{12}\\
C_{\mu\nu} &= p_{1,\mu}p_{1,\nu}C_{21}+p_{2,\mu}p_{2,\nu}C_{22}+(p_{1,\mu}p_{2,\nu}+p_{1,\nu}p_{2,\mu})C_{23}+g_{\mu\nu}C_{24}
\end{align}
The arguments of the functions are always inherited.

\HEPMath, \FeynCalc, \LoopTools, \cite{Ellis:2011cr}:
\begin{align}
B_\mu(p,m_1,m_2) &=p_{\mu} B_1(p,m_1,m_2)\\
B_{\mu\nu} &= g_{\mu\nu}B_{00}+p_{\mu}p_{\nu} B_{11}\\
C_{\mu}(p_1,p_2,m_1,m_2,m_3) &= p_{1,\mu}C_{1}+p_{2,\mu}C_{2}=\sum_{j=1}^2 p_{j,\mu}C_{j}\\
C_{\mu\nu} &= p_{1,\mu}p_{1,\nu}C_{11}+p_{2,\mu}p_{2,\nu}C_{22}+(p_{1,\mu}p_{2,\nu}+p_{1,\nu}p_{2,\mu})C_{12}+g_{\mu\nu}C_{00}\\
 &=g_{\mu\nu}C_{00} + \sum_{j,k=1}^2 p_{j,\mu}p_{k,\nu}C_{jk}
\end{align}
The arguments of the functions are always inherited.

I will stick to \cite{wiebusch_hepmath_2015} as it is the more generic and extensible form, I think.

\subsection{B decomposition}
define
\begin{align}
f_1 = m_2^2-m_1^2-p^2
\end{align}
then one finds easily
\begin{align}
B_1(p^2,m_1^2,m_2^2) &= \frac 1 {2p^2}\left(f_1B_0(p^2,m_1^2,m_2^2)+A_0(m_1)-A_0(m_2)\right) \label{eq:B1}\\
B_{11}(p^2,m_1^2,m_2^2) &= \frac 1 {2p^2}\left(f_1B_0(p^2,m_1^2,m_2^2)+A_0(m_2)-2B_{00}(p^2,m_1^2,m_2^2)\right)\\
B_{00}(p^2,m_1^2,m_2^2) &= \frac 1 {2(n-1)}\left(2m_1^2B_0(p^2,m_1^2,m_2^2)+A_0(m_2)-f_1B_1(p^2,m_1^2,m_2^2)\right)
\end{align}
in accordance with \cite{Bojak:2000eu,Ellis:2011cr}.

but NOT with \cite{Passarino:1978jh} and \LoopTools:

concering $B_1$ everything is ok, if one takes the following identity
\begin{equation}
A_0(m_1)-A_0(m_2)=(m_1^2-m_2^2)B_0(0,m_1^2,m_2^2)
\end{equation}
that might help away with
\begin{displayquote}[{\cite[below eq. D.6]{Passarino:1978jh}}]
In case $m_1$ and/or $m_2$ are very large the expression on the right-hand side of eq. (\ref{eq:B1}) suffers very strong cancellations: the total is very much smaller than the individual terms. For this reason we have not used these algebraic relations, except to rewrite self-energy diagrams as much as possible in a form most suitable for numerical evaluation.
\end{displayquote}
But this introduces an other quantity $B_0(0,m_1^2,m_2^2)$, that \cite{Bojak:2000eu,Ellis:2011cr} never mentions (keep in mind their shortlabels $B_{....}(1,2)$ ONLY refers to the masses, but NOT the momenta).

concerning $B_{00}$ there is also a matching, BUT NOT concerning $B_{11}$:

\cite{Passarino:1978jh}:
\begin{equation}
A(m_2)-m_1^2B_0 = p^2B_{11}+4B_{00}+\frac 1 2\left(m_1^2+m_2^2+\frac 1 3p^2\right)
\end{equation}
\LoopTools:
\begin{equation}
A(m_2)-m_1^2B_0 - (p^2B_{11}+4B_{00}) =-2B_0(p^2,m_1^2,m_2^2)-\frac 1 2\left(m_1^2+m_2^2-\frac 1 3p^2\right)
\end{equation}

What to do? ;-) whom to trust?

\subsection{outlook}
it might be worth to take a look into \cite{Ellis:2011cr}, because - see below; what do you think?
\includepdf[pages={11}]{../ref/oneloop_ka.pdf}

\newpage
\appendix

\bibliography{ref.bib}
\listoffixmes

\end{document}
