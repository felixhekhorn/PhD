\label{sec:NLO.PS}
In next-to-leading order we have to consider processes which involve an additional particle in the final state. The matrix elements will then depend on ten kinematical invariants:
\begin{align}
s &= (q+k_1)^2 &t_1 &=(k_1-p_2)^2-m^2 &u_1 &=(q-p_2)^2 -m^2\\
s_3 &= (k_2+p_2)^2-m^2 &s_4 &=(k_2+p_1)^2-m^2 &s_5 &= (p_1+p_2)^2=-u_5\\
t' &= (k_1-k_2)^2\\
u' &= (q-k_2)^2 &u_6 &=(k_1-p_1)^2-m^2 &u_7 &=(q-p_1)^2-m^2
\end{align}
from which only five are independent as can be seen from momentum conservation $k_1+q=p_1+p_2+k_2$ and $s,t_1,u_1$ match to their leading order definition.

The $2\rightarrow 3$ $n$-dimensional phase space is given by
\begin{align}
dPS_3 &= \!\int\!\!\frac{d^{n}p_1}{(2\pi)^{n-1}}\frac{d^{n}p_2}{(2\pi)^{n-1}}\frac{d^{n}k_2}{(2\pi)^{n-1}}(2\pi)^n\delta^{(n)}(k_1+q-p_1-p_2-k_2) \nonumber\\
 &\hspace{50pt}\Theta(p_{1,0})\delta(p_1^2-m^2)\Theta(p_{2,0})\delta(p_2^2-m^2)\Theta(k_{2,0})\delta(k_2^2) \label{eq:PS3}
\end{align}
This can be solved by writing eq. (\ref{eq:PS3}) as product of a $2\rightarrow 2$ decay and a subsequent $1\rightarrow 2$ decay\cite{PhysRevD4054}. We choose the following decomposition\cite{Harris:1995tu}:
\begin{align}
q &= (q^,0,0,\abs{\vec q})\\
k_1 &= k_0(1,0,\sin\psi,\cos\psi)\\
p_1 &= \frac {\sqrt{s_5}}{2}(1,\beta_5\sin\theta_2\sin\theta_1,\beta_5\sin\theta_2\cos\theta_1,\beta_5\cos\theta_1)\\
p_2 &= \frac {\sqrt{s_5}}{2}(1,-\beta_5\sin\theta_2\sin\theta_1,-\beta_5\sin\theta_2\cos\theta_1,-\beta_5\cos\theta_1)\\
k_2 &= (k_2^0,0,k_1\sin\psi,\abs{\vec q}+k_1^0\cos\psi)
\end{align}
where
\begin{align}
q_0 &= \frac {s+u'}{2\sqrt{s_5}},\quad
&\abs{\vec q} &= \frac {1}{2\sqrt{s_5}}\sqrt{(s+u')^2-4s_5q^2},\\
k_1^0 &= \frac{s_5-u'}{2\sqrt{s_5}},\quad
&\cos\psi &= \frac{2k_1^0q^0-s'}{2k_1^0\abs{\vec q}},\quad
&\beta_5 &= \sqrt{1-4m^2/s_5},\\
k_2^0 &= \frac{s-s_5}{2\sqrt{s_5}}
\end{align}
We further introduce $\rho^* = \frac{4m^2-q^2}{s-q^2} \leq x = \frac{s_5-q^2}{s-q^2} \leq 1$ and $-1\leq y\leq 1$ where $y$ is the cosine of the angle between $\vec q$ and $\vec k_2$ in the system with $\vec q+\vec k_2 = 0$. We then find\cite{Harris:1995tu}:
\begin{align}
dPS_3 &= \frac {T_\epsilon}{2\pi} \left(\frac{{s'}^2}{s}\right)^{1+\epsilon/2}(1-x)^{1+\epsilon}(1-y^2)^{\epsilon/2}dPS_2^{(5)}dy \sin^{\epsilon}(\theta_1)d\theta_1d\theta_2
\end{align}
with $0\leq\theta_1\leq\pi, 0\leq \theta_2\leq \pi, \rho^*\leq x\leq 1,-1\leq y\leq 1$ and
\begin{align}
S_\epsilon &= (4\pi)^{-2-\epsilon/2}\\
T_\epsilon &= \frac{\Gamma(1+\epsilon/2)}{\Gamma(1+\epsilon)}S_\epsilon = \frac 1 {16\pi^2}\left(1 + \frac {\epsilon} 2(\gamma_E - \ln(4\pi)) + O(\epsilon^2)\right)\\
dPS_2^{(5)} &= \frac{\beta_5 \sin(\theta_1)}{16\pi\Gamma(1+\epsilon/2)}\left(\frac{s_5\beta_5^2\sin^2(\theta_1)}{16\pi}\right)^{\epsilon/2}d\theta_1dx = dPS_2(s\rightarrow s_5) dx
\end{align}.
