\label{sec:NLO.g}
In next-to-leading order we have to consider the following process:
\begin{equation}
\Pggx(q) + \Pg(k_1) \rightarrow \PQ(p_1)+\PaQ(p_2) + \Pg(k_2)
\end{equation}
All contributing diagrams are depicted in figure \fxerror{do} and the result can be written as
\begin{equation}
\hat {\mathcal P}_{k}^{\Pgg,\mu\mu'}\hat {\mathcal P}_{k}^{\Pg}\sum_{j,j'}{\Md^{(1),g}_{j,\mu}\Md^{(1),g}_{j',\mu'}}^* = 8g^4\mu_D^{-2\epsilon}e^2e_H^2 N_C C_F\left( C_A R_{k,OK} + 2C_F R_{k,QED}\right)
\end{equation}
and it will depend on ten kinematical invariants:
\begin{align}
s &= (q+k_1)^2 &t_1 &=(k_1-p_2)^2-m^2 &u_1 &=(q-p_2)^2 -m^2\\
s_3 &= (k_2+p_2)^2-m^2 &s_4 &=(k_2+p_1)^2-m^2 &s_5 &= (p_1+p_2)^2=-u_5\\
t' &= (k_1-k_2)^2\\
u' &= (q-k_2)^2 &u_6 &=(k_1-p_1)^2-m^2 &u_7 &=(q-p_1)^2-m^2
\end{align}
from which only five are independent as can be seen from momentum conservation $k_1+q=p_1+p_2+k_2$ and $s,t_1,u_1$ match to their leading order definition.

The $2\rightarrow 3$ $n$-dimensional phase space is given by
\begin{align}
dPS_3 &= \!\int\!\!\frac{d^{n}p_1}{(2\pi)^{n-1}}\frac{d^{n}p_2}{(2\pi)^{n-1}}\frac{d^{n}k_2}{(2\pi)^{n-1}}(2\pi)^n\delta^{(n)}(k_1+q-p_1-p_2-k_2) \nonumber\\
 &\hspace{50pt}\Theta(p_{1,0})\delta(p_1^2-m^2)\Theta(p_{2,0})\delta(p_2^2-m^2)\Theta(k_{2,0})\delta(k_2^2) \label{eq:PS3}
\end{align}
This can be solved by writing eq. (\ref{eq:PS3}) as product of a $2\rightarrow 2$ decay and a subsequent $1\rightarrow 2$ decay\cite{PhysRevD4054}. We find
\begin{align}
dPS_3 &= \frac 1 {(4\pi)^n\Gamma(n-3)s'} \frac{s_4^{n-3}}{(s_4+m^2)^{n/2-1}}\left(\frac{(t_1u_1'-s'm^2)s' - q^2t_1^2}{s'^2}\right)^{(n-4)/2}\! dt_1 du_1 d\Omega_n d\hat{\mathcal I}\\
 &=h_3(n)\,dt_1 du_1 d\Omega_n d\hat{\mathcal I}
\end{align}
with $d\Omega_n = \sin^{n-3}(\theta_1)d\theta_1\sin^{n-4}(\theta_2)d\theta_2$ and $d\hat{\mathcal I}$ taking care of all occuring hat momenta:
\begin{align}
d\hat{\mathcal I} &= \frac 1 {B(1/2,(n-4)/2)}\frac{x^{(n-6)/2}}{\sqrt{1-x}}dx &\text{with}\,x &= \hat p_1^2/\hat p_{1,max}\\
\hat p_{1,max} &= \frac{s_4^2}{4(s_4+m^2)}\sin^2(\theta_1)\sin^2(\theta_2)
\end{align}
\begin{equation}
\Rightarrow \int\!d\hat{\mathcal I} = 1 \qquad \int\!d\hat{\mathcal I}\,\hat p_1^2 = \epsilon \hat p_{1,max} + O(\epsilon^2)
\end{equation}
The needed phase space integrals for $\theta_1$ and $\theta_2$ can be found in \cite{PhysRevD4054} and \cite{Bojak:2000eu}. We find for the difference to the $2\rightarrow 2$ phase space
\begin{align}
\frac{h_3(4+\epsilon)}{h_2(4+\epsilon)} &= \frac{S_\epsilon}{2\pi} \frac{\Gamma(1+\epsilon/2)}{\Gamma(1+\epsilon)} \frac{s_4^{1+\epsilon}}{(s_4+m^2)^{1+\epsilon/2}}\\
 &= \frac{C_\epsilon}{2\pi}\left(1-\frac 3 8 \zeta(2)\epsilon^2\right)\frac{s_4^{1+\epsilon}}{(s_4+m^2)^{1+\epsilon/2}} + O(\epsilon^3)
\end{align}
where $\zeta$ is Riemanns zeta function.
\fxerror{introduce psLogs? in appendix?}

Again when integrating the phase space angles the expressions become quite lengthy, but the (collinear) pole parts are compact:
\begin{align}
\frac{s_4}{4\pi(s_4+m^2)}\int\!d\Omega_n d\hat{\mathcal I}\,C_A R_{k,OK} &=-\frac 1 {u_1}B_{k,QED}\left(\!\begin{array}{l}s'\rightarrow x_1s'\\t_1\rightarrow x_1 t_1\end{array}\!\!\right) P^H_{k,\Pg\Pg}(x_1)\frac 2 \epsilon + O(\epsilon^0) \label{eq:ROKPoles}
\end{align}
with $x_1 = -u_1/(s'+t_1)$ and the hard part of the Altarelli-Parisi splitting functions $P^H_{k,\Pg\Pg}$\cite{Altarelli:1977zs,Vogelsang:1995vh}:
\begin{align}
P^H_{G,\Pg\Pg}(x) = P^H_{L,\Pg\Pg}(x) &= C_A\left(\frac 2 {1-x} + \frac 2 x - 4 + 2x - 2x^2\right)\\
P^H_{P,\Pg\Pg}(x) &= C_A\left(\frac 2 {1-x} - 4x + 2\right)
\end{align}
The $R_{k,QED}$ do not contain poles. \fxerror{shift to factorization?}

The double differential partonic cross section is given by
\begin{align}
{s'}^2\frac{d^2\sigma_{k,\Pg}^{(1),R}(s',t_1,u_1,q^2)}{dt_1du_1} &= 2^7\alpha\alpha_s^2 e_H^2 K_{\Pg\Pgg}N_CC_F E_k(\epsilon) b_k(\epsilon)\frac{\pi^3 S_\epsilon^2}{\Gamma(1+\epsilon)} \frac{s_4}{s_4+m^2}  \nonumber\\
 &\hspace{20pt} \left(\frac{(t_1u_1'-s'm^2)s' - q^2t_1^2}{m^2{s'}^2}\right)^{\epsilon/2} \left(\frac{s_4^2}{m^2(s_4+m^2)}\right)^{\epsilon/2} \left(\frac{\mu_D^2}{m^2}\right)^{-\epsilon} \nonumber\\
 &\hspace{20pt} \int\!d\Omega_n d\hat{\mathcal I}\,\left(C_A R_{k,OK} + 2C_F R_{k,QED}\right)
\end{align}

From the above expression we can obtain the soft limit $k_2\rightarrow 0$ and seperate their calculations:
\begin{equation}
\lim_{k_2\rightarrow 0}\left(C_A R_{k,OK} + 2C_F R_{k,QED}\right) = \left(C_A S_{k,OK} + 2C_F S_{k,QED}\right) + O(1/s_4,1/s_3,1/t')
\end{equation}
\begin{align}
S_{k,OK}  &= 2\left(\frac{t_1}{t's_3} + \frac{u_1}{t's_4}-\frac{s-2m^2}{s_3s_4}\right)B_{k,QED}\\
S_{k,QED} &= 2\left(\frac{s-2m^2}{s_3s_4} - \frac{m^2}{s_3^2} - \frac{m^2}{s_4^2}\right)B_{k,QED}
\end{align}
Note that the einkonal factors multiplying the Born functions $B_{k,QED}$ neither depend on $q^2$ nor on the projection $k$. We can then split the phase space by introducing an infrared cut-off $\Delta$ and distinguish then between soft $s_4\leq \Delta$ and hard $s_4>\Delta$ contributions. Let $\mathcal R(s_4)$ be a function with a soft pole $s_4^{-1+\epsilon}\mathcal S(s_4)$ and a finite part $\mathcal F(s_4)$, we then find \cite{Bojak:2000eu}:
\begin{align}
\int\limits_0^{s_{4,max}} \!\!\mathcal R(s_4) &= \int\limits_0^{s_{4,max}} \!\!\left(s_4^{-1+\epsilon}\mathcal S(s_4) + \mathcal F(s_4)\right)\\
 &\simeq \frac{\Delta^\epsilon}{\epsilon}\mathcal S(0) + \int\limits_\Delta^{s_{4,max}}\!\!\mathcal R(s_4)
\end{align}
This expansion is valid for $\Delta$ being small, i.e. smaller then any leading order scale or $m^2$; a typical choice is $\Delta/m^2 \sim 10^{-6}$. We then find
\begin{align}
&\frac{s_4^2}{4\pi(s_4+m^2)}\left(1-\frac 3 8 \zeta(2)\epsilon^2\right)\int\!d\Omega_n d\hat{\mathcal I}\,S_{k,QED}\nonumber\\
 &= B_{k,QED}\left[-\frac 2 \epsilon\left(1+\frac{s-2m^2}{s\beta}\ln(\chi)\right)
+1 - \frac{s-m^2}{s\beta}\left(\ln(\chi)\left(1+\ln(\chi)\right) + \DiLog(1-\chi^2)\right)
\right] \\
&\frac{s_4^2}{4\pi(s_4+m^2)}\left(1-\frac 3 8 \zeta(2)\epsilon^2\right)\int\!d\Omega_n d\hat{\mathcal I}\,S_{k,OK} \nonumber\\
 &=B_{k,QED}\left[\frac 4 {\epsilon^2} + \frac 2 \epsilon\left(\ln(t_1/u_1)+\frac{s-2m^2}{s\beta}\ln(\chi)\right)-\ln^2(\chi) - \frac 3 2 \zeta(2)+\frac 1 2\ln^2(t_1/(u_1\chi))\right.\nonumber\\
 &\hspace{20pt}\left. +\DiLog(1-t_1/(u_1\chi)) - \DiLog(1-u_1/(t_1\chi)) + \frac{s-2m^2}{s\beta}\left(
\DiLog(1-\chi^2)+\ln^2(\chi)
\right) \right]
\end{align}
(Note the mistyped sign of $\ln(\chi)^2$ in \cite[eq. (3.25)]{Laenen1993162}) The additional factors on the left hand sides originate from the difference between the $2\rightarrow 3$ phasespace of $R_{k}$ and the $2\rightarrow 2$ phasespace needed for $S_k$.

The double differential partonic cross section is given by
\begin{align}
&{s'}^2\frac{d^2\sigma_{k,\Pg}^{(1),S}(s',t_1,u_1,q^2)}{dt_1du_1} \nonumber\\
 &= 2^8\alpha\alpha_s^2 e_H^2 K_{\Pg\Pgg}N_CC_F E_k(\epsilon) b_k(\epsilon)\delta(s'+t_1+u_1)\frac{\pi^4 S_\epsilon}{\Gamma(1+\epsilon/2)} \nonumber\\
 &\hspace{20pt} \left(\frac{(t_1u_1'-s'm^2)s' - q^2t_1^2}{m^2{s'}^2}\right)^{\epsilon/2} C_\epsilon \left(\frac{\mu_D^2}{m^2}\right)^{-\epsilon} \left(\frac \Delta {m^2}\right)^\epsilon \nonumber\\
 &\hspace{20pt}\frac{s_4^2}{4\pi(s_4+m^2)}\left(1-\frac 3 8 \zeta(2)\epsilon^2\right)\int\!d\Omega_n d\hat{\mathcal I}\,\left(C_A S_{k,OK} + 2C_F S_{k,QED}\right)
\end{align}

The results agree in the photo-production limit ($q^2\rightarrow 0$) with \cite{Bojak:1998zm}.
